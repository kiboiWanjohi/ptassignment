
\documentclass{assignment}
\usepackage[pdftex]{graphicx} % FIGURAS
\usepackage{xcolor}
\definecolor{LightGray}{gray}{0.95}
\usepackage{fancyvrb, minted} % CÓDIGO
\usepackage[letterpaper, margin = 2.5cm]{geometry} % TAMAÑO DE PÁGINA Y MÁRGENES
\usepackage[T1]{fontenc} % Importante para acentos automáticos y símbolos de escritura
\usepackage[spanish, mexico]{babel} % Importante para Español
\usepackage{amsmath, amsfonts, amssymb} % Ecuaciones, caracteres y símbolos especiales
\usepackage{hyperref, url}  % Links y Hyperlinks en el documento
\usepackage{fancyhdr}

%-----------------------------------------ETIQUETAS--------------------------------------------

\student{Sammy Wanjohi Kiboi}                             % NOMBRE
\semester{1.1}                                % SEMESTRE (202X A/B)
\date{\today}                                   % Fecha (Modifica a DD/MM/AAAA)

\courselabel{RadioBiology}          % CLAVE Y MATERIA
\exercisesheet{Tarea X}{Título de la tarea}     % NÚMERO Y TÍTULO DE LA TAREA

\school{Licenciatura en Física, CUCEI}          % CARERA (Física, la mejor carrera)
\university{Universidad de Guadalajara}         % LA PODEROSÍSIMA

%%%%%%%%%%%%%%%%%%%%%%%%%%%%%%%%%%%%%%%%%%-DOCUMENTO-%%%%%%%%%%%%%%%%%%%%%%%%%%%%%%%%%%%%%%%%%%%%

\begin{document}

%-----------------------------------------------------------------------------------------------
\begin{problem}

\section{Section A}

\noindent Explain a) Linear Energy Transfer b) Oxygen Enhancement Ratio c)Radiation weighing factor d) Dose fractionation
\noindent Explain the Target theory
\noindent State the law of Bergonie and Tribondeau.
\noindent Explain radiation dóse-response relationships.
\noindent Distinguish between in vitro and in vivo?
\noindent QUESTION 2 (20 MARKS)
Give examples of three agents that enhance the effects of radiation and three radioprotective agents.
\noindent What is the difference between catabolism and anabolism?
\noindent Discuss three effects of in vitro irradiation of macromolecules.
\noindent Explain the effects of radiation on the DNA.
\noindent Write down the chemical reactions involved in the radiolysis of water.
\noindent QUESTION 4 (20 MARKS)
Under fully oxygenated conditions, \begin{equation}90%
\end{equation} of human cells in culture will be killed by 1.5 Gy X-rays. If
cells are made anoxic, the dose required for \begin{equation}
    90%
\end{equation} lethality is 4 Gy. What is the OER (Oxygen enhancement ratio)?
\noindent 
Describe the physical factors and biologic factors that affect radiation response.
\section{Section B}
\noindent 
Describe the physical factors and biologic factors that affect radiation response.
\noindent Describe the effects of in vivo irradiation.
\noindent Describe the principles of target theory. 
\noindent 
Discuss the kinetics of cell survival after irradiation.
\subsection{Solution}
\noindent Define direct effect and indirect effect and identify the importance of each.
\subsection{solution}
\noindent What type of interaction with tissue results in a hit?
\begin{equation}\label{eq:TeoremaFundamental}
    \int_a^b f(t)dt = F(a)-F(b).
\end{equation}
\subsection{Primera parte}
\noindent Explain the following terms (a) Latent period (b) \begin{equation}
    LD_{50/60}
\end{equation}
\noindent Describe the following Acute Radiation Lethality (Gastrointestinal death (GI) under the following:
(Approximate Dose; Gy, Mean Survival Time; Days, and the Clinical Signs and Symptoms)
\noindent Explain the prodromomal period and the Latent period that lead to acute radiation lethality.

\noindent A los profesores\footnote{Como a Jack el destripador} les encanta dividir un problema en varias partes.

\subsubsection*{Unordered Questions}
1.1 Explain the following terms. [8 marks]
(a) Anabolism
(c) Radiation weighting factor
State the law of Bergonie and Tribondeau. [4 marks]
1.2
1.3
1.4
1.5
1.6
1.7
2.1
2.2
2.3
3.1
3.2
3.3
3.4
4.1
4.2
4.3
QUESTION 1 (30 MARKS)
5.2
5.3
(b) Free radical
(d) Dose fractionation
Under fully oxygenated conditions, 90% of human cells in culture will be killed by 1.5
Gy x-rays. If cells are made anoxic, the dose required for 90% lethality is 4 Gy. What is
the OER? [3 marks]
Explain how age affect the radiosensitivity of tissue? [3 marks]
Explain the Target theory. [3 marks]
Explain the irradiation of macromolecules. [6 marks]
In radiotherapy, which factors influence the selection of appropriate dose/fraction, total
dose, and fractionation? [3 marks]
QUESTION 2 (20 MARKS)
Explain how ionizing radiation affect an atom within a large molecule? [4 marks]
Describe five types of radiation dose-response relationships. [8 marks]
Describe the physical factors and biologic factors that affect radiation response. [8 marks]
QUESTION 3 (20 MARKS)
Describe the relationship between RBE and OER. [3 marks]
Discuss the direct and indirect effects of radiation. [3 marks]
Explain the effects of radiation on the deoxyribonucleic acid DNA. [8 marks]
Write down the chemical reactions involved in the radiolysis of water. [6 marks]
QUESTION 4 (20 MARKS)
Describe the effects of in vivo irradiation. [6 marks]
Discuss the kinetics of cell survival after irradiation. [10 marks]
The D37 of a cellular species that follows the single-target, single-hit model is 2 Gy. What
percentage of cells will survive 6 Gy? [4 marks]
5.1 Explain the following terms. [4 marks]
(a) Latent period
(b) LD 50/60
QUESTION 5 (20 MARKS)
Discuss the "four Rs" of radiation biology that can enhance the responses of normal and
cancerous cells in fractionated therapy. [10 marks]
Describe the features of deterministic and stochastic effects of radiation exposure.
[6 marks]
r
6

\section{Unordered Questions}
\noindent 
QUESTION 1 (30 MARKS)
1.1
1.2
1.3
1.4
1.5
1.6
2.3
2.4
QUESTION 2 (20 MARKS)
3.1
3.2
2.1 State the law of Bergonie and Tribondeau [4 marks]
2.2
Explain how ionizing radiation affect an atom within a large molecule? [4 marks]
Discuss the direct and indirect effects of radiation [4 marks]
Describe five types of radiation dose-response relationships. [8 marks]
3.3
Explain the following terms: [4 marks]
(b) Catabolism,
(a) Free radical,
(d) Linear energy transfer,
QUESTION 3 (20 MARKS)
4.1
4.2
Explain how age affect the radiosensitivity of tissue? [5 marks]
Explain the difference between these terms transcription, transfer, and translation when
applied to molecular genetics. [6 marks]
4.3
Describe the physical factors and biologic factors that affect radiation response. [6 marks]
Under fully oxygenated conditions, 90% of human cells in culture will be killed by 1.5
Gyt x-rays. If cells are made anoxic, the dose required for 90% lethality is 4 Gyt. What is
the OER? [3]
(c) Threshold dose,
Explain the irradiation of macromolecules. [6 marks]
QUESTION 4 (20 MARKS)
Describe the relationship between RBE and OER [6 marks]
Approximately 8 Gyt of 220 kVp x-rays is necessary to produce death in a lizard. Cobalt-
60 gamma rays have a lower LET than 220 kVp x-rays; therefore, 9.4 Gyt is required for
lizard lethality. What is the RBE of 60CO compared with 220 kVp? [5 marks]
Discuss the radiation effects on the deoxyribonucleic acid (DNA). [9 marks]
Discuss the principles of target theory in radiobiology. [6 marks]
Explain cell survival curves [10 marks]
The D37 of a cellular species that follows the single-target, single-hit model is 1.5 Gyt.
What percentage of cells will survive 4.5 Gyt? [4 marks]
in
CO
d
S
QUESTION 5 (20 MARKS)
5.1 State the clinical signs and symptoms of hematologic and Gastrointestinal syndromes as
well as the Approximate Dose (Gyt) required for the signs to appear. [6 marks]
5.2
5.3
Discuss local tissue damage after high-dose irradiation. [8 marks]
Describe the features of deterministic and stochastic effects of radiation exposure.
[6 marks]
\noindent Discute y bosqueja la ruta o método que usarás para solucionar el problema. Vincúlalo con lo visto en clase o con algún libro, que puedes citar así: \cite{GustavoLopez}

\begin{enumerate}
    \item Si quieres

    \item Puedes explicar tu solución

    \item Por partes
\end{enumerate}

\noindent Recuerda los distintos tipos de ecuaciones que puedes agregar a tu documento:

\begin{itemize}
    \item Para una ecuación sencilla utilizamos el método \texttt{equation}, que encerramos con \texttt{boxed}:
    \begin{equation}\label{eq:Sencilla}
        \boxed{             % Así la encerramos para destacarla
        e^{\pi i} + 1 = 0
        }
    \end{equation}

    \item Podemos referenciar la ecuación anterior con el método \texttt{ref}: (Ec. \ref{eq:Sencilla}). Para una ecuación larga utilizamos el método \texttt{multiline}. Además, podemos agregar un asterisco (*) si no queremos numerarla:
    \begin{multline*}
        p(x) = 3x^6 + 14x^5y + 590x^4y^2 + 19x^3y^3 - 12x^2y^4 + 12xy^5 + 2y^6 - a^3b^3\\ 
            - 12x^2y^4 - 12xy^5 + 2y^6 - a^3b^3 + 3x^6 + 14x^5y - 590x^4y^2 + 19x^3y^3
    \end{multline*}

    \item Para dividir la ecuación larga en dos líneas o más (para por ejemplo mostrar un proceso de simplificación o de operaciones) utilizamos el comando \texttt{split}:
    \begin{equation} \label{eq1}
        \begin{split}
            A & = \frac{\pi r^2}{2} \\
             & = \frac{1}{2} \pi r^2
        \end{split}
    \end{equation}

    \item Para mostrar un \textit{sistema} de ecuaciones, utilizamos el ambiente \texttt{align}: 
    \begin{align*} 
        2x - 5y &=  8 \\ 
        3x + 9y &=  -12
    \end{align*}
    Que también podemos aplicar a sistemas más complejos:
    \begin{align*}
        x&=y           &  w &=z              &  a&=b+c\\
        2x&=-y         &  3w&=\frac{1}{2}z   &  a&=b\\
        -4 + 5x&=2+y   &  w+2&=-1+w          &  ab&=cb
    \end{align*}    

    \item Para \textit{agrupar} ecuaciones consecutivamente sin alineamiento, utilizamos \texttt{gather}:
    \begin{gather*} 
        2x - 5y =  8 \\ 
        3x^2 + 9y =  3a + c
    \end{gather*}
\end{itemize}

\noindent Muchas veces en nuestras tareas incorporamos imágenes como a la que hacemos referencia aquí (Figura \ref{fig:yo_en_la_vida}):
\begin{figure}[ht] % Especificamos ubicacion (h = aquí mero)
    \centering
    \includegraphics[width=0.5\textwidth]{Figuras/Cotorros.jpg} % Ancho especificado
    \caption{Esquema de las relaciones entre los componentes del núcleo atómico}
    \label{fig:yo_en_la_vida}
\end{figure}

\noindent Finalmente, recuerda utilizar el método \texttt{minted} para escribir el código que incluyas en tu tarea, y \texttt{verbatim} para el \textit{pseudo}-código\footnote{Muchas veces es mejor incluír \textit{ambos} para darte a entender con tu profesor}:

\begin{minted}[frame=lines, linenos, bgcolor=LightGray]{python}
    def funcion(argumento1, argumento2):
        """
        Explico que hace la funcón
        """
        if condicion == True:
            alumno = graduado
        else:
            alumno = baja
\end{minted}

\end{problem}
%---------------------------------------------------------------------------------------------
% \begin{problem}

% \section{Segundo Problema}

% \noindent
    
% \end{problem}
%---------------------------------------------------------------------------------------------



%--------------------------------------BIBLIOGRAFIA-------------------------------------------

\newpage

\nocite{*} % Agrega las referencias aunque no las hayas citado directamente

\bibliographystyle{unsrt}    % ESTILO DE BIBLIOGRAFÍA (Recomendados: abbrv, ieeetr, apalike, unsrt)
\bibliography{refs}     % REFERENCIAS EN ARCHIVO SEPARADO


\end{document}

