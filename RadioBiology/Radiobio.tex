%%%%%%%%%%%%%%%%%%%%%%%%%%%%%%%%%%%%%%%%%%%%%%%%%%%%%%%%%%%%%%%
%
% Welcome to Overleaf --- just edit your LaTeX on the left,
% and we'll compile it for you on the right. If you give
% someone the link to this page, they can edit at the same
% time. See the help menu above for more info. Enjoy!
%
%%%%%%%%%%%%%%%%%%%%%%%%%%%%%%%%%%%%%%%%%%%%%%%%%%%%%%%%%%%%%%%
%\title{Math 453 HW 1}
\documentclass[addpoints]{exam}

\usepackage{amsmath,enumitem,wrapfig}
\usepackage{tikz}

\newcommand{\StudentName}{Sammy Wanjohi Kiboi}
\newcommand{\AssignmentName}{RadioBiology}

\pagestyle{headandfoot}
\runningheadrule
\firstpageheadrule
\firstpageheader{Math 453}{\StudentName}{\AssignmentName}
\runningheader{Math 453}{\StudentName}{\AssignmentName}
\firstpagefooter{}{}{}
\runningfooter{}{}{}

\printanswers

\begin{document}


Organize your work and show any work that you want credit for. Use full sentences where possible.

\begin{questions}

\question \textbf{Question 1 30 Marks}
\begin{parts}
\part
\textbf{For this section most questions and answers are in notes so skip}
1.a) Explain the following terms
(a) Linear energy transfer (b) Oxygen enhancement ratio

(c) Radiation weighting factor (d) Dose fractionation

\begin{solution}
a) the amount
of energy transferred per unit length,
\newline b)It is defined as the ratio of radiation doses that produce the same biological effect in hypoxic compared to aerobic (well-oxygenated) conditions: \textbf{notes}
\end{solution}

\part 1.5) Explain the Target theory
\begin{solution}\\
Suppose an object (say a macromolecule) is irradiated.
Assume that the radiation deposits one or more primary ionizations (i.e., ion clusters) within the molecule. Assume that
the molecule has a particular function within our cells and
that this function is destroyed only if the ion cluster destroys
one particular part of the molecule and that the molecule still
works equally well if the ion cluster damages any other part.
The sensitive area inside the molecule is then called the target
\end{solution}

\end{parts}



\question \textbf{Question 3 20 marks}

\begin{parts}
\part What is the difference between catabolism and anabolism?
\begin{solution}
Catabolism refers to the breakdown of complex molecules into simpler ones, releasing energy in the process. Anabolism, on the other hand, refers to the synthesis of complex molecules from simpler ones, requiring energy in the process. In summary, catabolism breaks down molecules and releases energy, while anabolism builds up molecules and requires energy
\end{solution}


\part Effects of in-vivo and in-vitro radiation of macromolecules
\begin{solution}
Radiation exposure can damage macromolecules both inside living organisms (in vivo) as well as in isolated cells and tissues (in vitro). Some key effects are:

In vivo radiation effects:
- DNA damage from direct ionization or free radicals leading to cell death or mutations
- Oxidation of proteins altering structure/function, causing protein-DNA crosslinks
- Lipid peroxidation disrupting membrane structure and cellular pathways
- Tissue inflammation, fibrosis, necrosis at high doses
- Carcinogenesis and teratogenesis if germ cells impacted

In vitro radiation effects:
- Double and single strand DNA breaks depending on dose
- DNA base hydroxylation altering base pairing
- Protein unfolding, aggregation through impaired secondary interactions
- Oxidative damage to side chains of amino acids 
- Membrane rupture and lipid breakdown products
- Loss of protein synthesis capabilities
- Loss in viability for cell cultures
- Mutations in DNA replicated after irradiation exposure

Overall the most severe molecular effects result from radical formation and oxidative damage to key macromolecules. Multiple interrelated biomolecular pathways regulating metabolism, immunity, signaling etc get disrupted. The sensitivity depends on dose rate, exposure levels and antioxidant capacities inside cells providing defense mechanisms.
\end{solution}

\part 4.3) Discuss the kinetics of cell survival after irradiation.

\begin{solution}
Shoulder region - At low radiation doses, a shoulder is observed where cells display greater survival than expected linearly. This indicates cellular repair mechanisms mitigate some damage.
\newline
Dose-Response Relationship:

Linear Component (
α): Represents cell killing at low doses. It is proportional to the number of radiation-induced DNA double-strand breaks.
Quadratic Component (
β): Represents cell killing at higher doses. It is related to the repair of sublethal damage and interactions between different DNA lesions.
\newline
Survival Curves:

Linear-Quadratic Model: The relationship between cell survival and radiation dose is often modeled using the linear-quadratic (LQ) equation: \(S(D) = e^{-(\alpha D + \beta D^2)}\) In this equation, 
S(D) is the surviving fraction of cells, 
D is the radiation dose, and 
α and 
β are constants that characterize the linear and quadratic components of the curve.
\newline Radiosensitivity:

Highly Radiosensitive Cells: Cells with a high capacity for repair and a prominent shoulder on the survival curve.
Radioresistant Cells: Cells with less capacity for repair and a smaller or no shoulder.
\end{solution}

\part 5.a) Explain the following terms (a) Latent period (b) LD 50/60

\begin{solution}
\textbf{Read on latency on Radiobiology Textbook page 64 - 67}
\end{solution}
\end{parts}


\question \textbf{M3}
\begin{parts}
\part
Find the flaw in the following argument.
\begin{quote}
To solve $x(x+4)=x(2x-8)$ we divide both sides by $x$ (or apply Theorem 1.11) to get $x+4=2x-8$. Subtract $(x-8)$ from both sides to obtain $12=x$, so the solution is $x=12$. 
\end{quote}
\begin{solution}
x could be 0, so we can't divide both sides by x.
\end{solution}

\part 
Find the flaw in the following argument.
\begin{quote}
To solve $x(x-4)=12$ we factor the left-hand side and set the factors equal to zero $x=0$ and $x-4=0$ and conclude that $x=0,4$.  
\end{quote}
\begin{solution}
$x(x-4)=12$ should be $x^2-4x-12=0$ by distributive law\\
then, $(x-6)(x+2)=0$
Therefore, $x=6, -2$
\end{solution}
\end{parts}

\question \textbf{M4}
\begin{parts}
\part Work Exercise 1 from Investigation 1 (uniqueness of additive inverses).
\begin{solution}
If some integer a has two additive inverses, which are b and c, then we can write $a+b=0$ and $a+c=0$.\\
Then, $a+b=a+c=0$.\\
Since a is integer, we can say $b=c$
\end{solution}


\part Work Exercise 2 from Investigation 1 (additive cancellation).
\begin{solution}
Given that $a+b=a+c$, where a, b, and c are in integers Z.\\
$(-a)=(-a)$ ...(-a) exists by additive inverse.\\
Now, we can add $(-a)$ from both sides,\\
Then, $(a+(-a))+b=(a+(-a))+c$, by associative law,\\
$0+b=0+c$\\
Therefore, $b=c$ by additive identity.
\end{solution}
\end{parts}





\end{questions}
\end{document}
