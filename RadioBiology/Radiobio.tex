
\documentclass{assignment}
\usepackage[pdftex]{graphicx} % FIGURAS
\usepackage{xcolor}
\definecolor{LightGray}{gray}{0.95}
\usepackage{fancyvrb, minted} % CÓDIGO
\usepackage[letterpaper, margin = 2.5cm]{geometry} % TAMAÑO DE PÁGINA Y MÁRGENES
\usepackage[T1]{fontenc} % Importante para acentos automáticos y símbolos de escritura
\usepackage[spanish, mexico]{babel} % Importante para Español
\usepackage{amsmath, amsfonts, amssymb} % Ecuaciones, caracteres y símbolos especiales
\usepackage{hyperref, url}  % Links y Hyperlinks en el documento
\usepackage{fancyhdr}

%-----------------------------------------ETIQUETAS--------------------------------------------

\student{Sammy Wanjohi Kiboi}                             % NOMBRE
\semester{1.1}                                % SEMESTRE (202X A/B)
\date{\today}                                   % Fecha (Modifica a DD/MM/AAAA)

\courselabel{RadioBiology}          % CLAVE Y MATERIA
\exercisesheet{Tarea X}{Título de la tarea}     % NÚMERO Y TÍTULO DE LA TAREA

\school{Licenciatura en Física, CUCEI}          % CARERA (Física, la mejor carrera)
\university{Universidad de Guadalajara}         % LA PODEROSÍSIMA

%%%%%%%%%%%%%%%%%%%%%%%%%%%%%%%%%%%%%%%%%%-DOCUMENTO-%%%%%%%%%%%%%%%%%%%%%%%%%%%%%%%%%%%%%%%%%%%%

\begin{document}

%-----------------------------------------------------------------------------------------------
\begin{problem}

\section{Section A}
\noindent 1.1. Explain a) Linear Energy Transfer b) Oxygen Enhancement Ratio c)Radiation weighing factor d) Dose fractionation[8 marks]
\noindent 1.2. Explain the Target theory [3 marks]
\noindent 1.3. State the law of Bergonie and Tribondeau. [4 marks]
\noindent 1.4. Explain radiation dóse-response relationships. [6 marks]
\noindent 1.5. Distinguish between in vitro and in vivo? [3 marks]
\noindent 1.6. What type of interaction with tissue results in a hit? [3 marks]

\subsection{Question 2}
\noindent 2.1 Give examples of three agents that enhance the effects of radiation and three radioprotective agents.[6 marks]
\noindent 
2.2. Under fully oxygenated conditions, \begin{equation}90%
\end{equation} of human cells in culture will be killed by 1.5 Gy X-rays. If
cells are made anoxic, the dose required for \begin{equation}
    90%
\end{equation} lethality is 4 Gy. What is the OER (Oxygen enhancement ratio)?[4marks]
\noindent 
2.3. Describe the physical factors and biologic factors that affect radiation response.[10 marks]
\subsection*{Question 3}
\noindent 3.1.  What is the difference between catabolism and anabolism?[2 marks]
\noindent 3.2. Discuss three effects of in vitro irradiation of macromolecules.[8 marks]
\noindent 3.3. Explain the effects of radiation on the DNA.[4 marks]
\noindent 3.4. Write down the chemical reactions involved in the radiolysis of water.[6 marks]
\subsection*{Question 4 (20 marks)} 

\noindent 4.1. Describe the effects of in vivo irradiation.[6 marks]
\noindent 4.2. Describe the principles of target theory. [4 marks]
\noindent 4.3. Discuss the kinetics of cell survival after irradiation.[10 marks]
\subsection{Solution}
\subsection*{Question 5 20 marks}
\noindent 5.1. Explain the following terms (a) Latent period (b) \begin{equation}
    LD_{50/60}
\end{equation}[4 marks]

\noindent 5.2. Describe the following Acute Radiation Lethality (Gastrointestinal death (GI) under the following:
(Approximate Dose; Gy, Mean Survival Time; Days, and the Clinical Signs and Symptoms) [10 marks]
\noindent 5.3. Explain the prodromomal period and the Latent period that lead to acute radiation lethality. [6 marks]

\noindent 5.4. Define direct effect and indirect effect and identify the importance of each. [6 marks]
\subsection{solution}

\begin{equation}\label{eq:TeoremaFundamental}
    \int_a^b f(t)dt = F(a)-F(b).
\end{equation}
\subsection{Primera parte}

\noindent A los profesores\footnote{Como a Jack el destripador} les encanta dividir un problema en varias partes.
\section*{Question 1 30 marks}
\subsubsection*{Unordered Questions}
\noindent 1.1 Explain the following terms. [8 marks]
(a) Anabolism (b) Free radical
(c) Radiation weighting factor (d) Dose fractionation
\noindent 1.2. State the law of Bergonie and Tribondeau. [4 marks]
\noindent Under fully oxygenated conditions, 90 % of human cells in culture will be killed by 1.5
Gy x-rays. If cells are made anoxic, the dose required for 90% lethality is 4 Gy. What is
the OER? [3 marks]
\noindent Explain how age affect the radiosensitivity of tissue? [3 marks]
\noindent Explain the Target theory. [3 marks]
\noindent Explain the irradiation of macromolecules. [6 marks]
\noindent In radiotherapy, which factors influence the selection of appropriate dose/fraction, total
dose, and fractionation? [3 marks]
\subsection*{QUESTION 2 (20 MARKS)}
\noindent 2.1. Explain how ionizing radiation affect an atom within a large molecule? [4 marks]
\noindent 2.2. Describe five types of radiation dose-response relationships. [8 marks]
\noindent 2.3. Describe the physical factors and biologic factors that affect radiation response. [8 marks]
\subsection{QUESTION 3 (20 MARKS)}
\noindent 3.1 Describe the relationship between RBE and OER. [3 marks]
\noindent 3.2. Discuss the direct and indirect effects of radiation. [3 marks]
\noindent 3.4. Explain the effects of radiation on the deoxyribonucleic acid DNA. [8 marks]
\noindent 3.5. Write down the chemical reactions involved in the radiolysis of water. [6 marks]
\subsection{QUESTION 4 (20 MARKS)}
\noindent 4.1. Describe the effects of in vivo irradiation. [6 marks]
\noindent 4.2. Discuss the kinetics of cell survival after irradiation. [10 marks]
\noindent 4.3. The D37 of a cellular species that follows the single-target, single-hit model is 2 Gy. What
percentage of cells will survive 6 Gy? [4 marks]
\subsection{Question 20 MARKS}
\noindent 5.1 Explain the following terms. [4 marks]
(a) Latent period
(b) LD 50/60

\noindent 5.2. Discuss the "four Rs" of radiation biology that can enhance the responses of normal and
cancerous cells in fractionated therapy. [10 marks]
\noindent 5.3. Describe the features of deterministic and stochastic effects of radiation exposure.
[6 marks]
\section{Unordered Questions}
\noindent 
\section{ANAZA ONE }
\subsection{QUESTION 1 (30 MARKS)}
\noindent Explain the following terms: [4 marks]
(a) Free radical,
(b) Catabolism,
(c) Threshold dose,
(d) Linear energy transfer,
[4 marks]
\noindent 1.2. Explain how age affect the radiosensitivity of tissue? [5 marks]
\noindent 1.3. Explain the difference between these terms transcription, transfer, and translation when
applied to molecular genetics. [6 marks]
\noindent 1.4. Describe the physical factors and biologic factors that affect radiation response. [6 marks]
\noindent 1.5. Under fully oxygenated conditions, 90% of human cells in culture will be killed by 1.5
Gyt x-rays. If cells are made anoxic, the dose required for 90% lethality is 4 Gyt. What is
the OER? [3 marks]
\noindent 1.6. Explain the irradiation of macromolecules. [6 marks]
\subsection{Question 2}
\noindent 2.1 State the law of Bergonie and Tribondeau [4 marks]
\noindent 2.2. Explain how ionizing radiation affect an atom within a large molecule? [4 marks]
\noindent 2.3. Discuss the direct and indirect effects of radiation [4 marks]
\noindent 2.4. Describe five types of radiation dose-response relationships. [8 marks]

\subsection{QUESTION 3 (20 MARKS)}
\noindent 3.1. Describe the relationship between RBE and OER [6 marks]
\noindent 3.2. 
Approximately 8 Gyt of 220 kVp x-rays is necessary to produce death in a lizard. Cobalt-
60 gamma rays have a lower LET than 220 kVp x-rays; therefore, 9.4 Gyt is required for
lizard lethality. What is the RBE of 60CO compared with 220 kVp? [5 marks]
\noindent 3.3. 
Discuss the radiation effects on the deoxyribonucleic acid (DNA). [9 marks]
\subsection{QUESTION 4 (20 MARKS) 
\noindent 4.1. Discuss the principles of target theory in radiobiology. [6 marks]
\noindent 4.2. Explain cell survival curves [10 marks]
\noindent 4.3. The D37 of a cellular species that follows the single-target, single-hit model is 1.5 Gyt.
What percentage of cells will survive 4.5 Gyt? [4 marks]

\subsection{QUESTION 5 (20 MARKS)}
\noindent 5.1 State the clinical signs and symptoms of hematologic and Gastrointestinal syndromes as
well as the Approximate Dose (Gyt) required for the signs to appear. [6 marks]
\noindent 5.2 Discuss local tissue damage after high-dose irradiation. [8 marks]
\noindent 5.3 Describe the features of deterministic and stochastic effects of radiation exposure.[6 marks]
\noindent Discute y bosqueja la ruta o método que usarás para solucionar el problema. Vincúlalo con lo visto en clase o con algún libro, que puedes citar así: \cite{GustavoLopez}

\begin{enumerate}
    \item Si quieres

    \item Puedes explicar tu solución

    \item Por partes
\end{enumerate}

\noindent Recuerda los distintos tipos de ecuaciones que puedes agregar a tu documento:

\begin{itemize}
    \item Para una ecuación sencilla utilizamos el método \texttt{equation}, que encerramos con \texttt{boxed}:
    \begin{equation}\label{eq:Sencilla}
        \boxed{             % Así la encerramos para destacarla
        e^{\pi i} + 1 = 0
        }
    \end{equation}

    \item Podemos referenciar la ecuación anterior con el método \texttt{ref}: (Ec. \ref{eq:Sencilla}). Para una ecuación larga utilizamos el método \texttt{multiline}. Además, podemos agregar un asterisco (*) si no queremos numerarla:
    \begin{multline*}
        p(x) = 3x^6 + 14x^5y + 590x^4y^2 + 19x^3y^3 - 12x^2y^4 + 12xy^5 + 2y^6 - a^3b^3\\ 
            - 12x^2y^4 - 12xy^5 + 2y^6 - a^3b^3 + 3x^6 + 14x^5y - 590x^4y^2 + 19x^3y^3
    \end{multline*}

    \item Para dividir la ecuación larga en dos líneas o más (para por ejemplo mostrar un proceso de simplificación o de operaciones) utilizamos el comando \texttt{split}:
    \begin{equation} \label{eq1}
        \begin{split}
            A & = \frac{\pi r^2}{2} \\
             & = \frac{1}{2} \pi r^2
        \end{split}
    \end{equation}

    \item Para mostrar un \textit{sistema} de ecuaciones, utilizamos el ambiente \texttt{align}: 
    \begin{align*} 
        2x - 5y &=  8 \\ 
        3x + 9y &=  -12
    \end{align*}
    Que también podemos aplicar a sistemas más complejos:
    \begin{align*}
        x&=y           &  w &=z              &  a&=b+c\\
        2x&=-y         &  3w&=\frac{1}{2}z   &  a&=b\\
        -4 + 5x&=2+y   &  w+2&=-1+w          &  ab&=cb
    \end{align*}    

    \item Para \textit{agrupar} ecuaciones consecutivamente sin alineamiento, utilizamos \texttt{gather}:
    \begin{gather*} 
        2x - 5y =  8 \\ 
        3x^2 + 9y =  3a + c
    \end{gather*}
\end{itemize}

\noindent Muchas veces en nuestras tareas incorporamos imágenes como a la que hacemos referencia aquí (Figura \ref{fig:yo_en_la_vida}):
\begin{figure}[ht] % Especificamos ubicacion (h = aquí mero)
    \centering
    \includegraphics[width=0.5\textwidth]{Figuras/Cotorros.jpg} % Ancho especificado
    \caption{Esquema de las relaciones entre los componentes del núcleo atómico}
    \label{fig:yo_en_la_vida}
\end{figure}

\noindent Finalmente, recuerda utilizar el método \texttt{minted} para escribir el código que incluyas en tu tarea, y \texttt{verbatim} para el \textit{pseudo}-código\footnote{Muchas veces es mejor incluír \textit{ambos} para darte a entender con tu profesor}:

\begin{minted}[frame=lines, linenos, bgcolor=LightGray]{python}
    def funcion(argumento1, argumento2):
        """
        Explico que hace la funcón
        """
        if condicion == True:
            alumno = graduado
        else:
            alumno = baja
\end{minted}

\end{problem}
%---------------------------------------------------------------------------------------------
% \begin{problem}

% \section{Segundo Problema}

% \noindent
    
% \end{problem}
%---------------------------------------------------------------------------------------------



%--------------------------------------BIBLIOGRAFIA-------------------------------------------

\newpage

\nocite{*} % Agrega las referencias aunque no las hayas citado directamente

\bibliographystyle{unsrt}    % ESTILO DE BIBLIOGRAFÍA (Recomendados: abbrv, ieeetr, apalike, unsrt)
\bibliography{refs}     % REFERENCIAS EN ARCHIVO SEPARADO


\end{document}

