\documentclass[article]{article}
\usepackage{amsmath,amsthm,bm,mathrsfs}
\usepackage{listings}
\usepackage{xcolor}
\usepackage[utf8]{inputenc}
\usepackage{graphicx}

\definecolor{codegreen}{rgb}{0,0.6,0}
\definecolor{codegray}{rgb}{0.5,0.5,0.5}
\definecolor{codepurple}{rgb}{0.58,0,0.82}
\definecolor{backcolour}{rgb}{0.95,0.95,0.92}

\lstdefinestyle{mystyle}{
    backgroundcolor=\color{backcolour},   
    commentstyle=\color{codegreen},
    keywordstyle=\color{magenta},
    numberstyle=\tiny\color{codegray},
    stringstyle=\color{codepurple},
    basicstyle=\ttfamily\footnotesize,
    breakatwhitespace=false,         
    breaklines=true,                 
    captionpos=b,                    
    keepspaces=true,                 
    numbers=left,                    
    numbersep=5pt,                  
    showspaces=false,                
    showstringspaces=false,
    showtabs=false,                  
    tabsize=2
}

\lstset{style=mystyle}

\begin{document}
%%%%%%%%%%%%%%%%%%%%%% Title %%%%%%%%%%%%%%%%%%%%%%%%%%%%%%%%
\title{Practical Physics}
\author{Who TF is This Prof}
\date{\today}
\maketitle
%%%%%%%%%%%%%%%%%%%%% End of title %%%%%%%%%%%%%%%%%%%%%%%%%
%%%%%%%%%%%%%%%%%%%%% Introduction %%%%%%%%%%%%%%%%%%%%%%%%%
\section{Question 1}
\subsection*{} a)When protons are placed in a magnetic field, they precess. Explain why this occurs, and describe the resulting spin states of the protons and the significance of this in MRI (3marks)
\paragraph{Protons precess in a magnetic field due to a fundamental property called spin. Protons act like tiny spinning spheres with an intrinsic angular momentum. When placed in a magnetic field, this spinning creates a magnetic moment that interacts with the external field. In a magnetic field, the spin of the proton interacts with the magnetic field, causing it to align either parallel or anti-parallel to the magnetic field lines.When the body is placed in a strong magnetic field, such as an MRI scanner, the hydroge protons' axes all line up. This uniform alignment creates a magnetic vector oriented along the axis of the MRI scanner. When additional energy (in the form of a radio wave) is added to the magnetic field, the magnetic vector is deflected. The radio wave frequency (RF) that causes the hydrogen nuclei to resonate is dependent on the element sought (hydrogen in this case) and the strength of the magnetic field. The strength of the magnetic field can be altered electronically from head to toe using a series of gradient electric coils, and, by altering the local magnetic field by these small increments, different slices of the body will resonate as different frequencies are applied.}
\subsection*{}(b) Explain the meaning of the following terms with respect to medical physics.
Diagnostic imaging (2 marks)
\paragraph{Medical imaging is the technique and process of imaging the interior of a body for clinical analysis and medical intervention, as well as visual representation of the function of some organs or tissues. }
Medical imaging protocol (2 marks)
\paragraph{an outline that standardizes the way in which the images are acquired using the various modalities (PET, SPECT, CT, MRI)}
lonizing radiation (2 marks)
\paragraph{ a form of energy that acts by removing electrons from atoms and molecules of materials that include air, water, and living tissue}
\subsection*{}(c) Give a brief explanation of the principle of operation of Magnetic resonance imaging
(MRI) in disease diagnosis. (5marks)
\paragraph{The principle of operation of Magnetic Resonance Imaging (MRI) in disease diagnosis is based on the magnetic properties of hydrogen protons in the body's tissues and the phenomenon of nuclear magnetic resonance (NMR).

Strong magnetic field: The patient is placed inside a powerful magnetic field generated by the MRI scanner. This strong magnetic field causes the hydrogen protons in the body's tissues to align themselves with the direction of the field.
Radiofrequency (RF) pulses: A brief radiofrequency pulse is then applied, which causes the protons to absorb energy and become temporarily out of alignment with the magnetic field.
Precession and relaxation: After the RF pulse is turned off, the protons gradually return to their original alignment with the magnetic field. During this process, they release the absorbed energy in the form of radiofrequency signals, a phenomenon known as precession and relaxation.
Signal detection: The MRI scanner is equipped with receivers that detect these released radiofrequency signals from the precessing protons. The signals vary depending on the chemical environment and molecular properties of the tissues from which they originate.
Spatial encoding: By applying additional magnetic field gradients in different directions, the MRI scanner can determine the precise location of the protons within the body, allowing for the creation of detailed cross-sectional images.
Image reconstruction: The detected signals are then processed by a computer, which uses complex mathematical algorithms to reconstruct detailed images of the body's internal structures.

Disease Diagnosis:

Abnormal Tissue Properties: Diseases can alter tissue properties like water content, fat content, or blood flow. These variations affect the MRI signal, allowing doctors to identify abnormalities like tumors, injuries, or inflammation.
Detailed Images: MRI provides high-resolution images of soft tissues, which are not easily visualized with X-rays. This allows for better differentiation between healthy and diseased tissues.
Functional MRI (fMRI): Specialized techniques can even map brain activity by measuring blood flow changes associated with brain function. This helps diagnose neurological disorders.}
\subsection*{}(d) What are the problems encountered in biomedical measurements? (2 marks) 
\paragraph{Variability of variables – the measurements taken under fixed set of conditions at one time may not be the same under the same conditions at another time. Therefore, physiological variables can never have definite values but these can be represented by some probabilistic distributions.
Inaccessibility of variables – in a living system, it is impossible to gain access to many variables such as dynamic neurochemical activities, activities in the brain, etc. For this reasons, indirect measurements of these variables have to be made.
Poor interrelationship knowledge – better understanding of interrelationship of physiological variables is unavailable. That is why, physiological measurement with large tolerances are often accepted by the physicians where indirect measurements for inaccessible variables are used.
Interaction among physiological signals – given that many physiological signals are being generated by the body, and these have some degree of interaction which present a problem in measurement.
Artifacts –any outside signal may affect the variable being measured by the biomedical instrumentation system.
Effect of transducer – any measurement is affected in some way by the presence of the transducer in the biomedical instrumentation system.
}
\subsection*{}(e) Write down the steps necessary for developing an assembly language program
\paragraph{i. creating a template file that can be used as a starting point for any program the user will create

ii. writing assembly language source files using printf and scanf for a program to read input and print output for a program

iii. using an assembler and linker to translate their source programs into ARM executable programs

iv. running the programs from a shell command

v.finally, using the gbdtui to view the program execution and state

Alternatively:

1. Define the program's purpose and requirements: Clearly understand the task that the assembly program needs to accomplish and define its requirements, such as input/output, memory management, and any specific hardware or system interactions.

2. Choose the appropriate assembler: Different architectures and operating systems may use different assemblers. For example, the NASM (Netwide Assembler) is commonly used for x86 architecture on Linux and Windows, while the MASM (Microsoft Macro Assembler) is commonly used for x86 architecture on Windows.

3. Set up the development environment: Install the necessary assembler and any additional tools or libraries required for your specific project. This may include an integrated development environment (IDE), text editor, linker, and debugger.

4. Write the assembly code: Create a new text file and start writing the assembly code according to the syntax and directives of the chosen assembler. Assembly code consists of instructions, directives, and labels that specify the program's logic and data.

5. Define data structures: Declare and define any necessary data structures, such as variables, arrays, or structures, using the appropriate directives and instructions.

6. Implement program logic: Write the assembly instructions that perform the desired operations, such as arithmetic calculations, memory operations, control flow, and system calls.

7. Handle input/output: Implement code to handle input from the user or other sources, as well as code to output the program's results or messages.

8. Assemble the code: Use the assembler to translate the assembly code into machine code (object code). This process involves parsing the assembly code, resolving labels, and generating the corresponding binary instructions.

9. Link with libraries (if necessary): If your program requires external libraries or modules, use a linker to combine the object code with the necessary libraries, creating an executable file.

10. Test and debug: Execute the assembled program and test it with various inputs and scenarios. If any errors or issues are encountered, use a debugger to step through the code, inspect register values, and locate and fix the problems.

11. Optimize (optional): Depending on the performance requirements, you may need to optimize the assembly code by analyzing and improving critical sections, minimizing memory usage, or leveraging specific processor instructions or techniques.

12. Document the code: Properly document the assembly code, including comments explaining the purpose, logic, and any important considerations or assumptions. Good documentation is crucial for maintaining and understanding assembly programs.

}
\subsection*{}(f) In 8086 Microprocessors, if DS=1100H, BX=0200H and SI=0500 what is the the
address accessed by MOV CH, (BX +SI)?(5 marks) 
\paragraph{In the 8086 microprocessor, the addressing mode used in the instruction `MOV CH, [BX + SI]` is called the "base-index" addressing mode. This addressing mode allows you to access memory locations by combining the values of the BX (base) and SI (index) registers.

Given:
- DS = 1100H (Data Segment register)
- BX = 0200H
- SI = 0500H

To calculate the effective address accessed by the instruction `MOV CH, [BX + SI]`, we need to follow these steps:

1. Calculate the offset by adding the values of BX and SI:
   Offset = BX + SI
   Offset = 0200H + 0500H
   Offset = 0700H

2. Combine the segment value (DS) and the offset to form the physical address:
   Physical Address = (DS × 10H) + Offset
   Physical Address = (1100H × 10H) + 0700H
   Physical Address = 11000H + 0700H
   Physical Address = 11700H

Therefore, the instruction `MOV CH, [BX + SI]` accesses the memory location at the physical address 11700H.

In this addressing mode, the BX register typically holds the base address of a data structure or an array, while the SI register holds an index or offset within that data structure or array. By adding BX and SI, you can calculate the effective address of a specific element or location within that data structure or array.

The square brackets `[]` around `[BX + SI]` indicate that the value inside the brackets is used as a memory address, rather than a direct operand value.

So, the instruction `MOV CH, [BX + SI]` loads the value from the memory location 11700H into the CH register (the high byte of the CX register).



\textit{Easily explained as:}

MOV CH, [BX + SI]

After executing the above instruction, a byte moves from the address pointed by Bx + SI in data segment to CL
i.e. the Physical Address accessed will be:

DS * (10H) + BX + SI

Given DS = 1100H, BX = 0200 H and S1 = 0500H
Putting on the respective values, we get the physical address as:

= (1100 × 10) + 0200 + 0500

= 11000 + 0200 + 0500

= 11700 H

}
\subsection*{}(g) Give a classification of the available instructions in a microprocessor and describe
the function of each.
(7 marks)

\paragraph{Processor instruction sets can be broadly classified into several categories based on their function. Here's a breakdown of some common instruction types:


1. Data Transfer Instructions:

Function: These instructions move data between various locations within the processor, including registers, memory, and I/O ports.
Examples:
     `MOV` (Move): Transfers data between registers, memory, and immediate values.
     
     `PUSH` (Push): Pushes data onto the stack.
     
     `POP` (Pop): Pops data from the stack.
     
     `LOAD` (Load): Loads data from memory into a register.
     
     `STORE` (Store): Stores data from a register into memory.


2. Arithmetic and Logical Instructions:

Function: These instructions perform arithmetic (addition, subtraction, multiplication, division) and logical (AND, OR, NOT, XOR) operations on data stored in registers or memory.
Examples:
    `ADD` (Add): Adds two operands and stores the result.
    
    `SUB` (Subtract): Subtracts two operands and stores the result.
    
    `MUL` (Multiply): Multiplies two operands and stores the result.
    
    `DIV` (Divide): Divides two operands and stores the result (often with quotient and remainder).
    
    `AND` (Logical AND): Performs bitwise AND operation on two operands.
    
    `OR` (Logical OR): Performs bitwise OR operation on two operands.
    
    `NOT` (Logical NOT): Inverts the bits of an operand.
    
    `XOR` (Exclusive OR): Performs bitwise XOR operation on two operands.


3. Control Flow Instructions:

Function: These instructions control the flow of program execution by performing conditional jumps, loops, and subroutine calls.
Examples:
    
    `JMP` (Jump): Unconditionally jumps to a specific memory location.
    
    `JE` (Jump if Equal): Jumps if the condition (zero flag set) is true.
    
    `JNE` (Jump if Not Equal): Jumps if the condition (zero flag not set) is true.
    
    `JG` (Jump if Greater): Jumps if the condition (greater than flag set) is true.
    
    `JL` (Jump if Less): Jumps if the condition (less than flag set) is true.
    
    `CALL` (Call): Calls a subroutine and transfers control to it.
    
    `RET` (Return): Returns from a subroutine and transfers control back to the calling program.
    
    `LOOP` (Loop): Decrements a counter and jumps back to a specified instruction if the counter is not zero.


4. Bit Manipulation Instructions:

Function: These instructions operate on individual bits within a byte or word, allowing for setting, clearing, flipping, and testing specific bits.
Examples:
    
    `SET` (Set bit): Sets a specific bit to 1 in a register or memory location.
    
    `CLR` (Clear bit): Clears a specific bit to 0 in a register or memory location.
    
    `TEST` (Test bit): Tests the value of a specific bit in a register or memory location.


5. Processor Control Instructions:

Function: These instructions control various aspects of the processor's operation, such as enabling/disabling interrupts, setting flags, and entering/exiting special modes.
Examples:
     `HLT` (Halt): Halts processor execution until an interrupt occurs.
    
    `CLI` (Clear Interrupt Flag): Clears the interrupt flag, disabling interrupts.
    
    `STI` (Set Interrupt Flag): Sets the interrupt flag, enabling interrupts.
    
    `NOP` (No Operation): Performs no operation, but can be used for timing purposes.
}
\newpage
\section{Question 2}
\subsection*{}(a) What is microprocessor? (3 marks)
\paragraph{ computer processor for which the data processing logic and control is included on a single integrated circuit (IC), or a small number of ICs}
\subsection*{}(b) What is Accumulator? (3 marks)

\paragraph{a register in which intermediate arithmetic logic unit results are stored}
\subsection*{}(C) What is a Subroutine program? (3 marks)
\paragraph{a sequence of program instructions that perform a specific task, packaged as a unit.}
\subsection*{}(d) Define addressing mode. (3 marks)
\paragraph{ to the techniques and rules used by processors to calculate the effective memory address or operand location for data operations}
\subsection*{}(e) Define instruction cycle. (3 marks)
\paragraph{the cycle that the central processing unit (CPU) follows from boot-up until the computer has shut down in order to process instructions}
\subsection*{}(f) Why study instruction set of an 8086 microprocessor? (5 marks)
\paragraph{Low-level programming: The 8086 represents a foundational architecture in the x86 family of processors that power most modern computers. Studying its instruction set provides a deeper understanding of how processors execute instructions at the most basic level.
Machine code connection: Assembly language directly translates to machine code that the processor understands. By learning the 8086 instruction set, you gain insights into how high-level languages are ultimately translated into instructions the processor can execute.}
\newpage
\section{Question 3}
\subsection*{}(a) Write a program to add a data byte located at offset O500H in 2000H segment to another data byte available at O600H in the same segment and store the result at 0700H in the same segment. (5 marks)

\begin{lstlisting}
    ; Program to add two data bytes and store the result

.model small    ; Use the small memory model
.stack 100h     ; Reserve 256 bytes for the stack

.data           ; Data segment
    byte1 db 10h    ; First byte at offset 0500H
    byte2 db 20h    ; Second byte at offset 0600H
    result db ?     ; Uninitialized byte to store the result at offset 0700H

.code           ; Code segment
.startup        ; Entry point

    mov ax, 2000h   ; Initialize DS with the segment address
    mov ds, ax

    mov al, byte1   ; Move the first byte to AL register
    mov bl, byte2   ; Move the second byte to BL register
    add al, bl      ; Add the two bytes (AL = AL + BL)
    mov result, al  ; Store the result in the 'result' byte

    mov ah, 4Ch     ; Exit function
    int 21h         ; Call DOS interrupt

end .startup    ; End of program
\end{lstlisting}

\newpage


\subsection*{}(b) Implement the following Data Copy/Transfer instruction (5 marks)

MOV AX, BX 

AX BX


FF 33 10 AB


AH AL BH AL


\begin{lstlisting}
    ; Program to demonstrate the MOV AX, BX instruction

.model small
.stack 100h

.data
    source_value dw 33ABh   ; BX = 33ABh (initially)

.code
.startup

    mov ax, @data   ; Initialize DS with the data segment address
    mov ds, ax

    mov bx, source_value   ; Load the value 33ABh into BX

    ; Before the MOV instruction
    ; AX = FFFFh (uninitialized)
    ; BX = 33ABh

    mov ax, bx   ; Copy the contents of BX into AX

    ; After the MOV instruction
    ; AX = 33ABh
    ; BX = 33ABh
    ; AH = 33h (high byte of AX)
    ; AL = ABh (low byte of AX)
    ; BH = 33h (high byte of BX)
    ; BL = ABh (low byte of BX)

    mov ah, 4Ch   ; Exit function
    int 21h       ; Call DOS interrupt

end .startup
\end{lstlisting}
\subsection*{}(c) What are the key requirements for specifying biomedical instrumentation systems?
(5 marks)
\paragraph{Biomedical instrumentation systems are designed to measure, monitor, and analyze various physiological parameters and signals from the human body. These systems play a crucial role in diagnosing and treating medical conditions, as well as in medical research. To ensure the safety, accuracy, and effectiveness of these systems, several key requirements must be considered during their specification and design. Here are some of the essential requirements for specifying biomedical instrumentation systems:

1. Safety: Patient safety is of utmost importance in biomedical instrumentation. The systems should be designed to minimize potential risks and harm to patients. This includes electrical safety measures, protection against electromagnetic interference, and fail-safe mechanisms to prevent hazardous situations.

2. Accuracy and precision: Biomedical instrumentation systems must provide accurate and precise measurements of physiological parameters. The accuracy and precision requirements vary depending on the application and the parameter being measured, but they are critical for proper diagnosis and treatment.

3. Biocompatibility: Any part of the instrumentation system that comes into direct contact with the patient's body must be biocompatible, meaning it should not cause adverse reactions or harm to the patient's tissues or fluids.

4. Ergonomics and usability: The instrumentation systems should be designed with ergonomics in mind, ensuring ease of use for healthcare professionals and minimizing the risk of user errors. Clear and intuitive user interfaces, appropriate sizing, and comfortable operation are essential.

5. Portability and mobility: Depending on the application, biomedical instrumentation systems may need to be portable or mobile to facilitate patient monitoring and treatment in different settings, such as hospitals, ambulances, or home care environments.

6. Power requirements: The power requirements of biomedical instrumentation systems should be carefully considered, including the need for battery operation, backup power sources, and power management strategies to ensure continuous and reliable operation.

7. Data acquisition and processing: Biomedical instrumentation systems often require advanced data acquisition and processing capabilities to capture and analyze physiological signals accurately. This may involve analog-to-digital converters, signal processing algorithms, and data storage and transmission capabilities.

8. Connectivity and interoperability: Modern biomedical instrumentation systems should have the ability to connect and communicate with other healthcare systems, such as electronic medical records (EMRs), patient monitoring systems, and healthcare information networks. Interoperability and adherence to relevant standards are essential for seamless integration and data exchange.

9. Regulatory compliance: Biomedical instrumentation systems must comply with relevant regulatory requirements and standards, such as those set by organizations like the U.S. Food and Drug Administration (FDA), International Electrotechnical Commission (IEC), and International Organization for Standardization (ISO). These regulations ensure the safety, efficacy, and quality of the systems.

10. Maintainability and serviceability: Biomedical instrumentation systems should be designed for easy maintenance and servicing, including accessibility for repairs, software updates, and calibration. This ensures the longevity and reliable operation of the systems throughout their life cycle.
}
\subsection*{}(d) (i) What is an electroencephalogram (EEG)? (2 marks)
\paragraph{is a test that measures electrical activity in the brain using small, metal discs (electrodes) attached to the scalp.}

(ii) What is the difference between an EEG and an MRI? (2 marks)
\paragraph{MRI has a higher spatial resolution than electroencephalography (EEG).}

\begin{figure}[htp]
    \centering
    \includegraphics[width=6cm]{EEG-vs-MRI-vs-fMRI-compared.png}
    \caption{differences}
    \label{fig:differences}
\end{figure}

\newpage

\section*{Question 4}
\subsection*{}(a) Explain the difference between Low pass filter and High pass filter
(4 marks)
\paragraph{Low-pass filters, as the name suggests, allow low-frequency signals to pass through while attenuating high-frequency signals. On the other hand, high-pass filters allow high-frequency signals to pass through while attenuating low-frequency signals.  If we talk about high pass filter, so it is a circuit which allows the high frequency to pass through it while it will block low frequencies. On the contrary, low pass filter is an electronic circuit which allows the low frequency to pass through it and blocks the high-frequency signal.}
\subsection*{}(b) What is the practical significance of low or high frequency filtering?
(4 marks)
\paragraph{Biomedical instrumentation systems are designed to measure, monitor, and analyze various physiological parameters and signals from the human body. These systems play a crucial role in diagnosing and treating medical conditions, as well as in medical research. To ensure the safety, accuracy, and effectiveness of these systems, several key requirements must be considered during their specification and design. Here are some of the essential requirements for specifying biomedical instrumentation systems:

1. Safety: Patient safety is of utmost importance in biomedical instrumentation. The systems should be designed to minimize potential risks and harm to patients. This includes electrical safety measures, protection against electromagnetic interference, and fail-safe mechanisms to prevent hazardous situations.

2. Accuracy and precision: Biomedical instrumentation systems must provide accurate and precise measurements of physiological parameters. The accuracy and precision requirements vary depending on the application and the parameter being measured, but they are critical for proper diagnosis and treatment.

3. Bio-compatibility: Any part of the instrumentation system that comes into direct contact with the patient's body must be bio-compatible, meaning it should not cause adverse reactions or harm to the patient's tissues or fluids.

4. Ergonomics and usability: The instrumentation systems should be designed with ergonomics in mind, ensuring ease of use for healthcare professionals and minimizing the risk of user errors. Clear and intuitive user interfaces, appropriate sizing, and comfortable operation are essential.

5. Portability and mobility: Depending on the application, biomedical instrumentation systems may need to be portable or mobile to facilitate patient monitoring and treatment in different settings, such as hospitals, ambulances, or home care environments.

6. Power requirements: The power requirements of biomedical instrumentation systems should be carefully considered, including the need for battery operation, backup power sources, and power management strategies to ensure continuous and reliable operation.

7. Data acquisition and processing: Biomedical instrumentation systems often require advanced data acquisition and processing capabilities to capture and analyze physiological signals accurately. This may involve analog-to-digital converters, signal processing algorithms, and data storage and transmission capabilities.

8. Connectivity and interoperability: Modern biomedical instrumentation systems should have the ability to connect and communicate with other healthcare systems, such as electronic medical records (EMRs), patient monitoring systems, and healthcare information networks. Interoperability and adherence to relevant standards are essential for seamless integration and data exchange.

9. Regulatory compliance: Biomedical instrumentation systems must comply with relevant regulatory requirements and standards, such as those set by organizations like the U.S. Food and Drug Administration (FDA), International Electrotechnical Commission (IEC), and International Organization for Standardization (ISO). These regulations ensure the safety, efficacy, and quality of the systems.

10. Maintainability and serviceability: Biomedical instrumentation systems should be designed for easy maintenance and servicing, including accessibility for repairs, software updates, and calibration. This ensures the longevity and reliable operation of the systems throughout their life-cycle.

These key requirements should be carefully considered and addressed during the specification and design phases of biomedical instrumentation systems to ensure their safe, effective, and reliable operation in various healthcare settings.}
\subsection*{}(c) What are some of the applications of Low-Pass and High-Pass Filters?
(4 marks)
\paragraph{Low-pass filters and high-pass filters are two fundamental types of frequency filters, each with its own set of applications across various fields. Here are some common applications of low-pass and high-pass filters:

Applications of \textit{Low-Pass Filters:}

1. Audio Systems: Low-pass filters are used in audio systems to remove high-frequency noise and prevent aliasing in digital audio processing. They are also used in sub-woofer crossover networks to separate low-frequency signals for the sub-woofer.

2. Video Signal Processing: Low-pass filters are employed in video signal processing to remove high-frequency noise and prevent aliasing during video encoding and decoding.

3. Anti-Aliasing in Analog-to-Digital Converters (ADCs): Low-pass filters are used as anti-aliasing filters in ADCs to prevent aliasing by removing high-frequency components above the Nyquist frequency before sampling.

4. Power Supply Filtering: Low-pass filters are used in power supply circuits to smooth out high-frequency ripples and noise from the rectified output, providing a clean DC output.

5. Biomedical Signal Processing: Low-pass filters are used in biomedical signal processing, such as electrocardiogram (ECG) and electroencephalogram (EEG) signals, to remove high-frequency noise and interference.

Applications of \textit{High-Pass Filters:}

1. Audio Systems: High-pass filters are used in audio systems to remove low-frequency rumble and hum, improving the clarity of mid and high-frequency sounds.

2. Seismic Signal Processing: In seismic data analysis, high-pass filters are used to remove low-frequency noise caused by ocean waves, wind, and other environmental factors, allowing for better interpretation of seismic data.

3. Radar and Sonar Systems: High-pass filters are used in radar and sonar systems to remove low-frequency clutter and interference, improving target detection and ranging accuracy.

4. Image Processing: High-pass filters are used in image processing for tasks such as edge detection, sharpening, and enhancement of fine details, as high-frequency components often correspond to edges and sharp transitions in images.

5. Biomedical Signal Processing: High-pass filters are used in biomedical signal processing, such as electromyography (EMG) signals, to remove low-frequency motion artifacts and baseline drift.

6. AC Coupling: High-pass filters are used for AC coupling in electronic circuits, allowing the transmission of AC signals while blocking DC components.
}
\subsection*{}(c) State the functional difference between Low pass filter and High pass filter
(4 marks)
\paragraph{The functional difference between a low-pass filter and a high-pass filter lies in the frequency components they allow or reject from a signal.

\textit{Low-Pass Filter:}
A low-pass filter is a circuit or a signal processing function that allows low-frequency components of a signal to pass through while attenuating (reducing the amplitude of) high-frequency components above a specified cutoff frequency. The key characteristics of a low-pass filter are:

1. Pass band: It passes low-frequency components below the cutoff frequency with minimal attenuation.
2. Stop band: It attenuates or rejects high-frequency components above the cutoff frequency.
3. Cutoff frequency: The frequency at which the filter starts to attenuate the signal. Frequencies below the cutoff frequency are passed, and frequencies above the cutoff frequency are attenuated.

Low-pass filters are commonly used to remove high-frequency noise, prevent aliasing in digital signal processing, and separate low-frequency signals from high-frequency signals in audio and video applications.

\textit{High-Pass Filter:}
A high-pass filter is a circuit or a signal processing function that allows high-frequency components of a signal to pass through while attenuating low-frequency components below a specified cutoff frequency. The key characteristics of a high-pass filter are:

1. Pass band: It passes high-frequency components above the cutoff frequency with minimal attenuation.
2. Stop band: It attenuates or rejects low-frequency components below the cutoff frequency.
3. Cutoff frequency: The frequency at which the filter starts to pass the signal. Frequencies above the cutoff frequency are passed, and frequencies below the cutoff frequency are attenuated.

High-pass filters are commonly used to remove low-frequency noise, separate high-frequency signals from low-frequency signals, and remove DC offsets or baseline drift in various applications, such as audio systems, biomedical signal processing, and seismic data analysis.

In summary, the functional difference between a low-pass filter and a high-pass filter is:

- Low-pass filter: Passes low-frequency components and attenuates high-frequency components above the cutoff frequency.
- High-pass filter: Passes high-frequency components and attenuates low-frequency components below the cutoff frequency.

}
\subsection*{}(d) List two difference between Low pass filter and High pass filter (4 marks)
\begin{figure}
    \centering
    \includegraphics[width=14cm]{low-pass-vs-high-pass-filter-table.jpg}
    \caption{anaza one}
    \label{fig:enter-label}
\end{figure}
\newpage
%%%%%%%%%%%%%%%%%%%% End of Proof %%%%%%%%%%%%%%%%%%%%%%%%%%%
\end{document}
