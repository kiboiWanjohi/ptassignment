%%%%%%%%%%%%%%%%%%%%%%%%%%%%%%%%%%%%%%%%%%%%%%%%%%%%%%%%%%%%%%%
%
% Welcome to Overleaf --- just edit your LaTeX on the left,
% and we'll compile it for you on the right. If you give
% someone the link to this page, they can edit at the same
% time. See the help menu above for more info. Enjoy!
%
%%%%%%%%%%%%%%%%%%%%%%%%%%%%%%%%%%%%%%%%%%%%%%%%%%%%%%%%%%%%%%%
%\title{Math 453 HW 1}
\documentclass[addpoints]{exam}

\usepackage{amsmath,enumitem,wrapfig}
\usepackage{tikz}

\newcommand{\StudentName}{Heejin Mun}
\newcommand{\AssignmentName}{HW 1}

\pagestyle{headandfoot}
\runningheadrule
\firstpageheadrule
\firstpageheader{Math 453}{\StudentName}{\AssignmentName}
\runningheader{Math 453}{\StudentName}{\AssignmentName}
\firstpagefooter{}{}{}
\runningfooter{}{}{}

\printanswers

\begin{document}


Organize your work and show any work that you want credit for. Use full sentences where possible.

\begin{questions}
    \question \textbf{QUESTION 1 (30 MARKS)}
    \begin{parts}
        \part 
        1.1.
        Explain the following terms
        (a) Compton or recoil electrons, Beta particles and Pair Production electrons [3 marks] 
        (b) Synchrotron radiation and Annihilation quanta [2 marks]
        (c) Remote afterloading technique and Neutron activation [2 marks]
        \begin{solution}
        
        \end{solution}
        \part 
        1.2. (a) Write a nuclear equation when cesium-137 decays into barium-137. [3 marks] 
        (b) Calculate the decay energy Q- for the B- decay of cesium-137 into barium-137. [5mks] 
        (c) When Fluorodeoxyglucose (FDG) is labelled with radionuclide fluorine-18 it can be injected intravenously into a patient for use in positron emission tomography (PET) functional imaging. What are some of the uses of the FDG PET? [4 marks]
        \begin{solution}
            
        \end{solution}
        \part 1.4. Briefly explain directly ionizing radiation and indirectly ionizing radiation. Give examples for each. [6 marks]
       \begin{solution}
        
       \end{solution}
       \part 1.3. Write down the most important characteristics of radionuclides used in external beam radiotherapy. [5 marks]

    \end{parts}
    
\end{questions}

\begin{questions}
    \question \textbf{Question 2 (20 marks)}
    \begin{parts}
        \part 
        2.1. Which physical quantities are conserved during a nuclear transformation? [5 marks]
        \begin{solution}
            
        \end{solution}
        \part 2.2. The figure below shows an energy level diagram for
        (a) Explain the decay scheme shown above as it ends at the ground state. [5 marks] Calculate the decay energy (Q) for the x decay of 228Ra (radium). [5 marks] Calculate the kinetic energy Ex1 of the x- particle.

        (b) Calculate the decay energy \((Q_B)\) and \((Q_EC)\) for the decay of iridium-192 [8 marks]
        (c) Explain Briefly shy iridium-192 is more widely used in brachytherapy rather than the traditional radium or caesium. What is the half-life of iridium-192? [3 marks]
        
    \end{parts}
        
\end{questions}

\begin{questions}
    \question \textbf{Question 3 (20 marks)}
    \part 3.1. Electron Capture and B+ are competing processes. Write down conditions that are required for one process to proceed over the other process. [4 marks]
    \begin{solution}
        
    \end{solution}
    \part 3.2 Discuss any two ways or mechanisms by which photons can interact with matter that play a very important role in therapeutic as well as in diagnostic medical physics. [8 marks]
    \begin{solution}
        
    \end{solution}
    \part 3.3. An X-ray photon beam of photon energy 15 eV is allowed to fall on a metal surface. If the threshold frequency for the metal is \(1.2 × 1015 Hz\) , then find the maximum kinetic energy gained by a photoelectron, the maximum speed of a photoelectron, and the stopping potential corresponding to the maximum kinetic energy.
    [8 marks]
    \begin{solution}
        
    \end{solution}        
\end{questions}


\begin{questions}
    \question \textbf{QUESTION 4 (20 MARKS)}
    \part 
    4.1. Explain the following terms. (a) the linear attenuation coefficient and Mass attenuation coefficient [2 marks] (b) First half-value layer HVL,, the second half-value layer HVL2 [2 marks]
    \begin{solution}
        
    \end{solution}
    \part 4.2. Derive the functional relationship shown below between the activity of a radioactive substance and time. \begin{equation}
        A = A_oe-2z
    \end{equation}
    \part 4.3. Draw a typical curve for the radioactive substance as given by the expression in Question 4.2. Briefly explain how you will determine the 2. [6 marks]
    \begin{solution}
        
    \end{solution}
    \part 4.4. If a 0.2-cm thickness of material transmits 25% of a monoenergetic beam of photons, calculate the half-value layer (HVL) of the beam for that material.
    [2 marks]
    \begin{solution}
        
    \end{solution}
\end{questions}

\begin{questions}
    
    \question \textbf{QUESTION 5 (20 MARKS)}
    \part 5.1. Describe the term stopping power of a material.[2]
    \part 5.2. Explain the following quantities that are used for the purpose of quantifying radiation. (a) Activity A, (b) Kerma (K), (c) Equivalent Dose (H)[6]
    \begin{solution}
        
    \end{solution}
    \part 5.3 Write down factors that are considered to choose a radiation beam and dose prescription in treatment of disease with radiation[4]
    \begin{solution}
        
    \end{solution}
    \part5.4 Draw and explain the Bragg curve. What is the importance of the enhanced ionization in the Bragg peak?[8]
    \begin{solution}
        
    \end{solution}
\end{questions}

\begin{questions}
    \question \textbf{QUESTION 1 (30 MARKS)}
    \begin{parts}
        \part 
        1.1.
        Explain the following terms
        (a) Compton or recoil electrons, Beta particles and Pair Production electrons [3 marks] 
        (b) Synchrotron radiation and Annihilation quanta [2 marks]
        (c) Remote afterloading technique and Neutron activation [2 marks]
        \begin{solution}
        
        \end{solution}
        \part 
        1.2. (a) Write a nuclear equation when cesium-137 decays into barium-137. [3 marks] 
        (b) Calculate the decay energy Q- for the B- decay of cesium-137 into barium-137. [5mks] 
        (c) When Fluorodeoxyglucose (FDG) is labelled with radionuclide fluorine-18 it can be injected intravenously into a patient for use in positron emission tomography (PET) functional imaging. What are some of the uses of the FDG PET? [4 marks]
        \begin{solution}
            
        \end{solution}
        \part 1.3. Which physical quantities are conserved during a nuclear transformation[5 mks]
        \begin{solution}
            
        \end{solution}
        \part 1.4. Briefly explain directly ionizing radiation and indirectly ionizing radiation. Give examples for each. [6 marks]
       \begin{solution}
        
       \end{solution}
       
    \end{parts}
    
\end{questions}

\begin{questions}
    \question \textbf{QUESTION 2 (20 MARKS)}
    \begin{parts}
        \part 2.1. Write down the most important characteristics of radionuclides used in external beam radiotherapy. [4 marks] 
        \begin{solution}
            
        \end{solution}
        \part 2.2.The figure below shows an energy level diagram for 228Ra (radium) decaying through a-decay into 22Ra (radon).
        \begin{solution}
            
        \end{solution}
        \part (a)Explain the decay scheme shown above as it ends at the ground state.[5 marks]
        \begin{solution}
            
        \end{solution}
        \part (b)Calculate the decay energy (Q.) for the decay of 228Ra (radium).[5 marks]
        \begin{solution}
            
        \end{solution}
        \part (c)Calculate the kinetic energy E.1 of the- particle.[3 marks]
        \begin{solution}
            
        \end{solution}
        \part (d) Explain why modern brachytherapy is now carried out with other radionuclides (e.g., iridium-192, cesium-137, iodine-125, etc.) instead of the radium-226 which was very popular in the past century?[3 marks]
        \begin{solution}
            
        \end{solution}
    \end{parts}
\end{questions}


\begin{questions}
    \question \textbf{Question 3 (20 marks)}
    \begin{parts}
        \part 3.1. Briefly describe the areas where Ionizing radiation is used. [8 marks]
        \begin{solution}
        
        \end{solution}
        \part 3.2 Electron Capture and B are competing processes. Write down conditions that are required for one process to proceed over the other process. [4 marks]

        \begin{solution}
            
        \end{solution}
        \part 3.3. Discuss any two ways or mechanisms by which photons can interact with matter that play a very important role in therapeutic as well as in diagnostic medical physics. [8 marks]
        
    \end{parts}
    
\end{questions}

\begin{questions}
\question    \textbf{Question 4 (20 marks)}
\part 4.1. Explain the terms linear attenuation coefficient and mass attenuation coefficient [2 marks]
\begin{solution}
    
\end{solution}
\part 4.2. Derive the functional relationship shown below between the activity of a radioactive substance and time. Explain the terms.
\(A = A_oe-yt\) [7 marks]
\begin{solution}
    
\end{solution}
\part 4.3. Draw a typical curve for the radioactive substance as given by the expression in 4.2. Briefly explain how you will determine the \(gamma\) [6 marks]
\begin{solution}
    
\end{solution}
\part 4.4. The half-life of strontium-90, (Sr), is 28.8 years. Calculate the (a) decay constant and (b) the initial activity of 1.00g of the material. [3 marks]
\begin{solution}
    
\end{solution}
\end{questions}

\begin{questions}
    
    \question \textbf{QUESTION 5 (20 MARKS)}
    \part 5.1. Describe the term stopping power of a material.[2]
    \part 5.2. Explain the following quantities that are used for the purpose of quantifying radiation. (a) Activity A, (b) Kerma (K), (c) Equivalent Dose (H)[6]
    \begin{solution}
        
    \end{solution}
    \part 5.3 Write down factors that are considered to choose a radiation beam and dose prescription in treatment of disease with radiation[4]
    \begin{solution}
        
    \end{solution}
    \part 5.4 (a) The Bragg curve below shows a plot of the variation of ionization density as a function of distance travelled by an alpha (∞) particle in air. Explain that form of the curve shown below. [6 marks]
    \begin{solution}
        
    \end{solution}
    \part (b) What is the practical application of the enhanced ionization in the Bragg peak? [2 marks]
    \begin{solution}
        
    \end{solution}
\end{questions}

\begin{questions}
    \question \textbf{QUESTION 1 (30 MARKS)}
\part 1.1. Explain the following terms:
(a)Photoelectrons, Thermions and Megavoltage electrons [3 marks]
(b)Characteristic (fluorescence) x-rays and Bremsstrahlung x-rays [2 marks]
(c) Remote afterloading technique and Neutron activation [2 marks]
\begin{solution}
    
\end{solution}
\part 1.2. (a) Write a nuclear equation when fluorine-18 decays into oxygen-18.[3 marks]
\begin{solution}
    
\end{solution}
 \part (b) Calculate the decay energy Q+¿¿ for the B+ decay of fluorine-18 into oxygen-18. (c) When Fluorodeoxyglucose (FDG) is labelled with radionuclide fluorine-18 it can be injected intravenously into a patient for use in positron emission tomography (PET) functional imaging. What are some of the uses of the FDG PET?[5 marks]
 \begin{solution}
    
 \end{solution}
 \part The half-life of strontium-90, (Sr), is 28.8 years. Calculate the (a) decay constant and (b) the initial activity of 1.00g of the material. [5 marks]
 \begin{solution}
    
 \end{solution}
 \part Briefly describe the areas where ionizing radiation is used [6 marks]
 \begin{solution}
    
 \end{solution}
\end{questions}

\begin{questions}
    \question \textbf {QUESTION 2 (20 MARKS)}
    \begin{parts}
        \part 2.1.Which physical quantities are conserved during a nuclear transformation?[5 marks]
        \begin{solution}
            
        \end{solution}
        \part 2.2.
        The figure below shows an energy level diagram for 228Ra (radium) decaying through a-decay into 22Ra (radon).
        (a)
        Explain the decay scheme shown above as it ends at the ground state.
        [5]
        \part (b) Calculate the decay energy (Q.) for the decay of 228Ra (radium).[5]
\part (c) Calculate the kinetic energy E.1 of the- particle.[3]
\part (d) Explain why modern brachytherapy is now carried out with other radionuclides (e.g., iridium-192, cesium-137, iodine-125, etc.) instead of the radium-226 which was very popular in the past century? [2]

    \end{parts}

\end{questions}

\begin{questions}
\question \textbf{QUESTION 3 (20 MARKS)}    
\begin{parts}
    \part 3.1. Write down the possible outcomes when an X-ray beam or gamma radiation passes through an object.[3]
    \begin{solution}
        
    \end{solution}
    \part 3.2. Discuss any two ways or mechanisms by which photons can interact with matter that play a very important role in therapeutic as well as in diagnostic medical physics.[8]
    \begin{solution}
        
    \end{solution}
    \part 3.3. An X-ray photon beam of photon energy 10 eV is allowed to fall on a metal surface. If the threshold frequency for the metal is 1.5 × 1015 Hz, then find the maximum kinetic energy gained by a photoelectron, the maximum speed of a photoelectron, and the stopping potential corresponding to the maximum kinetic energy.[8 marks]
    \begin{solution}
        
    \end{solution}
\end{parts}

\end{questions}

\begin{questions}
    \question \textbf{QUESTION 4 (20 MARKS)}
\begin{parts}
    \part 4.1. Explain the following terms.
    (a) the linear attenuation coefficient and Mass attenuation coefficient[2]
    (b) First half-value layer HVL1, the second half-value layer HVL2 and tenth value layer [3]
    \begin{solution}
        
    \end{solution}
    \part 4.2. Derive the functional relationship shown below between the thickness of an absorber and intensity of a photon beam attenuated by the absorber. Explain the terms.[7]
\(I(x)\)
\begin{solution}
    
\end{solution}
\part 4.4. If a 0.2-cm thickness of material transmits 25% of a monoenergetic beam of photons, calculate the half-value layer (HVL) of the beam for that material.
[4]
\begin{solution}
    
\end{solution}
\end{parts}

\end{questions}

\begin{questions}
    \question \textbf{QUESTION 5 (20 MARKS)}
    \begin{parts}
        \part 5.1. Describe the term stopping power of a material. [2 marks]        
        \part 5.2.Explain the following quantities that are used for the purpose of quantifying radiation. (a) Activity A,
        (b) Kerma (K),
        and (c) Dose (H) [6 marks]
        \part 5.3. Write down the most important characteristics of radionuclides used in external beam radiotherapy [4]
        \part 5.4. (a)  The Bragg curve below shows a plot of the variation of ionization density as a function of distance travelled by an alpha (x) particle in air. Explain that form of the curve shown below. [6 marks]
        (b) What is the practical application of the enhanced ionization in the Bragg peak? [2 marks]

\begin{solution}
    
\end{solution}

    \end{parts}
\end{questions}

\end{document}
