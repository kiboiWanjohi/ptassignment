%%%%%%%%%%%%%%%%%%%%%%%%%%%%%%%%%%%%%%%%%%%%%%%%%%%%%%%%%%%%%%%
%
% Welcome to Overleaf --- just edit your LaTeX on the left,
% and we'll compile it for you on the right. If you give
% someone the link to this page, they can edit at the same
% time. See the help menu above for more info. Enjoy!
%
%%%%%%%%%%%%%%%%%%%%%%%%%%%%%%%%%%%%%%%%%%%%%%%%%%%%%%%%%%%%%%%
%\title{Math 453 HW 1}
\documentclass[addpoints]{exam}

\usepackage{amsmath,enumitem,wrapfig}
\usepackage{tikz}

\newcommand{\StudentName}{Heejin Mun}
\newcommand{\AssignmentName}{HW 1}

\pagestyle{headandfoot}
\runningheadrule
\firstpageheadrule
\firstpageheader{Math 453}{\StudentName}{\AssignmentName}
\runningheader{Math 453}{\StudentName}{\AssignmentName}
\firstpagefooter{}{}{}
\runningfooter{}{}{}

\printanswers

\begin{document}


Organize your work and show any work that you want credit for. Use full sentences where possible.

\begin{questions}

\question \textbf{M1}
\begin{parts}
\part
Consider the arithmetic computation below.
\begin{align}
3+4[5-12]-6(3) +(4+0)
&= 3+4[5-12]-6(3) +4\\
&= 4[5-12]-6(3) +4+3\\
&= 20-48-18 +4+3 \\
&=-39.\notag
\end{align}
For each of the steps (1), (2), and (3) identify which of the Axioms of Integer Arithmetic are used in the simplification step.

\begin{solution}
$3+4[5-12]-6(3) +4$....(1) additive identity\\
$4[5-12]-6(3) +4+3$ ....(2) commutativity of addition\\ 
$20-48-18 +4+3$ ....(3) distributive\\

\end{solution}

\part Create and simplify an expression that uses associativity of addition, multiplicative identity, and the distributive law.

\begin{solution}\\
 (1) Associativity of addition\\ 
     =a + (b + c) = (a + b) + c\\
 (2) Multiplication identity\\ 
     1*a =a\\
 (3) Distributive law\\
 	 a (b + c) = ab + ac\\
 Example) \\
 $3 + 4(6 + 4) + (7(3) + 5(1))$\\
 = $(3 + 4(6 + 4)) + 7(3) + 5(1)$ ...used (1)\\
 = $(3 + (24 + 14)) + 7(3) + 5(1)$ ....used (3)\\
 = $(3 + 24 + 14) + 21 + 5$ .... used (2)\\
 = $67$ \\
\end{solution}

\end{parts}



\question \textbf{M2}
For each statement below determine whether each statement is correct for integers $a$, $b$, and $c$. If the statement is correct, then prove it. If the statement is incorrect, then modify it so that it is correct. Be sure to state which Order Axiom(s) you have applied.
\begin{parts}
\part If $a<b$, then $c\cdot a< c\cdot b$.\\
\begin{solution}
incorrect.\\
If $a<b$, then $c\cdot a< c\cdot b$ when $c>0$
\end{solution}


\part If $a<b$, then $a+c<b+c$.
\begin{solution}
Correct\\
If $a<b$, then we can assume that $a + r= b$, where r is in Z\\
Hence, $a + c + r = b + c$, when c is in Z\\
Therefore, then $a+c<b+c$.
\end{solution}

\part If $a<b$, $b<c$, and $c<d$, then $a<d$.

\begin{solution}
Correct\\
with the same way part (b)\\
Since  If $a<b$, then If $a+r=b$, when r is in Z\\
and  $b<c$, so it follows that $b+r=c$, it is also same with $a+r+r=b+r=c$\\
also, if $c<d$, then $c+r=d$, it is same with $a+r+r+r=b+r+r=c+r=d$\\
Therefore, $a+3r=d$\\
Hence, $a<d$

\end{solution}

\part If $a\not > b$ and $a\not< b$, then $a=b$.

\begin{solution}
Given that $a\not > b$ and $a\not< b$\\
For $a\not > b$ \\
then, it can be either $a < b$ or $a=b$, but $a < b$ is a contradiction by given $a\not< b$.\\
For $a\not< b$,\\
then,  it can be either $a > b$ or $a=b$, but $a > b$ is a contradiction by given $a\not> b$.\\
Therefore, $a=b$.
\end{solution}
\end{parts}


\question \textbf{M3}
\begin{parts}
\part
Find the flaw in the following argument.
\begin{quote}
To solve $x(x+4)=x(2x-8)$ we divide both sides by $x$ (or apply Theorem 1.11) to get $x+4=2x-8$. Subtract $(x-8)$ from both sides to obtain $12=x$, so the solution is $x=12$. 
\end{quote}
\begin{solution}
x could be 0, so we can't divide both sides by x.
\end{solution}

\part 
Find the flaw in the following argument.
\begin{quote}
To solve $x(x-4)=12$ we factor the left-hand side and set the factors equal to zero $x=0$ and $x-4=0$ and conclude that $x=0,4$.  
\end{quote}
\begin{solution}
$x(x-4)=12$ should be $x^2-4x-12=0$ by distributive law\\
then, $(x-6)(x+2)=0$
Therefore, $x=6, -2$
\end{solution}
\end{parts}

\question \textbf{M4}
\begin{parts}
\part Work Exercise 1 from Investigation 1 (uniqueness of additive inverses).
\begin{solution}
If some integer a has two additive inverses, which are b and c, then we can write $a+b=0$ and $a+c=0$.\\
Then, $a+b=a+c=0$.\\
Since a is integer, we can say $b=c$
\end{solution}


\part Work Exercise 2 from Investigation 1 (additive cancellation).
\begin{solution}
Given that $a+b=a+c$, where a, b, and c are in integers Z.\\
$(-a)=(-a)$ ...(-a) exists by additive inverse.\\
Now, we can add $(-a)$ from both sides,\\
Then, $(a+(-a))+b=(a+(-a))+c$, by associative law,\\
$0+b=0+c$\\
Therefore, $b=c$ by additive identity.
\end{solution}
\part 

TECHNICAL UNIVERSITY OF KENYA
FACULTY OF APPLIED SCIENCES AND TECHNOLOGY SCHOOL OF PHYSICS AND EARTH SCIENCES
DEPARTMENT OF TECHNICAL AND APPLIED PHYSICS
MASTER OF SCIENCE IN MEDICAL PHYSICS (M Sc. Medical Physics)
YEAR: (1) SEMESTER: (ONE)
FEBRUARY 2023 EXAMINATIONS SERIES
PAPER:
RADIATION PHYSICS
TIME:
UNIT CODE: (SPTV 7112)
3 HOURS INSTRUCTIONS
You should have the following for this examination:
Answer booklet
Scientific calculator
This paper contains FIVE questions.
Attempt question ONE and any other TWO questions.
Question ONE carries 30 marks, all other questions carry 20marks
1u931.5 MeV
lev | c2
88
228Ra = 226.025403 u
Atomic Rest Masses
222Rn = 222.017571 u He = 4.002603 u
86
18F = 18.000938 u ==
==
180= 17.999160 u
Nuclear Rest Masses
228Ra = 210496.3482
86
222Rn 206764.0985
moc =
3727.3791
Nuclear Binding Energies
226
Ra
= 1731.610
222 864
Rn= 1708. 185
He = 28.296
137Cs = 136.907084 u
137 Ba= 136.905821 u
0| Page
\begin{solution}
    
\end{solution}
\end{parts}
\begin{parts}
\part 

QUESTION 3 (20 MARKS)
3.1. Briefly describe the areas where Ionizing radiation is used. [8 marks]
3.2 Electron Capture and B are competing processes. Write down conditions that are required for one process to proceed over the other process. [4 marks]
3.2.
Discuss any two ways or mechanisms by which photons can interact with matter that play a very important role in therapeutic as well as in diagnostic medical physics. [8 marks]
QUESTION 4 (20 MARKS)
4.1. Explain the following terms.
4.2.
4.3.
4.4.
(a) the linear attenuation coefficient and Mass attenuation coefficient [2 marks] (b) First half-value layer HVL,, the second half-value layer HVL2 [2 marks]
Derive the functional relationship shown below between the activity of a radioactive substance and time. Explain the terms. [7 marks]
A = Age-At
Draw a typical curve for the radioactive substance as given by the expression in Question 4.2. Briefly explain how you will determine the 2. [6 marks]
The half-life of strontium-90, (Sr), is 28.8 years. Calculate the (a) decay constant and (b) the initial activity of 1.00g of the material. [3 marks]
QUESTION 5 (20 MARKS)
5.1.
5.2.
5.3
Describe the term stopping power of a material. [2 marks]
Explain the following quantities that are used for the purpose of quantifying radiation. (a) Activity A,
(b) Kerma (K),
and (c) Dose (H) [6 marks]
Write down factors that are considered to choose a radiation beam and dose prescription in treatment of disease with radiation. [4 marks]
5.4 (a) The Bragg curve below shows a plot of the variation of ionization density as a function of distance travelled by an alpha (∞) particle in air. Explain that form of the curve shown below. [6 marks]
(b) What is the practical application of the enhanced ionization in the Bragg peak? [2 marks]
0.0
10
2.0 Distance travelled (cm)
30
2 | Page
\end{parts}

\begin{parts}
    \part 

QUESTION 1 (30 MARKS)
1.1.
Explain the following terms
(a)
(b)
(c)
Compton or recoil electrons, Beta particles and Pair Production electrons [3 marks] Synchrotron radiation and Annihilation quanta
Remote afterloading technique and Neutron activation
[2 marks]
[2 marks]
1.2. (a) Write a nuclear equation when cesium-137 decays into barium-137. [3 marks] (b) Calculate the decay energy Q- for the B- decay of cesium-137 into barium-137. [5mks] (c) When Fluorodeoxyglucose (FDG) is labelled with radionuclide fluorine-18 it can be injected intravenously into a patient for use in positron emission tomography (PET) functional imaging. What are some of the uses of the FDG PET? [4 marks]
1.3. Which physical quantities are conserved during a nuclear transformation? [5 marks] Briefly explain directly ionizing radiation and indirectly ionizing radiation. Give examples for each. [6 marks]
1.4
2.1.
2.2.
QUESTION 2 (20 MARKS)
[4 marks]
Write down the most important characteristics of radionuclides used in external beam radiotherapy.
88
The figure below shows an energy level diagram for 228Ra (radium) decaying through a-decay into 222Ra (radon).
86
Relative mass-energy (MeV)
228Ra
4.78
a decay (5.4%)
Ex2
a decay (94.6%)
Ex1
0.18
0
y decay Eye
222 86
Rn
(a)
(b)
(c)
Explain the decay scheme shown above as it ends at the ground state. [5 marks] Calculate the decay energy (Q) for the x decay of 228Ra (radium). [5 marks] Calculate the kinetic energy Ex1 of the x- particle.
88
[3 marks]
(d)
Explain why modern brachytherapy is now carried out with other radionuclides (e.g., iridium-192, cesium-137, iodine-125, etc.) instead of the radium-226 which was very popular in the past century?
[3 marks]
1| Page
NT


3.1
3.2.
3.3.
QUESTION 3 (20 MARKS)
Electron Capture and B+ are competing processes. Write down conditions that are required for one process to proceed over the other process.
[4] Discuss any two ways or mechanisms by which photons can interact with matter that play a very important role in therapeutic as well as in diagnostic medical physics. [8] An X-ray photon beam of photon energy 15 eV is allowed to fall on a metal surface. If the threshold frequency for the metal is 1.2 x 1015 Hz, then find the maximum kinetic energy gained by a photoelectron, the maximum speed of a photoelectron, and the stopping potential corresponding to the maximum kinetic energy.
[8]
QUESTION 4 (10 MARKS)
4.1.
Explain the following terms.
(a) the linear attenuation coefficient and Mass attenuation coefficient
[2]
(b) First half-value layer HVL1, the second half-value layer HVL2 and tenth value layer [3]
4.2.
Derive the functional relationship shown below between the activity of a radioactive substance and time. Explain the terms.
[7]
A =
Age-At
4.3.
Draw a typical curve for the radioactive substance as given by the expression in Question 4.2. Briefly explain how you will determine the 2.
[6]
4.4.
If a 0.2-cm thickness of material transmits 25% of a monoenergetic beam of photons, calculate the half-value layer (HVL) of the beam for that material.
[2]
QUESTION 5 (20 MARKS)
5.1.
Describe the term stopping power of a material.
[2]
5.2.
(b) Kerma (K),
Explain the following quantities that are used for the purpose of quantifying radiation. (a) Activity A, and (c) Equivalent Dose (H)
[6]
5.3
Write down factors that are considered to choose a radiation beam and dose prescription in treatment of disease with radiation
[4]
↓
5.4
Draw and explain the Bragg curve. What is the importance of the enhanced ionization in the Bragg peak?
[8]
\end{parts}

\begin{parts}
    \part 

QUESTION 1 (30 MARKS)
Photoelectrons, Thermions and Megavoltage electrons
1.1.
Explain the following terms
(a)
(b)
(c)
Characteristic (fluorescence) x-rays and Bremsstrahlung x-rays Remote afterloading technique and Neutron activation
1.2. (a) Write a nuclear equation when fluorine-18 decays into oxygen-18.
B+22
+66
[3]
[2]
[2]
[3]
[5]
(b) Calculate the decay energy Q+¿¿ for the B+ decay of fluorine-18 into oxygen-18. (c) When Fluorodeoxyglucose (FDG) is labelled with radionuclide fluorine-18 it can be injected intravenously into a patient for use in positron emission tomography (PET) functional imaging. What are some of the uses of the FDG PET?
4.1.
[4]
1.3.
The half-life of strontium-90, ¿, is 28.8 years. Calculate the (a) decay constant and (b) the [5] initial activity of 1.00g of the material.
ERSI
DEAR
[6]
1.4.
Briefly describe the areas where Ionizing radiation is used.
AND
QUESTION 2 (20 MARKS)
DICAL F
2.1.
Which physical quantities are conserved during a nuclear transformation?
[5]
PHYS
2.2.
The figure below shows an energy level diagram for 228Ra (radium) decaying through a-decay into 22Ra (radon).
226, 88
Relative mass-energy (MeV)
228Ra
4.78
a decay (5.4%) Ex2
a decay (94.6%) Ex1
0.18
y decay Ey
0
222Rn
86
и
(a)
Explain the decay scheme shown above as it ends at the ground state.
[5]
(b)
Calculate the decay energy (Q.) for the
decay of 228Ra (radium).
T
[5]
requer
(c)
Calculate the kinetic energy E.1 of the
- particle.
[3]
(d)
Explain why modern brachytherapy is now carried out with other radionuclides (e.g., iridium-192, cesium-137, iodine-125, etc.) instead of the radium-226 which was very popular in the past century?
[2]
QUESTION 3 (20 MARKS)
3.1.
Write down the possible outcomes when an X-ray beam or gamma radiation passes through an object.
[3]
1.01uF)
3.2.
Discuss any two ways or mechanisms by which photons can interact with matter that play a very important role in therapeutic as well as in diagnostic medical physics.
[8]
3.3.
An X-ray photon beam of photon energy 10 eV is allowed to fall on a metal surface. If the threshold frequency for the metal is 1.5 × 1015 Hz, then find the maximum kinetic energy gained by a photoelectron, the maximum speed of a photoelectron, and the stopping potential corresponding to the maximum kinetic energy.
[9]
1 | Page


QUESTION 3 (20 MARKS)
3.1. Briefly describe the areas where Ionizing radiation is used. [8 marks]
3.2
3.2.
Electron Capture and B+ are competing processes. Write down conditions that are required for one process to proceed over the other process. [4 marks]
Discuss any two ways or mechanisms by which photons can interact with matter that play a very important role in therapeutic as well as in diagnostic medical physics. [8 marks]
QUESTION 4 (20 MARKS)
4.1. Explain the following terms.
4.2.
4.3.
4.4.
(a) the linear attenuation coefficient and Mass attenuation coefficient [2 marks]
(b) First half-value layer HVL1, the second half-value layer HVL2 [2 marks]
Derive the functional relationship shown below between the activity of a radioactive substance and time. Explain the terms. [7 marks]
-λt
A = Age-
Draw a typical curve for the radioactive substance as given by the expression in Question 4.2. Briefly explain how you will determine the 2. [6 marks]
90
The half-life of strontium-90, (38Sr), is 28.8 years. Calculate the (a) decay constant and (b) the initial activity of 1.00g of the material. [3 marks]
QUESTION 5 (20 MARKS)
5.1.
Describe the term stopping power of a material. [2 marks]
5.2.
5.3
Explain the following quantities that are used for the purpose of quantifying radiation. (a) Activity A,
(b) Kerma (K),
and (c) Dose (H) [6 marks]
Write down factors that are considered to choose a radiation beam and dose prescription in treatment of disease with radiation. [4 marks]
5.4 (a) The Bragg curve below shows a plot of the variation of ionization density as a function of distance travelled by an alpha (x) particle in air. Explain that form of the curve shown below. [6 marks]
(b) What is the practical application of the enhanced ionization in the Bragg peak? [2 marks]
lonization density
0.0
1.0
2.0
3.0
Distance travelled (cm)
2 | Page


4.1.
QUESTION 4 (10 MARKS)
Explain the following terms.
(a) the linear attenuation coefficient and Mass attenuation coefficient
[2]
(b) First half-value layer HVL1, the second half-value layer HVL2 and tenth value layer [3]
4.2.
Derive the functional relationship shown below between the thickness of an absorber and intensity of a photon beam attenuated by the absorber. Explain the terms.
[7]
I(x) = I。ex
4.3.
μα
Draw a typical attenuation curve for a mono-energetic photon beam with incident intensity lo against absorber thickness x, as given by the expression in Question 4.2. [4]
4.4. If a 0.2-cm thickness of material transmits 25% of a monoenergetic beam of photons, calculate the half-value layer (HVL) of the beam for that material.
[4]
QUESTION 5 (20 MARKS)
J
5.1. Describe the term stopping power of a material.
[2]
5.2.
Explain the following quantities that are used for the purpose of quantifying radiation. (a) Exposure X, (b) Kerma (K),
and (c) Dose (D)
[6]
5.3.
Write down the most important characteristics of radionuclides used in external beam radiotherapy.
[4]
山
5.4 (a) The Bragg curve below shows a plot of the variation of ionization density as a function of distance travelled by an alpha (x) particle in air. Explain that form of the curve shown below.
(b) What is the practical application of the enhanced ionization in the Bragg peak?
[6]
[2]
Ionization density
0.0
1.0
2.0
3.0
Distance travelled (cm)
2 | Page



4.1.
QUESTION 4 (10 MARKS)
Explain the following terms.
(a) the linear attenuation coefficient and Mass attenuation coefficient
[2]
(b) First half-value layer HVL1, the second half-value layer HVL2 and tenth value layer [3]
4.2.
Derive the functional relationship shown below between the thickness of an absorber and intensity of a photon beam attenuated by the absorber. Explain the terms.
[7]
I(x) = I。ex
4.3.
μα
Draw a typical attenuation curve for a mono-energetic photon beam with incident intensity lo against absorber thickness x, as given by the expression in Question 4.2. [4]
4.4. If a 0.2-cm thickness of material transmits 25% of a monoenergetic beam of photons, calculate the half-value layer (HVL) of the beam for that material.
[4]
QUESTION 5 (20 MARKS)
J
5.1. Describe the term stopping power of a material.
[2]
5.2.
Explain the following quantities that are used for the purpose of quantifying radiation. (a) Exposure X, (b) Kerma (K),
and (c) Dose (D)
[6]
5.3.
Write down the most important characteristics of radionuclides used in external beam radiotherapy.
[4]
山
5.4 (a) The Bragg curve below shows a plot of the variation of ionization density as a function of distance travelled by an alpha (x) particle in air. Explain that form of the curve shown below.
(b) What is the practical application of the enhanced ionization in the Bragg peak?
[6]
[2]
Ionization density
0.0
1.0
2.0
3.0
Distance travelled (cm)
2 | Page
\end{parts}
\end{questions}
\end{document}
