%%%%%%%%%%%%%%%%%%%%%%%%%%%%%%%%%%%%%%%%%%%%%%%%%%%%%%%%%%%%%%%
%
% Welcome to Overleaf --- just edit your LaTeX on the left,
% and we'll compile it for you on the right. If you give
% someone the link to this page, they can edit at the same
% time. See the help menu above for more info. Enjoy!
%
%%%%%%%%%%%%%%%%%%%%%%%%%%%%%%%%%%%%%%%%%%%%%%%%%%%%%%%%%%%%%%%
%\title{Math 453 HW 1}
\documentclass[addpoints]{exam}
\usepackage{hyperref}
\usepackage{amsmath,enumitem,wrapfig}
\usepackage{tikz}

\newcommand{\StudentName}{Radiation Physics}
\newcommand{\AssignmentName}{HW 1}

\pagestyle{headandfoot}
\runningheadrule
\firstpageheadrule
\firstpageheader{Math 453}{\StudentName}{\AssignmentName}
\runningheader{Math 453}{\StudentName}{\AssignmentName}
\firstpagefooter{}{}{}
\runningfooter{}{}{}

\printanswers

\begin{document}


Organize your work and show any work that you want credit for. Use full sentences where possible.

\begin{questions}
    \question \textbf{QUESTION 1 (30 MARKS)}
    \begin{parts}
        \part 
        1.1.
        Explain the following terms
        (a) Compton or recoil electrons, Beta particles and Pair Production electrons [3 marks] 
        (b) Synchrotron radiation and Annihilation quanta [2 marks]
        (c) Remote afterloading technique and Neutron activation [2 marks]
        \begin{solution}
        A Compton/Recoil electron refers to an electron that is ejected from an atom during the Compton scattering process. Compton scattering is a phenomenon in quantum physics where a photon (usually an X-ray or gamma-ray photon) collides with an electron, and as a result of the collision, the photon loses energy, and the electron is ejected from its orbit around an atomic nucleus.
        
A beta particle, also known as a beta ray or beta radiation, is a high-energy, high-speed electron or positron emitted by the radioactive decay of an atomic nucleus during the process of beta decay.A beta electron refers to an electron that is emitted during a type of radioactive decay known as beta decay. Beta decay involves the transformation of a neutron into a proton or a proton into a neutron within an atomic nucleus, accompanied by the emission of a beta particle.
        Synchrotron radiation refers to electromagnetic radiation emitted by charged
particles following a curved trajectory in free space under the influence of a
magnetic field.
Annihilation quanta refers to the high-energy photons (gamma rays) that are produced when a particle and its antiparticle collide and annihilate each other. During the annihilation process, the total energy and momentum of the initial pair are conserved and distributed among the resulting particles. 
Photons resulting from positron–electron annihilation is the annhiliation quanta. 
The remote afterloading technique is a method used in brachytherapy, a form of radiation therapy, where the radioactive sources are placed within or near the treatment area.
        \end{solution}
        \part 
        1.2. (a) Write a nuclear equation when cesium-137 decays into barium-137. [3 marks] \newline
        (b) Calculate the decay energy Q- for the B- decay of cesium-137 into barium-137. [5mks] \newline
        (c) When Fluorodeoxyglucose (FDG) is labelled with radionuclide fluorine-18 it can be injected intravenously into a patient for use in positron emission tomography (PET) functional imaging. What are some of the uses of the FDG PET? [4 marks]\newline
        \begin{solution}
            1.2 (a) \(^{137}_{55}Cs \rightarrow ^{137}_{56}Ba + e^- + \bar{\nu}_e\) Beta decay: The unstable cesium-137 nucleus undergoes beta decay, emitting a beta particle (negatron or electron) and an antineutrino. This increases the neutron number by one while decreasing the proton number by one, transforming the cesium into a metastable excited state of barium (barium-137m).\newline
            (b) The decay energy (Q-value) for beta decay can be calculated using the mass-energy equivalence principle. The 
Q-value is the difference in mass between the parent nucleus and the daughter nucleus, taking into account the masses of the emitted beta particle (electron) and antineutrino.
         The 
Q-value is given by the equation:  \(Q = (M_{\text{initial}} - M_{\text{final}})c^2\)
\neline \(M_{\text{initial}} = 136.907 \, \text{u}\)
\newline \(M_{\text{final}} = M_{^{137}_{56}Ba} + M_{e^-} + M_{\bar{\nu}_e}\)
\newline \(M_{\text{final}} = 136.905 \, \text{u} + 0.00054857 \, \text{u} + (\text{negligible})\)
\newline \(Q = (136.907 \, \text{u} - (136.905 \, \text{u} + 0.00054857 \, \text{u})) \cdot (931.5 \, \text{MeV/u})\)
\newline \(Q \approx 1.949 \, \text{MeV}\)
\newline (c) FDG visualizes the glucose
metabolism and is hence taken up and trapped in organs with
extensive glucose metabolism such as brain and heart and
aberrant growth. Because of this FDG can be applied in
the diagnostic imaging of cancer((FDG is preferentially taken up and accumulated in cancerous tissue.)), infammation((In chronic infections, immune cells and pathogens at infection sites tend to take up more glucose. Localized FDG uptake can therefore pinpoint sites of infection when other testing is inconclusive.)), cardiology((FDG-PET can identify areas of decreased blood flow and metabolic activity in heart tissue caused by coronary artery disease))
and neurology
        \end{solution}
        \part 1.4. Briefly explain directly ionizing radiation and indirectly ionizing radiation. Give examples for each. [6 marks]
       \begin{solution}
        Directly ionizing radiation involves charged particles that interact directly with the atoms they encounter, causing ionization. Examples include alpha and beta particles, which carry a positive and negative charge, respectively, and interact directly with atomic electrons through coulombic forces. On the other hand, indirectly ionizing radiation consists of uncharged particles or electromagnetic waves that produce ionization by first setting free a charged particle, which then interacts with the atoms. Examples of indirectly ionizing radiation include gamma rays, X-rays, and neutrons, which produce ionization indirectly through their interactions with atomic nuclei.
       \end{solution}
       \part 1.3. Write down the most important characteristics of radionuclides used in external beam radiotherapy. [5 marks]
\begin{solution}
    The most important characteristics of radionuclides used in external beam radiotherapy are:

1. Type of decay - Radionuclides that decay with gamma emission are preferred, since gamma rays have high penetrating power to reach deep tumors with external beams. Examples include cobalt-60, iridium-192.

2. Half-life - The half life must be long enough to deliver adequate dose over treatment timescales, but short enough to minimize unwanted dose after completion. Half-lives for therapy are typically in the range of days to months, such as 5.3 years for cobalt-60.

3. Gamma ray energy - Gamma rays for therapy are high energy (usually >1 MeV) in order to ensure sufficient dose deposition at depth while minimizing skin dose. Higher energy photons also allow for better collimation and penetration. 

4. Source production - The radioactive sources must be reasonably producible and available at radiation facilities. Ideal sources like cobalt-60 can be produced in nuclear reactors exposed to neutron flux.

5. Source safety - Containing the gamma radiation safely requires special shielding containers and radiation safety controls to protect staff and surroundings. The sources should ideally be sealed to prevent contamination.
\end{solution}
    \end{parts}
    
\end{questions}

\begin{questions}
    \question \textbf{Question 2 (20 marks)}
    \begin{parts}
        \part 
        2.1. Which physical quantities are conserved during a nuclear transformation? [5 marks]
        \begin{solution}
            In each nuclear transformation a number of physical quantities must be
conserved. The most important of these quantities are:
1. Total energy
2. Momentum
3. Charge
4. Atomic number
5. Atomic mass number (number of nucleons)
        \end{solution}
        \part 2.2. The figure below shows an energy level diagram for
        (a) Explain the decay scheme shown above as it ends at the ground state. [5 marks] Calculate the decay energy (Q) for the x decay of 228Ra (radium). [5 marks] Calculate the kinetic energy Ex1 of the x- particle.

        (b) Calculate the decay energy \((Q_B)\) and \((Q_EC)\) for the decay of iridium-192 [8 marks]
        (c) Explain Briefly shy iridium-192 is more widely used in brachytherapy rather than the traditional radium or caesium. What is the half-life of iridium-192? [3 marks]
        
    \end{parts}
        \begin{solution}
            \textbf{answer is at book **Radiation Physics for Medical Physicists page 499**}
        \end{solution}
\end{questions}

\begin{questions}
    \question \textbf{Question 3 (20 marks)}
    \part 3.1. Electron Capture and B+ are competing processes. Write down conditions that are required for one process to proceed over the other process. [4 marks]
    \newline Electron capture is an alternative to B+ decay. Comparison of Eqs. 3.61 and 
3.66 shows that nuclei that cannot experience B+ decay can undergo EC, since 
a smaller mass difference is required for the latter process. Of course, EC is 
always possible if B+ decay is decays both by B+ and EC.
    \begin{solution}
        The condition on EC decay is that the parent
atomic mass M(P) simply exceeds the daughter atomic mass M(D), while
the condition on β+ decay is that the parent atomic mass exceeds that of the
daughter by a minimum of two electron masses.
    \end{solution}
    \part 3.2 Discuss any two ways or mechanisms by which photons can interact with matter that play a very important role in therapeutic as well as in diagnostic medical physics. [8 marks]
    \begin{solution}
        The six modes of photon interactions with absorber atoms of relevance to
medical physics are:
1. Photoelectric effect.
2. Rayleigh scattering.
3. Compton effect.
4. Nuclear pair production.
5. Electronic pair production (also known as triplet production).
6. Photonuclear reactions (also known as photodisintegration).
    \end{solution}
    \part 3.3. An X-ray photon beam of photon energy 15 eV is allowed to fall on a metal surface. If the threshold frequency for the metal is \(1.2 × 1015 Hz\) , then find the maximum kinetic energy gained by a photoelectron, the maximum speed of a photoelectron, and the stopping potential corresponding to the maximum kinetic energy.
    [8 marks]
    \begin{solution}
        To find the maximum kinetic energy (\(K_{\text{max}}\)), the maximum speed of a photoelectron (\(v_{\text{max}}\)), and the stopping potential (\(V_{\text{stop}}\)), we can use the following relationships in the photoelectric effect:

1. The energy of a photon is given by \(E = hf\), where \(h\) is Planck's constant (\(6.626 \times 10^{-34} \, \text{J}\cdot\text{s}\)) and \(f\) is the frequency of the photon.

2. The energy required to remove an electron from the metal surface is given by \(E_{\text{kin}} = hf - \phi\), where \(\phi\) is the work function of the metal.

3. The kinetic energy of the photoelectron is related to its speed by \(E_{\text{kin}} = \frac{1}{2} m v^2\), where \(m\) is the mass of the electron (\(9.11 \times 10^{-31} \, \text{kg}\)).

4. The stopping potential (\(V_{\text{stop}}\)) is related to the maximum kinetic energy by \(eV_{\text{stop}} = K_{\text{max}}\), where \(e\) is the elementary charge (\(1.602 \times 10^{-19} \, \text{C}\)).

Now, let's calculate each of these values:

Given data:
- Photon energy (\(E\)) = 15 eV
- Threshold frequency (\(f_{\text{threshold}}\)) = \(1.2 \times 10^{15} \, \text{Hz}\)

1. Calculate the frequency of the incident photon using \(E = hf\):
   \[f = \frac{E}{h} \]
   \[f = \frac{15 \, \text{eV}}{6.626 \times 10^{-34} \, \text{J}\cdot\text{s}} \]
   \[f \approx 2.263 \times 10^{15} \, \text{Hz}\]

2. Determine the kinetic energy of the photoelectron using \(E_{\text{kin}} = hf - \phi\):
   \[E_{\text{kin}} = hf - \phi \]
   \[E_{\text{kin}} = (6.626 \times 10^{-34} \, \text{J}\cdot\text{s})(2.263 \times 10^{15} \, \text{Hz}) - (1.2 \times 10^{-19} \, \text{J}) \]
   \[E_{\text{kin}} \approx 2.223 \times 10^{-17} \, \text{J}\]

3. Calculate the maximum speed of the photoelectron using \(E_{\text{kin}} = \frac{1}{2} m v^2\):
   \[v_{\text{max}} = \sqrt{\frac{2E_{\text{kin}}}{m}} \]
   \[v_{\text{max}} = \sqrt{\frac{2(2.223 \times 10^{-17} \, \text{J})}{9.11 \times 10^{-31} \, \text{kg}}} \]
   \[v_{\text{max}} \approx 5.69 \times 10^6 \, \text{m/s}\]

4. Find the stopping potential (\(V_{\text{stop}}\)) using \(eV_{\text{stop}} = K_{\text{max}}\):
   \[V_{\text{stop}} = \frac{K_{\text{max}}}{e} \]
   \[V_{\text{stop}} = \frac{2.223 \times 10^{-17} \, \text{J}}{1.602 \times 10^{-19} \, \text{C}} \]
   \[V_{\text{stop}} \approx 13.89 \, \text{V}\]

So, the results are:
- Maximum kinetic energy (\(K_{\text{max}}\)) ≈ \(2.223 \times 10^{-17} \, \text{J}\)
- Maximum speed of the photoelectron (\(v_{\text{max}}\)) ≈ \(5.69 \times 10^6 \, \text{m/s}\)
- Stopping potential (\(V_{\text{stop}}\)) ≈ \(13.89 \, \text{V}\)
    \end{solution}        
\end{questions}


\begin{questions}
    \question \textbf{QUESTION 4 (20 MARKS)}
    \part 
    4.1. Explain the following terms. (a) the linear attenuation coefficient and Mass attenuation coefficient [2 marks] (b) First half-value layer HVL,, the second half-value layer HVL2 [2 marks]
    \begin{solution}
        \newline (a)  linear attenuation coefficient is described as the probability per unit path length
that a photon will have an interaction with the absorber. mass attenuation coefficient μm is defined as the linear attenuation
coefficient μ divided by the mass per unit volume of the absorber (absorber mass density) ρ. \newline (b) 
HVL1 or x1/2 is the thickness of a homogeneous
absorber that attenuates the narrow beam intensity I(0) to one-half (50 %)
of the original intensity, i.e., Ix1/2= 0.5I(0). second half value layer is defined as the
thickness of the absorber that attenuates the narrow beam intensity from 0.5I(0) to 0.25I(0). (( If its shitty look into Radiation Physics for Medical Physicists page 283-286))
    \end{solution}
    \part 4.2. Derive the functional relationship shown below between the activity of a radioactive substance and time. \begin{equation}
        A = A_oe-2z
    \end{equation}
    \begin{solution}
        (((in notes )))
    \end{solution}
    \part 4.3. Draw a typical curve for the radioactive substance as given by the expression in Question 4.2. Briefly explain how you will determine the 2. [6 marks]
    \begin{solution}
        ((( notes )))
    \end{solution}
\end{questions}

\begin{questions}
    
    \question \textbf{QUESTION 5 (20 MARKS)}
    \part 5.1. Describe the term stopping power of a material.[2]
    \part 5.2. Explain the following quantities that are used for the purpose of quantifying radiation. (a) Activity A, (b) Kerma (K), (c) Equivalent Dose (H)[6]
    \begin{solution}
        (((notes)))
    \end{solution}
    \part 5.3 Write down factors that are considered to choose a radiation beam and dose prescription in treatment of disease with radiation[4]
    \begin{solution}
        Several important factors are considered when choosing radiation beam and dose prescription in disease treatment:

1. Tumor/target characteristics - size, location, stage, histology, volume, radioresistance, proximity to organs at risk

2. Intent of treatment - curative, adjuvant, palliative, symptom management

3. Normal tissue tolerance - tolerance doses for surrounding sensitive structures like spinal cord, optic nerves, etc. 

4. Fractionation scheme - single dose vs multiple fractions; higher fractional doses have greater cell-killing effect 

5. Total dose - higher doses increase tumor control probability but also normal tissue complications; typical prescription target doses range from 45 Gy to 80 Gy

6. Dose heterogeneity and distribution - more uniform target coverage is desirable; hot and cold spots should be minimized 

7. Radiation modality - external beam, brachytherapy, or systemic therapy depending on tumor accessibility

8. Beam energies - high enough to reach target depth while sparing superficial tissues

9. Patient factors - medical history, comorbidities, functional status

The radiation oncologist selects the optimal combination of beam angles, beam modifiers, total dose, dose per fraction and radiation type to maximize the therapeutic ratio between tumor cell kill and normal tissue preservation given the clinical context.
    \end{solution}
    \part5.4 Draw and explain the Bragg curve. What is the importance of the enhanced ionization in the Bragg peak?[8]
    \begin{solution}
        (((notes))
    \end{solution}
\end{questions}

\begin{questions}
    \question \textbf{QUESTION 1 (30 MARKS)}
    \begin{parts}
  
        \part 
       (((All this shit is repeated in section (1) above ......)))
       
    \end{parts}
    
\end{questions}

\begin{questions}
    \question \textbf{QUESTION 2 (20 MARKS)}
    \begin{parts}
      
        \part 2.2.The figure below shows an energy level diagram for 228Ra (radium) decaying through a-decay into 22Ra (radon).
        \begin{solution}
            ((books and notes))
        \end{solution}
        \part (a)Explain the decay scheme shown above as it ends at the ground state.[5 marks]
        \begin{solution}
            ((notes and Books))
        \end{solution}
        \part (b)Calculate the decay energy (Q.) for the decay of 228Ra (radium).[5 marks]
        \begin{solution}
            ((notes and Books))
        \end{solution}
        \part (c)Calculate the kinetic energy E.1 of the- particle.[3 marks]
        \begin{solution}
            ((notes and Books))
        \end{solution}
        \part (d) Explain why modern brachytherapy is now carried out with other radionuclides (e.g., iridium-192, cesium-137, iodine-125, etc.) instead of the radium-226 which was very popular in the past century?[3 marks]
        \begin{solution} e. While radium-based brachytherapy was
very popular in the past century, modern brachytherapy is now carried out with other radionuclides (e.g., iridium-192, cesium-137, iodine-125, etc.) that do not pose safety hazards associated with the radon-222 gas that may leak through damaged radium-226 source encapsulation.

\textbf{All this answers are in Radiation Physics for Medical Physicists page 480 thereabout}
        \end{solution}
    \end{parts}
\end{questions}


\begin{questions}
    \question \textbf{Question 3 (20 marks)}
    \begin{parts}
        \part 3.1. Briefly describe the areas where Ionizing radiation is used. [8 marks]
        \begin{solution}
        Ionizing radiation has applications in many areas due to its ability to penetrate matter and deposit energy that causes chemical and biological changes. Some of the major areas using ionizing radiation are:

1. Medicine - Radiation is used for diagnostic imaging (X-rays, CT scans, nuclear medicine) and cancer treatment (radiation therapy). It allows visualization inside the body non-invasively.

2. Industry - Radiation is used to inspect welds, check for cracks/impurities in materials, and analyze composition of materials. This includes X-ray radiography and gamma scanning techniques.

3. Food Irradiation - Gamma irradiation and electron beams help preserve food, deactivate pathogens, and prevent sprouting and ripening. This improves shelf life and food safety.

4. Research - Radiation from X-ray machines, particle accelerators, and radioactive sources are used to analyze structure and properties of materials in physics, chemistry, biology disciplines.

5. Energy production - Nuclear fission chain reactions produce heat to make steam and generate electricity in nuclear power plants. 

6. Security and Defense - Radiation detectors and radioactive tracers help monitor unauthorized movement of nuclear materials and prevent smuggling. X-rays screen luggage for contraband.

        \end{solution}
    \end{parts}
    
\end{questions}

\begin{questions}
\question    \textbf{Question 4 (20 marks)}

\part 4.3. Draw a typical curve for the radioactive substance as given by the expression in 4.2. Briefly explain how you will determine the \(gamma\) [6 marks]
\begin{solution}
    (((Read Bragg Curve on Notes and BOOK)))
\end{solution}
\part 4.4. The half-life of strontium-90, (Sr), is 28.8 years. Calculate the (a) decay constant and (b) the initial activity of 1.00g of the material. [3 marks]
\begin{solution}
\textbf{Claude AI and ChatGPT}
Half-life (T1/2) of strontium-90 (Sr-90) = 28.8 years
Initial amount of Sr-90 (m) = 1.00 g

(a) Decay constant (λ) 
The decay constant is related to half-life as:
λ = ln(2)/T1/2

Plugging in the half-life value:
λ = ln(2)/28.8 years
λ = 0.024 yr−1

(b) Initial activity (A0)
The activity A0 is given by: 
A0 = λNm

Where N is the number of radioactive nuclei, calculated from:
N = (m/M)NA
Where m is the mass, M is the molar mass (90 g/mol for Sr-90), and NA is Avogadro's constant.

Calculating number of nuclei:
N = (1.00 g / 90 g/mol) x (6.022x1023 nuclei/mol) 
N = 6.69 x 1021 nuclei

Plugging this into the A0 equation:
A0 = λNm 
A0 = (0.024 yr−1) x (6.69 x 1021 nuclei) x (90 g/mol)
A0 = 1.48 x 1014 Bq (decays per second)

Therefore, the decay constant is 0.024 yr−1, and the initial activity is 1.48 × 1014 Bq.
\end{solution}
\end{questions}

\begin{questions}
    
    \question \textbf{QUESTION 5 (20 MARKS)}
    \part 5.1. Describe the term stopping power of a material.[2] \begin{solution}
         is the parameter used to describe the gradual loss of energy
of the charged particle, as it penetrates into an absorbing medium.
    \end{solution}
    \part 5.2. Explain the following quantities that are used for the purpose of quantifying radiation. (a) Activity A, (b) Kerma (K), (c) Equivalent Dose (H)[6]
    \begin{solution}
        (((notes)))
    \end{solution}
    
 \part \textbf{IN short read and understand everything about the Bragg Peak}
\end{questions}

\begin{questions}
    \question \textbf{QUESTION 1 (30 MARKS)}
\part 1.1. Explain the following terms:
(a)Photoelectrons, Thermions and Megavoltage electrons [3 marks]
(b)Characteristic (fluorescence) x-rays and Bremsstrahlung x-rays [2 marks]
(c) Remote afterloading technique and Neutron activation [2 marks]
\begin{solution}
   (a) Electrons released in medium by photoelectric effect are referred to as
photoelectrons. \newline Electrons released from metallic surface in thermionic emission are referred
to as thermions \newline Electrons produced by linear accelerators (linacs), betatrons or microtrons
for use in radiotherapy with kinetic energies typically in the range from
4 MeV to 30 MeV are referred to as megavoltage electrons. \newline (b) Gamma rays 
Photons resulting from nuclear transitions.
• Annihilation quanta 
Photons resulting from positron–electron annihilation.
• Characteristic (fluorescence) x rays (see Sect. 4.1):
Photons resulting from electron transitions between atomic shells.
• Bremsstrahlung x rays 
Photons resulting from electron–nucleus Coulomb interactions.
• Synchrotron radiation or magnetic bremsstrahlung 
Photons resulting from electrons moving in circular orbits in storage
rings.
\newline (c) The remote afterloading technique is a method used in brachytherapy, a form of radiotherapy where a radioactive source is placed directly into or near the tumor. Neutron activation is a process in which stable nuclei absorb neutrons and become radioactive. 
\end{solution}
\part 1.2. (a) Write a nuclear equation when fluorine-18 decays into oxygen-18.[3 marks]
\begin{solution}
    (((BOOK page 495)))
\end{solution}
 \part (b) Calculate the decay energy Q+¿¿ for the B+ decay of fluorine-18 into oxygen-18. (c) When Fluorodeoxyglucose (FDG) is labelled with radionuclide fluorine-18 it can be injected intravenously into a patient for use in positron emission tomography (PET) functional imaging. What are some of the uses of the FDG PET?[5 marks]
 \begin{solution}
    (((BOOK page 495)))
 \end{solution}
 \part The half-life of strontium-90, (Sr), is 28.8 years. Calculate the (a) decay constant and (b) the initial activity of 1.00g of the material. [5 marks]
 \begin{solution}
    
 \end{solution}
 
\end{questions}
\textbf{Most section 2 questions are repeated }
\begin{questions}
\question \textbf{QUESTION 3 (20 MARKS)}    
\begin{parts}
    \part 3.1. Write down the possible outcomes when an X-ray beam or gamma radiation passes through an object.[3]
    \begin{solution}
        \textbf{Claude AI} \newline When an X-ray or gamma radiation beam passes through matter, there are several possible outcomes, including:

1. Transmission - Some photons directly pass through the object without interaction. The amount transmitted depends on the size, density and composition of the material.

2. Absorption via Photoelectric effect - The photon transfers all its energy to eject an inner shell electron from an atom. The likelyhood increases with higher atomic number Z and lower photon energy.

3. Scattering 
- Coherent/Rayleigh scattering - Photon is deflected with no energy loss. Causes negligible attenuation.
- Incoherent/Compton scattering - Photon transfers portion of its energy to an outer shell electron and is deflected. Remaining portion of original photon energy is transmitted in a new direction.

4. Pair production - Very high energy photons (> 1.022 MeV) interact with nucleus and convert to an electron-positron pair. Significant attenuation occurs beyond a material's pair production threshold. 

5. Radiative capture - Capture of photon by nucleus, emitting a different energetic photon. Critical in neutron radiography.

6. Photodisintegration - Incident photon energy causes emission of particles from nucleus. Critical in deuterium-tritium fusion reactors. 

The transmitted beam has lower intensity and may be shifted in energy and direction compared to original beam that entered the object due to these various probabilistic interaction mechanisms.
    \end{solution}
   
    \part 3.3. An X-ray photon beam of photon energy 10 eV is allowed to fall on a metal surface. If the threshold frequency for the metal is 1.5 × 1015 Hz, then find the maximum kinetic energy gained by a photoelectron, the maximum speed of a photoelectron, and the stopping potential corresponding to the maximum kinetic energy.[8 marks]
    \begin{solution}
        \textbf{copied from chatGPT and perplexityAI} \newline To find the maximum kinetic energy (\(K_{\text{max}}\)), maximum speed of a photoelectron (\(v_{\text{max}}\)), and the stopping potential (\(V_{\text{stop}}\)), we can use the photoelectric effect equations.

1. **Maximum Kinetic Energy (\(K_{\text{max}}\)):**
   The maximum kinetic energy of a photoelectron is given by the energy of the incident photon minus the work function of the material.

   \[ K_{\text{max}} = h f - \phi \]

   where:
   - \( h \) is Planck's constant (\(6.626 \times 10^{-34} \ \text{J}\cdot\text{s}\)),
   - \( f \) is the frequency of the incident photon, and
   - \( \phi \) is the work function of the material.

   The frequency \( f \) can be calculated using the relation \( f = \frac{c}{\lambda} \), where \( c \) is the speed of light (\(3.00 \times 10^8 \ \text{m/s}\)) and \( \lambda \) is the wavelength.

2. **Maximum Speed of a Photoelectron (\(v_{\text{max}}\)):**
   The maximum speed of a photoelectron can be calculated using the kinetic energy equation \( K_{\text{max}} = \frac{1}{2} m v_{\text{max}}^2 \), where \( m \) is the mass of the electron (\(9.11 \times 10^{-31} \ \text{kg}\)).

   \[ v_{\text{max}} = \sqrt{\frac{2 K_{\text{max}}}{m}} \]

3. **Stopping Potential (\(V_{\text{stop}}\)):**
   The stopping potential is related to the maximum kinetic energy by \( eV_{\text{stop}} = K_{\text{max}} \), where \( e \) is the elementary charge (\(1.602 \times 10^{-19} \ \text{C}\)).

   \[ V_{\text{stop}} = \frac{K_{\text{max}}}{e} \]

Now, let's calculate these values.

### Step 1: Calculate the frequency \( f \):
\[ f = \frac{c}{\lambda} \]
\[ f = \frac{3.00 \times 10^8 \ \text{m/s}}{\lambda} \]

The wavelength \( \lambda \) can be calculated using the relationship between energy and wavelength for a photon: \( E = hf \).

\[ E = hf \]
\[ hf = \text{energy of the incident photon} = 10 \ \text{eV} \]

Now, convert energy from electron volts to joules using the conversion factor: \( 1 \ \text{eV} = 1.602 \times 10^{-19} \ \text{J} \).

\[ 10 \ \text{eV} = 10 \times 1.602 \times 10^{-19} \ \text{J} \]

Now, we have \( E \) and \( f \), and we can find \( \lambda \).

\[ E = hf \]
\[ \lambda = \frac{c}{f} \]

### Step 2: Calculate \( K_{\text{max}} \):
\[ K_{\text{max}} = hf - \phi \]

### Step 3: Calculate \( v_{\text{max}} \):
\[ v_{\text{max}} = \sqrt{\frac{2 K_{\text{max}}}{m}} \]

### Step 4: Calculate \( V_{\text{stop}} \):
\[ V_{\text{stop}} = \frac{K_{\text{max}}}{e} \]

I'll now perform these calculations.

Let's calculate the values:

### Step 1: Calculate the frequency \( f \):
\[ f = \frac{c}{\lambda} \]

The energy of the incident photon is given as 10 eV. Convert this energy to joules:

\[ E = 10 \ \text{eV} \times 1.602 \times 10^{-19} \ \text{J/eV} \]

Now, use the relationship \( E = hf \) to find \( f \):

\[ f = \frac{E}{h} \]

where \( h \) is Planck's constant (\(6.626 \times 10^{-34} \ \text{J}\cdot\text{s}\)).

\[ f = \frac{10 \times 1.602 \times 10^{-19}}{6.626 \times 10^{-34}} \ \text{Hz} \]

Now, let's calculate the other values.

### Step 2: Calculate \( K_{\text{max}} \):
\[ K_{\text{max}} = hf - \phi \]

where \( \phi \) is the work function of the material. Since it's not given, we'll assume it's negligible.

\[ K_{\text{max}} = hf \]

### Step 3: Calculate \( v_{\text{max}} \):
\[ v_{\text{max}} = \sqrt{\frac{2 K_{\text{max}}}{m}} \]

where \( m \) is the mass of the electron (\(9.11 \times 10^{-31} \ \text{kg}\)).

### Step 4: Calculate \( V_{\text{stop}} \):
\[ V_{\text{stop}} = \frac{K_{\text{max}}}{e} \]

where \( e \) is the elementary charge (\(1.602 \times 10^{-19} \ \text{C}\)).

Now, let's perform these calculations.

I apologize for any confusion. It seems I made an error in my response. Let's go through the calculations again.

### Step 1: Calculate the frequency \( f \):
\[ f = \frac{c}{\lambda} \]

The energy of the incident photon is given as 10 eV. Convert this energy to joules:

\[ E = 10 \ \text{eV} \times 1.602 \times 10^{-19} \ \text{J/eV} \]

Now, use the relationship \( E = hf \) to find \( f \):

\[ f = \frac{E}{h} \]

where \( h \) is Planck's constant (\(6.626 \times 10^{-34} \ \text{J}\cdot\text{s}\)).

\[ f = \frac{10 \times 1.602 \times 10^{-19}}{6.626 \times 10^{-34}} \ \text{Hz} \]

### Step 2: Calculate \( K_{\text{max}} \):
\[ K_{\text{max}} = hf \]

### Step 3: Calculate \( v_{\text{max}} \):
\[ v_{\text{max}} = \sqrt{\frac{2 K_{\text{max}}}{m}} \]

where \( m \) is the mass of the electron (\(9.11 \times 10^{-31} \ \text{kg}\)).

### Step 4: Calculate \( V_{\text{stop}} \):
\[ V_{\text{stop}} = \frac{K_{\text{max}}}{e} \]

where \( e \) is the elementary charge (\(1.602 \times 10^{-19} \ \text{C}\)).

Now, let's perform these corrected calculations.

### Step 1: Calculate the frequency \( f \):
\[ f = \frac{c}{\lambda} \]

The energy of the incident photon is given as 10 eV. Convert this energy to joules:
\[ E = 10 \ \text{eV} \times 1.602 \times 10^{-19} \ \text{J/eV} \]

Now, use the relationship \( E = hf \) to find \( f \):
\[ f = \frac{E}{h} \]
\[ f = \frac{10 \times 1.602 \times 10^{-19}}{6.626 \times 10^{-34}} \ \text{Hz} \]

\[ f \approx 2.409 \times 10^{15} \ \text{Hz} \]

### Step 2: Calculate \( K_{\text{max}} \):
\[ K_{\text{max}} = hf \]
\[ K_{\text{max}} = 6.626 \times 10^{-34} \times 2.409 \times 10^{15} \]

\[ K_{\text{max}} \approx 1.59 \times 10^{-18} \ \text{J} \]

### Step 3: Calculate \( v_{\text{max}} \):
\[ v_{\text{max}} = \sqrt{\frac{2 K_{\text{max}}}{m}} \]
\[ v_{\text{max}} = \sqrt{\frac{2 \times 1.59 \times 10^{-18}}{9.11 \times 10^{-31}}} \]

\[ v_{\text{max}} \approx 5.13 \times 10^6 \ \text{m/s} \]

### Step 4: Calculate \( V_{\text{stop}} \):
\[ V_{\text{stop}} = \frac{K_{\text{max}}}{e} \]
\[ V_{\text{stop}} = \frac{1.59 \times 10^{-18}}{1.602 \times 10^{-19}} \]

\[ V_{\text{stop}} \approx 9.94 \ \text{V} \]

So, the corrected values are:
- Maximum kinetic energy (\(K_{\text{max}}\)) ≈ \(1.59 \times 10^{-18} \ \text{J}\)
- Maximum speed of the photoelectron (\(v_{\text{max}}\)) ≈ \(5.13 \times 10^6 \ \text{m/s}\)
- Stopping potential (\(V_{\text{stop}}\)) ≈ \(9.94 \ \text{V}\)
    \end{solution}
\end{parts}

\end{questions}

\begin{questions}
    \question \textbf{QUESTION 4 (20 MARKS)}
\begin{parts}
 
    \part 4.2. Derive the functional relationship shown below between the thickness of an absorber and intensity of a photon beam attenuated by the absorber. Explain the terms.
\(I(x)\) [7]
\begin{solution}
    ((notes))
    \textbf{Perplexity AI} \newline The relationship between the thickness of an absorber and the intensity of a photon beam attenuated by the absorber can be derived using the Beer-Lambert Law, which describes the exponential decrease in intensity as a photon beam passes through a material. The law is often expressed as:

\[ I(x) = I_0 e^{-\mu x} \]

where:
- \( I(x) \) is the intensity of the photon beam after passing through a thickness \( x \) of the absorber.
- \( I_0 \) is the initial intensity of the photon beam (at \( x = 0 \), i.e., before passing through the absorber).
- \( \mu \) is the linear attenuation coefficient of the material, representing the probability of interaction (absorption or scattering) per unit length.

Now, let's derive this relationship.

### Derivation:

Consider a thin layer of thickness \( \Delta x \) within the absorber. The change in intensity (\( \Delta I \)) in passing through this layer is proportional to the intensity at that point (\( I \)) and the thickness of the layer (\( \Delta x \)), according to the attenuation coefficient \( \mu \):

\[ \Delta I = -\mu I \Delta x \]

The negative sign indicates a decrease in intensity as \( x \) increases. Now, we can express this relationship differentially:

\[ \frac{dI}{dx} = -\mu I \]

This is a first-order linear ordinary differential equation. To solve it, we can separate variables and integrate:

\[ \frac{1}{I} \, dI = -\mu \, dx \]

Integrating both sides:

\[ \int \frac{1}{I} \, dI = -\mu \int dx \]

\[ \ln|I| = -\mu x + C \]

where \( C \) is the constant of integration. Exponentiate both sides to solve for \( I \):

\[ I(x) = e^{-\mu x + C} \]

Using the property \( e^{A + B} = e^A \cdot e^B \):

\[ I(x) = e^{-\mu x} \cdot e^C \]

Let \( e^C = I_0 \), where \( I_0 \) is the initial intensity when \( x = 0 \):

\[ I(x) = I_0 e^{-\mu x} \]

This is the derived functional relationship between the thickness of an absorber (\( x \)) and the intensity of a photon beam (\( I(x) \)) attenuated by the absorber. The term \( e^{-\mu x} \) represents the exponential attenuation of the intensity as the thickness of the absorber increases, and \( I_0 \) is the initial intensity.
\end{solution}
\part 4.4. If a 0.2-cm thickness of material transmits 25% of a monoenergetic beam of photons, calculate the half-value layer (HVL) of the beam for that material.
[4]
\begin{solution}
    link \href{https://homework.study.com/explanation/if-a-2-mm-thick-of-material-transmits-40-of-a-monoenergetic-photon-beam-calculate-the-half-value-layer-of-the-beam.html}{link here} \newline \url{https://homework.study.com/explanation/if-a-2-mm-thick-of-material-transmits-40-of-a-monoenergetic-photon-beam-calculate-the-half-value-layer-of-the-beam.html}
\end{solution}
\end{parts}

\end{questions}

\begin{questions}
    \question \textbf{QUESTION 5 (20 MARKS)}
    \begin{parts}
      \textbf{Section 5 questions are also repeated}


    \end{parts}
\end{questions}

\end{document}

