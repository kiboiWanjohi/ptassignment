% LISTA DE EXERCÍCIOS Template using "book"
% Created by Milena Lima 
%Email:milenascimentolima@gmail.com
% Science Project at school 2017-FAPEAM
% View: https://www.overleaf.com/read/syjxcffdygch
%=======================================================
%------------LISTA DE EXERCÍCIOS
%=======================================================
\documentclass[12pt,a4paper,oneside,openany]{book} 
%=======================================================
%-------------PACOTES 
%=======================================================
\usepackage[top=1cm,left=1cm,right=1.5cm,bottom=2cm]{geometry}
\usepackage[T1]{fontenc}%Especif. a codif. de caracteres
\usepackage{ae}%Auxílio para fontes e pdf
\usepackage[utf8]{inputenc}
\usepackage{lipsum}%Gerar Texto Aleatório
\usepackage[brazil]{babel}%Traduzir para Português
\usepackage{indentfirst}%Faz indentações em parágrafos
\usepackage{graphicx}%Permite incluir figuras
\usepackage{subfig} %para criar sub figuras
\usepackage{float}% figuras
\usepackage{tabularx}
\usepackage{ragged2e}
\usepackage{multirow}
\usepackage[dvipsnames]{xcolor}%Admitir cores
\usepackage{amsmath,amssymb,amsthm}%Incluir expressões  matemáticas, equações, teoremas, símbolos, etc
\usepackage{lastpage}%Incluir Ficha catalográfica
\usepackage{epigraph}%Incluir Epígrafo
\usepackage{enumerate}
\usepackage{enumitem}
\newlist{questions}{enumerate}{3}
\setlist[questions]{label=\arabic*.}
\newcommand{\question}{\item}
\setlist[enumerate,1]{% (
leftmargin=*, itemsep=12pt, label={\textbf{\arabic*.)}}}
%---
\newlist{partes}{enumerate}{3}
\setlist[partes]{label=(\alph*)}
\newcommand{\parte}{\item}
%---
\newlist{subpartes}{enumerate}{3}
\setlist[subpartes]{label=\roman*)}
\newcommand{\subparte}{\item}
%---
\usepackage{array}
\usepackage{tikz}
\newcommand*\circled[1]{\tikz[baseline=(char.base)]{\node[shape=circle,draw,inner sep=2pt] (char) {#1};}}
\usepackage[sort&compress,round,comma,authoryear]{natbib}
\usepackage{makeidx}
\usepackage[colorlinks=true,urlcolor=magenta,citecolor=red,linkcolor=violet,bookmarks=true]{hyperref}
\usepackage{lscape}%Altera a orientação de uma página
\usepackage{pdflscape}
\usepackage{epstopdf} %converte figs eps em figs pdf
\usepackage{booktabs}
\usepackage{pdfpages}
\usepackage{textcomp}
\usepackage[many]{tcolorbox}
\usepackage{empheq}
\usepackage{tasks}%lista alfabética
\pagestyle{plain}
%================================================================
%------------DIGITE AQUI
%===============================================================
\newcommand{\orgao}{Governo do Estado do Amazonas}
\newcommand{\instituto}{Secretaria de Estado da Educação}
\newcommand{\departamento}{EE Luiz Vaz de Camões}
\newcommand{\curso}{Ciências da Computação}
\newcommand{\professor}{Milena N. Lima}
\newcommand{\disciplina}{Matemática}
\newcommand{\titulo}{1ª Lista de Exercícios: A. Combinatória}
\newcommand{\data}{\today}
\newcommand{\aluno}{ALUNO:}
\newcommand{\email}{{\bf mile}nascimentolima@gmail.com}
\newcommand{\turma}{kkkkkkkkkkkkkkkkkkkkkkkkkk}
%===========================================================
\newcommand{\X}{\mathbf{X}}
\newcommand{\x}{\mathbf{x}}
\newcommand{\Z}{\mathbf{Z}}
\newcommand{\z}{\mathbf{z}}
\newcommand{\y}{\boldsymbol{y}}
\newcommand{\balpha}{\mbox{${ \bm \alpha}$}}
\newcommand{\bmu}{\mbox{${\bm \mu}$}}
\newcommand{\bbeta}{\mbox{${\bm \beta}$}}
\newcommand{\bteta}{\mbox{${\bm \theta}$}}
\newcommand{\bgama}{\mbox{${\bm \gamma}$}}
\newcommand{\bxi}{\mbox{${\bm \xl}$}}
\newcommand{\bvarphi}{\mbox{${ \bm \varphi}$}}
\newcommand{\SZ}{\mbox{ $Z$}}
\newcommand{\muz}{\mu_{z,l}}
\newcommand{\muo}{\mu_{0,l}}
\newcommand{\etao}{\eta_{0,l}}
\newcommand{\etaz}{\eta_{z,l}}
\newcommand{\xbeta}{x_{l}\bgama}
\newcommand{\mui}{\mu_{l}}
\newcommand{\zetaind}{\zeta \mathtt{I}_{\{s_{l} \in Z \}}} 
\newcommand{\spz}{ s_{l} \in z}
\newcommand{\snpz}{ s_{l} \notin z}
\newcommand{\sps}{ s_{l} \in S}
\newcommand{\gphi}{ \Gamma(\phi)}
\newcommand{\scan}{ \Lambda_{z}}
\newcommand{\gmuop}{ \Gamma(\muo\phi)}
\newcommand{\gmuzp}{ \Gamma(\muz \phi)}
\newcommand{\gumuop}{ \Gamma((1-\muo)\phi)}
\newcommand{\gumuzp}{ \Gamma((1-\muz)\phi)}
\newcommand{\dlobeta}{ \frac{\parteial l_{0}(\bgama, \phi, 0) }{\parteial \bgama}}
\newcommand{\lz}{  l_{z}(\bgama, \phi, \tau)}
\newcommand{\lo}{  l_{0}(\beta, \phi, 0)}
\newcommand{\E}{\mathbb{E}}
\newcommand{\dis}{\displaystyle}
\linespread{1.0}%espaço entre linhas
\begin{document}
%%%%%%%%%%%%%%%%%%%%%%%%%%%%%%%%%%%%%%%%%%%%%%%%%%%%%%%%
%                      CABEÇALHO                     %
%%%%%%%%%%%%%%%%%%%%%%%%%%%%%%%%%%%%%%%%%%%%%%%%%%%%%%%%
\begin{table}[H]
\centering
\begin{tabular*}{\textwidth}{l@{\extracolsep{\fill}}l@{\extracolsep{\fill}}}
\begin{tabular}[l]{@{}l@{}}\textbf{\orgao}\\\textbf{\instituto}\\\textbf{\departamento} \end{tabular} & \begin{tabular}[l]{@{}l@{}}\textbf{Professor: \professor}\\ {\email}\\ \textbf{Disciplina: \disciplina}\end{tabular}                                                       
\end{tabular*}
\end{table}
\begin{center}
\rule[2ex]{\textwidth}{1pt}\\
{\Large{\titulo}}
\end{center}
\rule[2ex]{\textwidth}{1pt}\\
\begin{questions}[label=\protect\circled{\bfseries\arabic*}]
%%%%%%%%%%%%%%%%%%%%%%%%%%%%%%%%%%%%%%%%%%%%%%%%%%%%%%%%
%                      Questões                   %
%%%%%%%%%%%%%%%%%%%%%%%%%%%%%%%%%%%%%%%%%%%%%%%%%%%%%%%%

%=========================================================
\question
\citep{ Hazzan(1993)}
Question 1 30 Marks
\begin{partes}
\parte 
\begin{subpartes}
\subparte (a) State and explain briefly the importance of two parameters necessary for the detection of Radiation by a detector [4 Marks]

\subparte (b) Explain briefly what is meant by stopping power of a material [2 Marks]
\subparte (c) Give a brief discussion of the main processes through which heavy charged particles interact
with matter
[6 Marks]
\subparte (d) Using (b) and (c) above explain how you would select a material to give the best results for
detecting heavy charged particles
[4 Marks]
\subparte (e) A particle of mass M and charge q-Z.e, moving with a velocity v interacts with a material
which contains N atoms per unit volume. If the density of the material is p gcm-³;
i.Draw a sketch to represent the scenario described above, taking care to label all the
important features
[4Marks]
ii.Derive an expression for the stopping power of the material
[10 Marks]
\end{subpartes}
\end{partes}


\question
\citep{ Question 2 20 Marks} 
Question 2 20 marks
\begin{partes}
\parte
\begin{subpartes}
\subparte a) Explain how Electromagnetic Radiation is produced and categorized
[4 Marks]
\subparte b) Give a brief summary of the main modes of interaction of Electromagnetic Radiation with
[6 marks]
matter
\subparte c) A y-ray photon of energy E-hv is incident on a Hyper pure Germanium detector material.
Explain how the photon losses energy and hence derive an expression to quantify this
[10 marks]
energy loss.

\end{subpartes}
\end{partes}

\question
\citep{Question 3}
Question 3 20 marks
\begin{partes}
\parte
\begin{subpartes}
\subparte a)With the help of a neat diagram, describe and explain the main features of typical Voltage
Current characteristics of Gas filled detectors
[8 marks]
\subparte b) Using (a) above, compare and contrast the operation of a Geiger and proportional Counter
[6 marks]
A Radiologist and his Assistant are commissioning a new X-Ray machine to be used in
your Laboratory. They report to you that; after they switched on the machine, the Geiger in
the room stopped working. They switched off the machine, and after a few minutes the
counter started working. The machine was switched on and; the Geiger stopped yet again!
The Radiologist is convinced that it has something to do with the X-rays produced. On the
other hand the Assistant is convinced something is wrong with the detector. What do you
think? Do you agree with any or none of the two? If yes why? No, Why? As a Medical
Physicist, how do you explain this?[8 marks]
\end{subpartes}
\end{partes}

\question
\citep{Question 4}
Question 4 20 marks
\begin{partes}
\parte
\begin{partes}
\begin{subpartes}
    \subparte
    a) State the factors which influence the efficiency of a typical detector
    [5 Marks]
    \subparte b) Using (a) above, find an expression for the relative efficiency of the detector above
    [5 Marks]
    \subparte c) A detector with a circular window of radius 'a' is placed at a distance 'd' from a point
    source;
    i.Give a definition for the solid angle d2 subtended at the source by the detector
    [2 marks]
    ii. Using (i) above, find an expression for the solid angle subtended by the detector at the
    [5 Marks]
    source
    iii. Explain how the expression in (ii) above changes when the detector is very far from
    the source i.e., for a
    [3 Marks]
    \end{subpartes}
\end{partes}

\question
\citep{Question 5}
Question 5 20 marks
\begin{partes}
\parte

\begin{subpartes}
    \subparte a) With the help of a diagram, give a definition of Resolution of a detector and distinguish
    between "good" and "poor" resolution [4Marks]
    \subparte b) State under what conditions the detector response can be modelled by a Gaussian curve

    of the form: G(H)= exp
    A, Hoand H are related to the detector resolution
    [6 Marks]
    \subparte c) Assuming a linear relationship between H and N, find an expression for the limiting
    Resolution of a detector
    [4Marks]
    
    and explain what and how the parameters
    
    \subparte d) Explain what the Fano factor is, and discuss the conditions which give rise to this factor
    and hence explain how it modifies the expression for the limiting Resolution in (c) above
    [6 Marks].        
\end{subpartes}
\end{partes}

\question
\citep{Question 1 30 marks}
Question 1 30 marks
\begin{partes}
\parte
\begin{subpartes}
\subparte a) State and explain briefly how Radiation is categorised. (4 Marks)
\subparte b) Explain briefly what needs to happen in a detector for detection of Radiation to occur (2 Marks).

\subparte c) Nuclear medicine normally uses Electromagnetic Radiations. Give a brief discussion of the main processes through which Electromagnetic Radiation (EM) interacts with matter (6 Marks)
\subparte d) Using (b) and (c) above explain how you would select a material to give the best results for detecting EM radiation (4 Marks)
\subparte e) Gas filled detectors can be used to detect EM radiation. Sketch the Current Voltage characteristics of a gas filled detector and label all the important regions (5 Marks)
\subparte f) Most gas filled detectors have concentric cylindrical geometry, with an outer cylinder of radius "b" and an inner wire of radius "a". The outer cylinder is normally kept at ground potential.
i) With the help of a neat diagram, derive an expression for the voltage V between the inner and outer electrode (6 Marks)
ii)Using (i) above find an expression for the charge on the central electrode (2 Marks)

\end{subpartes}    
\end{partes}

\question
\citep{Question 2 20 marks}
Question 2 20 marks
\begin{partes}
\parte
\begin{subpartes}
\subparte Q2. A Geiger counter with the dimensions given in (f) above, is used to monitor y radiation emitted by a specimen of sun dried Tilapia fish. An interaction event in the Geiger occurs at a position a+8 from the central electrode:
\subparte a) Describe what happens (2 marks)
\subparte b) Assuming that the Geiger can be modelled by a parallel plate capacitor, write an equation to represent the energy density in the detector and explain all the terms in your equation ( 4 marks)
\subparte c) Using the work energy theorem, calculate the change in energy density due to electrons at the site of the interaction event (6 Marks)
\subparte d) Calculate the change in energy density due to the positive ions (4 marks)
\subparte e) Using the results from (c) and (d) above, find an expression for the electric field inside the Geiger (4 Marks)

\end{subpartes}    
\end{partes}


\question
\citep{Question 3 20 marks}
Question 3 20 marks
\begin{partes}
\parte
\begin{subpartes}
\subparte a) With the help of a neat diagrams, write short notes on the following:
i.Scintillation detectors (6 marks)
ii.Solid state detectors (6 marks)
\subparte b) Compare and highlight the advantages and disadvantages if any in the use of the two types of detectors (8 marks)

\end{subpartes}    
\end{partes}


\question
\citep{Question 4 20 marks}
Question 4 20 marks
\begin{partes}
\parte
\begin{subpartes}
\subparte a) Explain what you understand by Geometric factor, and how it influences the efficiency of a detector (5 Marks)
\subparte b) A detector with a rectangular window of width 'a' and breadth "b" is placed at a
distance 'd' from a point source S;

i. Divide the detector window into horizontal and vertical strips. Consider an element of area dxdy at a distance "y" from the horizontal. Let R be the distance from the centre
of the element of area to S. Draw the situation described and label all the important features in the sketch (5Marks)

ii. Write an expression and for the differential solid angle d2 (2 Marks)

iii. Using (i) and (ii) above, show that the solid angle subtended by the window at the
source is given by
(8 Marks)        
\end{subpartes}    
\end{partes}


\question
\citep{Question 5 20 marks}
Question 5 20 marks
\begin{partes}
\parte
\begin{subpartes}
\subparte a) Explain what is meant by the "Dead time" of a detector (4 Marks)\
\subparte b) State the main causes of detector dead time (6 Marks)
\subparte c) With the help of neatly labelled diagrams, compare and contrast the Paralysable and Non- Paralysable models of detector "dead time" (4Marks)
\subparte d) While on a routine Radiation survey in the Mrima region in Kwale, your field Radiometer headphones record a rapidly increasing series of clicks which reach a
i.
crescendo as you approach a peculiar looking shrub. Suddenly the meter falls silent:
Explain what has happened (3 Marks).
ii. Explain carefully what action you should take if any (3 Marks).
\end{subpartes}    
\end{partes}





\end{questions}
%===========================================================
%          BIBLIOGRAFIA
%===========================================================
\begin{thebibliography}{99}
\thispagestyle{empty}%myheadings
%===========================================================
%==============Livro
\bibitem[Hazzan(1993)]{Hazzan(1993)}
[1] {Hazzan, Samuel}.
\emph{Combinatória e Probabilidade}.
{Volume  5}, {1;ed}, {São Paulo}, {Atual}, {1993}.

\bibitem[ENEM(2012)]{ENEM(2012)}
[2] \textbf{ENEM 2012}
(Exame Nacional do Ensino Médio).
\emph{INEP-Instituto Nacional de Estudos e Pesquisas Educacionais Anísio Teixeira}.{Ministério da Educação}. 
Disponível em:
\url{http://www.enem.inep.gov.br/}.
%===========================================================
\end{thebibliography}
\end{document}