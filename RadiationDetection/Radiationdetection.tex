% LISTA DE EXERCÍCIOS Template using "book"
% Created by Milena Lima 
%Email:milenascimentolima@gmail.com
% Science Project at school 2017-FAPEAM
% View: https://www.overleaf.com/read/syjxcffdygch
%=======================================================
%------------LISTA DE EXERCÍCIOS
%=======================================================
\documentclass[12pt,a4paper,oneside,openany]{book} 
%=======================================================
%-------------PACOTES 
%=======================================================
\usepackage[top=1cm,left=1cm,right=1.5cm,bottom=2cm]{geometry}
\usepackage[T1]{fontenc}%Especif. a codif. de caracteres
\usepackage{ae}%Auxílio para fontes e pdf
\usepackage[utf8]{inputenc}
\usepackage{lipsum}%Gerar Texto Aleatório
\usepackage[brazil]{babel}%Traduzir para Português
\usepackage{indentfirst}%Faz indentações em parágrafos
\usepackage{graphicx}%Permite incluir figuras
\usepackage{subfig} %para criar sub figuras
\usepackage{float}% figuras
\usepackage{tabularx}
\usepackage{ragged2e}
\usepackage{multirow}
\usepackage[dvipsnames]{xcolor}%Admitir cores
\usepackage{amsmath,amssymb,amsthm}%Incluir expressões  matemáticas, equações, teoremas, símbolos, etc
\usepackage{lastpage}%Incluir Ficha catalográfica
\usepackage{epigraph}%Incluir Epígrafo
\usepackage{enumerate}
\usepackage{enumitem}
\newlist{questions}{enumerate}{3}
\setlist[questions]{label=\arabic*.}
\newcommand{\question}{\item}
\setlist[enumerate,1]{% (
leftmargin=*, itemsep=12pt, label={\textbf{\arabic*.)}}}
%---
\newlist{partes}{enumerate}{3}
\setlist[partes]{label=(\alph*)}
\newcommand{\parte}{\item}
%---
\newlist{subpartes}{enumerate}{3}
\setlist[subpartes]{label=\roman*)}
\newcommand{\subparte}{\item}
%---
\usepackage{array}
\usepackage{tikz}
\newcommand*\circled[1]{\tikz[baseline=(char.base)]{\node[shape=circle,draw,inner sep=2pt] (char) {#1};}}
\usepackage[sort&compress,round,comma,authoryear]{natbib}
\usepackage{makeidx}
\usepackage[colorlinks=true,urlcolor=magenta,citecolor=red,linkcolor=violet,bookmarks=true]{hyperref}
\usepackage{lscape}%Altera a orientação de uma página
\usepackage{pdflscape}
\usepackage{epstopdf} %converte figs eps em figs pdf
\usepackage{booktabs}
\usepackage{pdfpages}
\usepackage{textcomp}
\usepackage[many]{tcolorbox}
\usepackage{empheq}
\usepackage{tasks}%lista alfabética
\pagestyle{plain}
%================================================================
%------------DIGITE AQUI
%===============================================================
\newcommand{\orgao}{Governo do Estado do Amazonas}
\newcommand{\instituto}{Secretaria de Estado da Educação}
\newcommand{\departamento}{EE Luiz Vaz de Camões}
\newcommand{\curso}{Ciências da Computação}
\newcommand{\professor}{Milena N. Lima}
\newcommand{\disciplina}{Matemática}
\newcommand{\titulo}{Radiation Detection and Quantification}
\newcommand{\data}{\today}
\newcommand{\aluno}{ALUNO:}
\newcommand{\email}{{\bf mile}nascimentolima@gmail.com}
\newcommand{\turma}{kkkkkkkkkkkkkkkkkkkkkkkkkk}
%===========================================================
\newcommand{\X}{\mathbf{X}}
\newcommand{\x}{\mathbf{x}}
\newcommand{\Z}{\mathbf{Z}}
\newcommand{\z}{\mathbf{z}}
\newcommand{\y}{\boldsymbol{y}}
\newcommand{\balpha}{\mbox{${ \bm \alpha}$}}
\newcommand{\bmu}{\mbox{${\bm \mu}$}}
\newcommand{\bbeta}{\mbox{${\bm \beta}$}}
\newcommand{\bteta}{\mbox{${\bm \theta}$}}
\newcommand{\bgama}{\mbox{${\bm \gamma}$}}
\newcommand{\bxi}{\mbox{${\bm \xl}$}}
\newcommand{\bvarphi}{\mbox{${ \bm \varphi}$}}
\newcommand{\SZ}{\mbox{ $Z$}}
\newcommand{\muz}{\mu_{z,l}}
\newcommand{\muo}{\mu_{0,l}}
\newcommand{\etao}{\eta_{0,l}}
\newcommand{\etaz}{\eta_{z,l}}
\newcommand{\xbeta}{x_{l}\bgama}
\newcommand{\mui}{\mu_{l}}
\newcommand{\zetaind}{\zeta \mathtt{I}_{\{s_{l} \in Z \}}} 
\newcommand{\spz}{ s_{l} \in z}
\newcommand{\snpz}{ s_{l} \notin z}
\newcommand{\sps}{ s_{l} \in S}
\newcommand{\gphi}{ \Gamma(\phi)}
\newcommand{\scan}{ \Lambda_{z}}
\newcommand{\gmuop}{ \Gamma(\muo\phi)}
\newcommand{\gmuzp}{ \Gamma(\muz \phi)}
\newcommand{\gumuop}{ \Gamma((1-\muo)\phi)}
\newcommand{\gumuzp}{ \Gamma((1-\muz)\phi)}
\newcommand{\dlobeta}{ \frac{\parteial l_{0}(\bgama, \phi, 0) }{\parteial \bgama}}
\newcommand{\lz}{  l_{z}(\bgama, \phi, \tau)}
\newcommand{\lo}{  l_{0}(\beta, \phi, 0)}
\newcommand{\E}{\mathbb{E}}
\newcommand{\dis}{\displaystyle}
\linespread{1.0}%espaço entre linhas
\begin{document}
%%%%%%%%%%%%%%%%%%%%%%%%%%%%%%%%%%%%%%%%%%%%%%%%%%%%%%%%
%                      CABEÇALHO                     %
%%%%%%%%%%%%%%%%%%%%%%%%%%%%%%%%%%%%%%%%%%%%%%%%%%%%%%%%
\begin{table}[H]
\centering
\begin{tabular*}{\textwidth}{l@{\extracolsep{\fill}}l@{\extracolsep{\fill}}}
\begin{tabular}[l]{@{}l@{}}\textbf{\orgao}\\\textbf{\instituto}\\\textbf{\departamento} \end{tabular} & \begin{tabular}[l]{@{}l@{}}\textbf{Professor: \professor}\\ {\email}\\ \textbf{Disciplina: \disciplina}\end{tabular}                                                       
\end{tabular*}
\end{table}
\begin{center}
\rule[2ex]{\textwidth}{1pt}\\
{\Large{\titulo}}
\end{center}
\rule[2ex]{\textwidth}{1pt}\\
\begin{questions}[label=\protect\circled{\bfseries\arabic*}]
%%%%%%%%%%%%%%%%%%%%%%%%%%%%%%%%%%%%%%%%%%%%%%%%%%%%%%%%
%                      Questões                   %
%%%%%%%%%%%%%%%%%%%%%%%%%%%%%%%%%%%%%%%%%%%%%%%%%%%%%%%%

%=========================================================
\question
\citep{ Hazzan(1993)}
Question 1 30 Marks
\begin{partes}
\parte 
\begin{subpartes}
\subparte \textbf{(a) State and explain briefly the importance of two parameters necessary for the detection of Radiation by a detector [4 Marks]}
\newline Sensitivity refers to the ability of a radiation detector to respond to radiation and convert it into a measurable signal. High sensitivity is essential for detecting low levels of radiation accurately, 
A sensitive detector can register and amplify even weak signals produced by low levels of radiation. This is particularly important in situations where the radiation levels are low, and precision in measurement is crucial. 
Energy Resolution refers to the ability of a detector to distinguish between different energy levels of incident radiation. In applications like spectroscopy, where identification of specific radiation energies is essential, high energy resolution is crucial for accurate analysis and characterization of radioactive sources.Different radiation sources emit radiation at distinct energy levels. Energy resolution allows a detector to discriminate between these energy levels with precision. A detector with good energy resolution can identify and separate closely spaced energy peaks, providing detailed information about the radiation spectrum. 
\subparte \textbf{(b) Explain briefly what is meant by stopping power of a material [2 Marks]}
\newline 
The stopping power of a material refers to its ability to slow down and absorb the energy of charged particles, such as ions or electrons, as they traverse through the material.
\(S = -\frac{dE}{dx}\)

\subparte \textbf{(c) Give a brief discussion of the main processes through which heavy charged particles interact
with matter
[6 Marks]}
\newline - Ionization: is the process by which a charged particle, such as an alpha particle, strips electrons from atoms in the material, creating positively charged ions and free electrons.
Ionization is a dominant process for heavy charged particles at higher energies. It leads to the creation of ion pairs, causing the material to become electrically charged. The energy loss due to ionization increases with the charge and energy of the incident particle.
\newline - Excitation occurs when a charged particle transfers energy to an atomic electron, causing it to move to a higher energy level. The electron may return to its original state, releasing the excess energy as a photon.
While not as dominant as ionization, excitation contributes to the energy loss of heavy charged particles. The emitted photons may contribute to the production of visible light or other forms of electromagnetic radiation.
\newline - Radiative Energy Loss (Bremsstrahlung for heavy ions)
Heavy charged particles, particularly ions, may emit electromagnetic radiation (Bremsstrahlung) as they experience acceleration or deceleration in the electric field of the atomic nuclei.
At high energies, radiative energy loss becomes more relevant. It can contribute to the overall energy loss of heavy ions interacting with matter.
\newline - Coulomb Scattering refers to the deflection of charged particles due to the electromagnetic interaction with atomic electrons. The trajectory of the charged particle is altered by the electric field of the atomic nuclei and electrons.
Coulomb scattering is prominent for heavy charged particles at lower energies. It affects the trajectory of the incident particle, leading to a spread in the direction of movement.
\newline - Nuclear Reactions Heavy charged particles may undergo nuclear reactions, where they interact with the atomic nuclei of the material, leading to the production of new particles and alterations in the nuclear structure.
Nuclear reactions become increasingly important at higher energies. These reactions can result in the transmutation of elements and the release of additional particles.
\subparte \textbf{(d) Using (b) and (c) above explain how you would select a material to give the best results for
detecting heavy charged particles
[4 Marks]}
\newline - Stopping power is a key parameter that describes how quickly charged particles lose energy as they traverse a material. For heavy charged particles, stopping power is influenced by the material's density, atomic number (Z), and ionization potential.High Atomic Number (Z) Select a material with a high atomic number (Z). Heavy charged particles interact more strongly with materials containing high-Z elements. High-Z materials maximize the probability of ionization, leading to effective energy deposition and particle detection.
\newline Thickness of the Material Adjust the thickness of the material based on the energy of the heavy charged particles you aim to detect. Thicker materials increase the likelihood of interactions but also result in greater energy loss through multiple scattering events. Find a balance that optimizes stopping power without sacrificing too much particle energy.
\subparte \textbf{(e) A particle of mass M and charge q-Z.e, moving with a velocity v interacts with a material
which contains N atoms per unit volume. If the density of the material is \(��\) gcm-³;
i.Draw a sketch to represent the scenario described above, taking care to label all the
important features
[4Marks]
ii.Derive an expression for the stopping power of the material
[10 Marks]}
\newline
To derive an expression for the stopping power of a material, we need to consider the energy loss experienced by a charged particle as it interacts with the material. The stopping power is defined as the mean energy loss per unit path length, given by the equation: \(S = -\frac{dE}{dx}\) For a particle of mass (M) and charge \((q_Ze)\), moving with a velocity (v), the energy loss can be related to the initial kinetic energy \((E_i)\) and the final kinetic energy (E_f) after interacting with the material. The energy loss is given by the equation: \(E\index{loss}= E\index{i}- E\index{f}\)
Since the particle's velocity is constant, the final kinetic energy is given by:
\(E\index{f}= E\index{i}- qze.v.x\)
Combining these equations, we get the stopping power:
\(S(E) = \frac{-qze|M}.v.N.\)
\newline
Derivation of Stopping Power for a Charged Particle:
The stopping power (S) of a material is defined as the average rate of energy loss per unit path length of a charged particle traversing the material. It can be expressed as:
S = - dE/dx
where:
dE is the average energy lost by the particle over a path length dx.
Here, we consider a particle of mass M and charge q-Z.e (where Z is the atomic number and e is the elementary charge) moving with a velocity v in a material with N atoms per unit volume and density ρ.
Energy Loss Mechanisms:
There are two primary mechanisms for energy loss in this scenario:
Collisional interactions: The charged particle interacts with the electrons in the atoms of the material through Coulombic forces. These interactions cause the particle to lose energy by exciting or ionizing the atoms.
Radiative interactions: The charged particle emits electromagnetic radiation, such as bremsstrahlung, as it accelerates while interacting with the electric field of the atomic nuclei. This also contributes to energy loss.
Collisional Energy Loss:
The average energy loss per collision (ΔE) can be estimated using the Bethe-Bloch formula:
\(ΔE = 4πN Z^2 e^4/mv^2 * [ln(2mv^2/I) - ln(1 - β^2)]\)
where:

I is the average ionization potential of the atoms in the material.
β = v/c is the velocity of the particle relative to the speed of light.
The total energy loss due to collisions over a path length dx can be calculated as:
\(dE_c = N * ΔE * dx\)
The average energy loss per unit path length due to radiative interactions can be estimated using the following formula:
\(dE_r/dx = 4π N Z^2 e^4/mc^2 * [ln(183Z^(-1/3)) - f(β)]\)
Combining the energy loss from both mechanisms, we get the total energy loss per unit path length
\(dE/dx = dE_c/dx + dE_r/dx\)
Therefore, the stopping power (S) of the material is:
\(S = - dE/dx = - (N * ΔE + 4π N Z^2 e^4/mc^2 * [ln(183Z^(-1/3)) - f(β)])\)
We can express the stopping power in terms of the material's density (ρ) by noting that the number of atoms per unit volume (N) is related to the density as follows:
\(N = ρ * N_A / A\)
where:

N_A is Avogadro's constant.
A is the atomic mass of the material.
Substituting this expression for N in the stopping power equation, we get the final expression:
\(S = - (ρ * N_A / A * ΔE + 4π ρ N_A Z^2 e^4 / (Amc^2) * [ln(183Z^(-1/3)) - f(β)])\)
\newline
Starting with the Bethe-Bloch Formula:
The Bethe-Bloch formula for stopping power (
S) is given by:
\(S = \frac{4\pi e^4 Z^2 N}{Mv^2} \left[ \ln\left(\frac{2m_ev^2}{I}\right) - \frac{v^2}{c^2} \right]\)
where:


e is the elementary charge,

Z is the atomic number of the material,

N is the number of atoms per unit volume,

M is the mass of the incident particle,

v is its velocity,
\(m_e\) is the electron mass,

I is the mean excitation energy of the material,

c is the speed of light.
The charge of the incident particle is \(q = Z \cdot e - Z \cdot e = 0\) which means it's uncharged (
    Z is canceled out in the formula).
    Substitute and Simplify:
    Substitute 
q=0 into the Bethe-Bloch formula.
Simplify the expression by eliminating the terms with 
q.
The final expression for the stopping power (

S) with an uncharged particle is then:
\(S = \frac{4\pi e^4 Z^2 N}{Mv^2} \left[ \ln\left(\frac{2m_ev^2}{I}\right) - \frac{v^2}{c^2} \right]\)
\end{subpartes}
\end{partes}


\question
\citep{ Question 2 20 Marks} 
Question 2 20 marks
\begin{partes}
\parte
\begin{subpartes}
\subparte \textbf{a) Explain how Electromagnetic Radiation is produced and categorized
[4 Marks]}
\newline
One of the many ways in which different types of radiation are grouped 
together is in terms of ionizing and nonionizing radiation. The word ionizing 
refers to the ability of the radiation to ionize an atom or a molecule of the 
medium it traverses. 
Nonionizing radiation is electromagnetic radiation with wavelength A of 
about 10 nm or longer. That part of the electromagnetic spectrum includes 
radiowaves, microwaves, visible light (A = 770-390 nm), and ultraviolet light 
(A = 390-10 nm). 
Ionizing radiation includes the rest of the electromagnetic spectrum (X-rays, 
A = 0.01-10 nm) and y-rays with wavelength shorter than that of X-rays. It also
includes all the atomic and subatomic particles, such as electrons, positrons, 
protons, alphas, neutrons, heavy ions, and mesons. 
\subparte \textbf{b) Give a brief summary of the main modes of interaction of Electromagnetic Radiation with matter
[6 marks]
}
\newline Scattering:
Scattering involves the redirection of electromagnetic radiation in different directions without changing its energy. The scattering process can be elastic (no change in energy) or inelastic (energy change).
\newline Compton Scattering: Compton scattering occurs when X-rays or gamma rays interact with electrons, leading to a change in the wavelength of the radiation and the ejection of high-energy electrons.
\newline Ionization: When EMR has a higher energy, it can ionize atoms, causing them to lose electrons and become ions. This process is similar to the photoelectric effect and can occur in materials exposed to high-energy EMR, such as alpha particles, beta particles, and gamma rays
\newline Heating: EMR can heat up materials through various mechanisms, such as rotating molecules and increasing molecular vibrational energy. Infrared light, for example, is absorbed more strongly than microwaves but less strongly than visible light, leading to heating of the tissue
\subparte \textbf{ c) A gamma-ray photon of energy E=hv is incident on a Hyper pure Germanium detector material.
Explain how the photon losses energy and hence derive an expression to quantify this energy loss.
[10 marks]}
\newline 

When a gamma-ray photon interacts with a highly pure germanium (HPGe) detector material, it can lose energy through three primary mechanisms:

1. Photoelectric Effect:

This process is dominant at low photon energies (typically below 100 keV).
In this effect, the gamma-ray photon completely transfers its energy to an electron in the atomic shell of the germanium atom, ionizing the atom.
The ejected electron, known as a photoelectron, carries away a kinetic energy equal to the difference between the incident photon energy (E) and the binding energy of the electron in the atomic shell (Eb).
The remaining energy (Eb) is released as X-rays or Auger electrons.
2. Compton Scattering:

This process occurs at intermediate photon energies (100 keV to 10 MeV).
In this process, the gamma-ray photon interacts with an electron in the germanium atom, transferring a part of its energy to the electron.
The scattered photon changes direction and its energy is reduced compared to the incident photon.
The ejected electron, known as a Compton electron, carries away the remaining energy.
3. Pair Production:

This process occurs at high photon energies (above 1.022 MeV).
In this process, the gamma-ray photon interacts with the electric field of the nucleus of a germanium atom and is converted into an electron-positron pair.
The energy of the incident photon is used to create the electron-positron pair and provide them with kinetic energy.
The positron eventually annihilates with an electron, releasing two gamma-ray photons with energy of 511 keV each.
Energy Loss Quantification:

The energy loss of the gamma-ray photon in each process can be quantified using the following equations:

1. Photoelectric Effect:
\(ΔE = E - Eb\) ΔE: energy loss of the gamma-ray photon
E: incident photon energy
Eb: binding energy of the electron
2. Compton Scattering:
\(ΔE = E - E'\) E': scattered photon energy
E: incident photon energy
The relationship between E and E' is given by the Compton scattering formula:
\(ΔE = E'/(1 + (E'/mc^2)(1 - cos(θ)))\) m: electron mass
c: speed of light
θ: scattering angle
The energy of the photon after scattering \((E')\)is related to its initial energy E by the equation:
\(E' = \frac{E}{1 + \frac{E}{m_e c^2} \cdot (1 - \cos θ)}\)
Starting with the energy-momentum relation for a photon:
\(E^2 = p^2 c^2\)
For Compton scattering, we consider the conservation of energy and momentum. The energy conservation equation is:
\(E + m_e c^2 = E' + \sqrt{p'^2 c^2 + m_e^2 c^4}\)
Given that \(p = \frac{E}{c}\) and \(p' = \frac{E'}{c}\) and using the relation \(E' = \frac{E}{1 + \frac{E}{m_e c^2} \cdot (1 - \cos θ)}\) one can solve for \(E'\)
3. Pair production 
\(ΔE = E - 2mc^2\)
ΔE: energy loss of the gamma-ray photon
E: incident photon energy
mc^2: energy equivalent of the electron mass

\end{subpartes}
\end{partes}

\question
\citep{Question 3}
Question 3 20 marks
\begin{partes}
\parte
\begin{subpartes}
\subparte \textbf{a)With the help of a neat diagram, describe and explain the main features of typical Voltage
Current characteristics of Gas filled detectors
[8 marks]}
\subparte \textbf{ b) Using (a) above, compare and contrast the operation of a Geiger and proportional Counter
[6 marks]}
\newline
\subparte \textbf{A Radiologist and his Assistant are commissioning a new X-Ray machine to be used in
your Laboratory. They report to you that; after they switched on the machine, the Geiger in
the room stopped working. They switched off the machine, and after a few minutes the
counter started working. The machine was switched on and; the Geiger stopped yet again!
The Radiologist is convinced that it has something to do with the X-rays produced. On the
other hand the Assistant is convinced something is wrong with the detector. What do you
think? Do you agree with any or none of the two? If yes why? No, Why? As a Medical
Physicist, how do you explain this?[8 marks]}
\end{subpartes}
\end{partes}

\question
\citep{Question 4}
Question 4 20 marks
\begin{partes}
\parte
\begin{partes}
\begin{subpartes}
    \subparte
    \textbf{a) State the factors which influence the efficiency of a typical detector
    [5 Marks]}
    \newline The efficiency of a radiation detector is influenced by several factors, including intrinsic efficiency, geometric efficiency, and detector dead time.
Intrinsic Efficiency: This is influenced by the properties of the detector, such as density, charge collection efficiency, and detector attenuation, as well as the incident radiation's type and energy

Geometric Efficiency: It is affected by the area of the detector entrance, the source-to-detector distance, and the isotropic nature of radiation emissions. The geometric efficiency can be calculated based on these factors

Detector Dead Time: This is the time during which the detector is non-responsive to a subsequent event, and it affects the efficiency of the detector in recording all radiation events

Type and Energy of Incident Radiation: The efficiency of a detector depends on the type and energy of the incident radiation. For example, for incident charged particles such as alpha or beta particles, many detectors have a total efficiency close to 100 percent, while for incident gamma rays, the situation is different

Detector Design and Material: The design and material of the detector also play a significant role in its efficiency

The efficiency of a typical radiation detector depends on several factors, which can be categorized into three main groups:

1. Interaction of radiation with the detector material:

Type of radiation: Different types of radiation interact with matter differently. For example, gamma rays are more penetrating than alpha particles, requiring thicker detector material for efficient detection.
Energy of the radiation: Higher energy radiation is generally easier to detect than lower energy radiation. This is because higher energy radiation deposits more energy in the detector material, making it more likely to produce a detectable signal.
Detector material: The atomic number (Z) of the detector material plays a significant role in its efficiency. Higher Z materials are more likely to interact with radiation, resulting in a higher detection efficiency.
Density of the detector material: Denser materials have a higher chance of interacting with radiation per unit volume, leading to increased efficiency.
Size and shape of the detector: Larger detectors have a greater surface area, increasing the probability of interacting with radiation. However, larger detectors can also be less efficient due to increased scattering and self-absorption of the radiation.
2. Detector design and electronics:

Active volume: The active volume of the detector is the portion of the material that is sensitive to radiation. A larger active volume generally leads to higher efficiency.
Window material: The window material allows the radiation to enter the detector. Ideally, the window material should be thin and have minimal attenuation of the radiation of interest.
Electronics: The electronics associated with the detector amplify and process the signal produced by the interaction of radiation with the detector material. High-quality electronics can improve the signal-to-noise ratio and improve the detection efficiency.
Pulse shaping: Pulse shaping techniques can be used to improve the signal-to-noise ratio and distinguish between different types of radiation.
3. Environmental factors:

Temperature: Temperature can affect the efficiency of some detector materials. For example, scintillation detectors often have a lower efficiency at higher temperatures.
Magnetic field: Magnetic fields can interfere with the operation of some detectors, such as photomultiplier tubes, and reduce their efficiency.
Background radiation: Background radiation can create a noisy environment for the detector, making it more difficult to distinguish between the signal of interest and background noise.

The efficiency of a detector refers to its ability to accurately and effectively detect and respond to incoming radiation. Several factors influence the efficiency of a typical detector. Here are some key factors:

Detector Material:

The choice of material used in the detector is critical. Different materials have varying probabilities of interacting with incoming radiation. The material's density, atomic number, and composition influence its efficiency.
Detector Size and Geometry:

The physical size and geometry of the detector impact its efficiency. Larger detectors generally have higher efficiency as they can capture more radiation. Additionally, the design, such as the thickness and shape of the detector, can influence how efficiently it interacts with radiation.
Energy Range:

The efficiency of a detector is often energy-dependent. Detectors may be more efficient for certain energy ranges, depending on their design and the characteristics of the detector material. Matching the detector to the energy range of interest is crucial for optimal performance.
Detector Response Time:

The time it takes for a detector to respond to incoming radiation can affect its efficiency, especially in dynamic or fast-changing radiation fields. Fast response times are crucial for accurate measurements, particularly in applications where timing is critical.
Detector Sensitivity and Resolution:

Sensitivity refers to the detector's ability to produce a signal in response to incoming radiation. Higher sensitivity often correlates with higher efficiency. Additionally, the detector's resolution, or its ability to distinguish between different energy levels, contributes to its overall efficiency.
Efficiency Calibration:

Proper calibration of the detector is essential. Calibration ensures that the detector response is accurately correlated with the actual radiation dose or intensity. Calibrating the detector for specific energy ranges and types of radiation enhances its efficiency.
Background Radiation:

The presence of background radiation in the environment can influence detector efficiency. Shielding or background subtraction techniques may be employed to mitigate the impact of ambient radiation on measurements.
Purity of Detector Material:

The purity of the detector material is critical, especially in semiconductor detectors. Impurities in the material can introduce unwanted noise and affect the detector's performance.
Environmental Conditions:

The conditions in which the detector operates, such as temperature and pressure, can affect its efficiency. Some detectors may have specific operational ranges, and deviations from these conditions might impact their performance.
Detector Electronics:

The electronics associated with the detector, including amplifiers and signal processing components, play a role in overall efficiency. Efficient electronics contribute to accurate signal detection and measurement.
Efficiency at Different Angles:

The efficiency of some detectors can vary with the angle of incidence of radiation. The angular dependence of efficiency is a consideration in applications where radiation arrives from different directions.

    \subparte \textbf{ b) Using (a) above, find an expression for the relative efficiency of the detector above
    [5 Marks]}
    \newline The relative efficiency (RE) of a radiation detector is a measure of its effectiveness in detecting radiation compared to a reference detector. It is expressed as the ratio of the count rates or measured intensities of the detector in question to that of a standard or reference detector under the same conditions. The relative efficiency is often denoted by the symbol RE, and its expression is given by:
    \( RE = \frac{N}{M} \cdot \frac{A_M}{A_N}\)
    Where:
N is the count rate or measured intensity of the detector in question.
M is the count rate or measured intensity of the reference detector.
 \(A_M\) is the detector efficiency (absolute efficiency) of the detector in question.
\(A_N\)  is the detector efficiency (absolute efficiency) of the reference detector.
The absolute efficiency (A) is the ratio of the number of events detected by the detector to the number of incident particles. It is often expressed as a percentage.
\(A = \frac{\text{Number of Events Detected}}{\text{Number of Incident Particles}} \times 100\)
The relative efficiency of a radiation detector is the ratio of the number of counts detected by the detector to the number of incident particles or photons. It is a dimensionless quantity, typically expressed as a percentage.

Here is the general expression for the relative efficiency of a radiation detector:
\(Relative Efficiency = (Number of counts detected) / (Number of incident particles or photons) * 100\)
    \subparte \textbf{ c) A detector with a circular window of radius 'a' is placed at a distance 'd' from a point
    source; \newline
    i.Give a definition for the solid angle dΩ subtended at the source by the detector
    [2 marks] \newline 
    ii. Using (i) above, find an expression for the solid angle subtended by the detector at the source
    [5 Marks] \newline
    iii. Explain how the expression in (ii) above changes when the detector is very far from
    the source
    [3 Marks]} \newline 
    \end{subpartes}
\end{partes}

\question
\citep{Question 5}
Question 5 20 marks
\begin{partes}
\parte

\begin{subpartes}
    \subparte \textbf{a) With the help of a diagram, give a definition of Resolution of a detector and distinguish
    between "good" and "poor" resolution [4Marks]}
    \subparte \textbf{b) State under what conditions the detector response can be modelled by a Gaussian curve
    of the form: \(G(H) = \frac{A|σ\sqrt{2��}}exp-[\frac{(H-H\index{0})\power{2}|2��\power{2}}]\) and explain what and how the parameters A, \(H_o\)  and H are related to the detector resolution
    [6 Marks]}
    \newline A Gaussian curve is often used to model the response of a radiation detector when the energy distribution of detected events follows a normal or Gaussian distribution. Several conditions influence the applicability of a Gaussian model for the response of a radiation detector:

Monoenergetic Radiation Source:

The radiation source emitting particles or photons should be monoenergetic, meaning that all emitted particles have the same energy. In practical terms, this is rarely the case, but the emitted energies should be sufficiently close to each other.
Energy Resolution:

The detector should have good energy resolution. Energy resolution refers to the ability of the detector to distinguish between different energy levels of incident radiation. A Gaussian response is more appropriate when the energy resolution is high.
Parameters in the Gaussian Model:
Where:
A is the amplitude or normalization factor of the Gaussian peak.
\(H_0\)  is the peak position, representing the most probable energy value.
\(σ\) is the standard deviation, which characterizes the width of the Gaussian curve.
The amplitude A represents the maximum height of the Gaussian peak. In the context of a radiation detector, 
A is related to the efficiency and the number of counts recorded. A higher amplitude corresponds to a higher efficiency or a larger number of detected events.
The peak position 
\(H_0\)  corresponds to the most probable energy value of the detected events. In terms of detector resolution, 
\(H_0\)  aligns with the true energy of the incident radiation. If\(H_0\) is accurately determined and centered on the true energy value, it reflects good energy calibration and resolution.
The standard deviation 
\(σ\) characterizes the width of the Gaussian curve. In the context of a radiation detector, 
\(σ\) is related to the energy resolution of the detector. A smaller 
\(σ\) indicates better energy resolution, as the spread of detected energies around the most probable value is narrower. \newline 
    \subparte \textbf{c) Assuming a linear relationship between H and N, find an expression for the limiting
    Resolution of a detector   
    [4Marks] 
    }
    \newline The limiting resolution of a detector, often expressed in terms of Full Width at Half Maximum (FWHM), can be derived assuming a linear relationship between the measured energy 
H and the number of counts 
N. Let's denote the slope of this linear relationship as 
m and the intercept as 
b. The equation of the line is then \(N = mH + b\) The FWHM is a measure of the width of the response curve at half of its maximum height. It is related to the standard deviation \((σ)\)  in the case of a Gaussian distribution by the equation: \(\text{FWHM} = 2\sqrt{2\ln 2} \cdot \sigma\) For a linear relationship, the standard deviation (

σ) is related to the slope (

m) and the limiting resolution \(\Delta H\) by \(\sigma = \frac{\Delta H}{2\sqrt{2\ln 2}}\) Substitute this expression for 

σ into the equation for FWHM: \(\text{FWHM} = 2\sqrt{2\ln 2} \cdot \frac{\Delta H}{2\sqrt{2\ln 2}}\) Simplify: \(\text{FWHM} = \Delta H\) So, assuming a linear relationship between 

H and N, the limiting resolution (
Δ
ΔH) of the detector is equal to the Full Width at Half Maximum (FWHM) of the energy distribution.
\newline 
Assuming a linear relationship between the energy H and the number of counts N, the limiting resolution of a detector can be expressed as: \(⧍H = \frac{2.35 × FWHM|m}\) where FWHM is the full width at half maximum of the peak, and m is the slope of the linear relationship between H and N
The full width at half maximum (FWHM) is a measure of the detector's energy resolution, and it is related to the standard deviation σ of the Gaussian curve by FWHM = 2.35σ. The slope m of the linear relationship between H and N is related to the detector's gain and calibration

In summary, the limiting resolution of a detector can be expressed in terms of the FWHM and the slope of the linear relationship between energy and counts. The FWHM is related to the detector's energy resolution, while the slope is related to the detector's gain and calibration.
    \subparte \textbf{d) Explain what the Fano factor is, and discuss the conditions which give rise to this factor
    and hence explain how it modifies the expression for the limiting Resolution in (c) above
    [6 Marks].}        
    \newline The Fano factor is a parameter used to describe the statistical fluctuations in the number of charge carriers produced by ionizing radiation in a detector. It is denoted by 
F and is defined as the ratio of the variance \(\sigma^2\) of the number of electron-hole pairs produced by the incident radiation to the mean number of such pairs \(\bar{N}\): \newline \(F = \frac{\sigma^2}{\bar{N}}\) The Fano factor is associated with the concept of statistical fluctuations in the ionization process, and it can significantly impact the energy resolution of a radiation detector.

Conditions Leading to the Fano Factor:
The Fano factor arises due to the stochastic nature of ionization processes. Key conditions and factors contributing to the presence of the Fano factor include:

Statistical Nature of Ionization:

The ionization process involves the creation of electron-hole pairs due to the interaction of radiation with the detector material. The creation of these charge carriers is a fundamentally probabilistic process, leading to statistical fluctuations.
Multiplicity of Ionization:

In some detectors, a single incident particle can create multiple electron-hole pairs. The Fano factor is particularly relevant in cases where the multiplicity of ionization events can vary.
Charge Carrier Generation:

The Fano factor is influenced by the energy required to create an electron-hole pair. If the energy required is relatively constant, fluctuations in the number of generated pairs can be attributed to the statistical nature of the ionization process.
Modification of the Expression for Limiting Resolution:
The presence of the Fano factor modifies the expression for the limiting resolution \(\Delta H\)  can be expressed as the product of the standard deviation \((\sigma)\) and a factor related to the Fano factor (F): \(\Delta H = \sigma \cdot \sqrt{F}\)
\newline The Fano factor is a measure of the statistical fluctuations in the number of charge carriers produced by a radiation detector. It is introduced as an adjustment factor to relate the observed variance to the Poisson theory predicted variance. The Fano factor is defined as the ratio of the variance to the mean number of charge carriers, and it is material-specific. The Fano factor arises due to the statistical nature of the energy deposition process in the detector material. The energy deposited by a radiation particle is converted into a number of charge carriers, and the number of charge carriers produced is subject to statistical fluctuations. The Fano factor is affected by several factors, including the detector material, the energy of the incident radiation, and the temperature of the detector
The Fano factor modifies the expression for the limiting resolution by affecting the standard deviation σ of the Gaussian curve. The standard deviation is related to the FWHM by \(FWHM =2.35σ\), and the Fano factor affects the standard deviation by introducing additional statistical fluctuations. As a result, the limiting resolution expression becomes: \(⧍H = \frac{2.35 × FWHM|\sqrt{m+ F}}\) where F is the Fano factor, and m is the slope of the linear relationship between energy and counts. The Fano factor reduces the energy resolution of the detector, and for good energy resolution, it should be as small as possible
\end{subpartes}
\end{partes}

\question
\citep{Question 1 30 marks}
Question 1 30 marks
\begin{partes}
\parte
\begin{subpartes}
\subparte \textbf{a) State and explain briefly how Radiation is categorised. (4 Marks)}
\subparte \textbf{b) Explain briefly what needs to happen in a detector for detection of Radiation to occur (2 Marks).}

\subparte \textbf{c) Nuclear medicine normally uses Electromagnetic Radiations. Give a brief discussion of the main processes through which Electromagnetic Radiation (EM) interacts with matter (6 Marks)}
\subparte \textbf{d) Using (b) and (c) above explain how you would select a material to give the best results for detecting EM radiation (4 Marks)}
\subparte \textbf{e) Gas filled detectors can be used to detect EM radiation. Sketch the Current Voltage characteristics of a gas filled detector and label all the important regions (5 Marks)}
\subparte \textbf{f) Most gas filled detectors have concentric cylindrical geometry, with an outer cylinder of radius "b" and an inner wire of radius "a". The outer cylinder is normally kept at ground potential.
i) With the help of a neat diagram, derive an expression for the voltage V between the inner and outer electrode (6 Marks)
ii)Using (i) above find an expression for the charge on the central electrode (2 Marks)
}
\end{subpartes}    
\end{partes}

\question
\citep{Question 2 20 marks}
Question 2 20 marks
\begin{partes}
\parte
\begin{subpartes}
\subparte \textbf{Q2. A Geiger counter with the dimensions given in (f) above, is used to monitor y radiation emitted by a specimen of sun dried Tilapia fish. An interaction event in the Geiger occurs at a position a+8 from the central electrode:
}
\subparte \textbf{a) Describe what happens (2 marks)}
\subparte \textbf{b) Assuming that the Geiger can be modelled by a parallel plate capacitor, write an equation to represent the energy density in the detector and explain all the terms in your equation ( 4 marks)}
\subparte \textbf{c) Using the work energy theorem, calculate the change in energy density due to electrons at the site of the interaction event (6 Marks)}
\subparte \textbf{d) Calculate the change in energy density due to the positive ions (4 marks)}
\subparte \textbf{e) Using the results from (c) and (d) above, find an expression for the electric field inside the Geiger (4 Marks)}

\end{subpartes}    
\end{partes}


\question
\citep{Question 3 20 marks}
Question 3 20 marks
\begin{partes}
\parte
\begin{subpartes}
\subparte \textbf{a) With the help of a neat diagrams, write short notes on the following:
i.Scintillation detectors (6 marks)
ii.Solid state detectors (6 marks)}
\newline Scintillators are materials-solids, liquids, gases-that produce sparks or scintillations of light when ionizing radiation passes through them. The amount of light produced in the scintillator is very small. It must be 
amplified before it can be recorded as a pulse or in any other way. The 
amplification or multiplication of the scintillator's light is achieved with a device 
known as the photomultiplier tube (or phototube). Its name denotes its function: 
it accepts a small amount of light, amplifies it many times, and delivers a strong 
pulse at its output. The operation of a scintillation counter may be divided into two broad steps: 
1. Absorption of incident radiation energy by the scintillator and production of 
photons in the visible part of the electromagnetic spectrum 
2. Amplification of the light by the photomultiplier tube and production of the 
output pulse 
The sections that follow analyze these two steps in detail. The different types of 
scintillators are divided, for the present discussion, into three groups: 
1. Inorganic scintillators (Most of the inorganic scintillators are crystals of the alkali metals, in particular 
alkali iodides, that contain a small concentration of an impurity)
2. Organic scintillators (The materials that are efficient organic scintillators belong to the class of 
aromatic compounds. They consist of planar molecules made up of benzenoid 
rings. Two examples are toluene and anthracene,)
3. Gaseous scintillators (Gaseous scintillators are mixtures of noble ga~es.'~.~~ The scintillations are 
produced as a result of atomic transitions. Since the light emitted by noble gases 
belongs to the ultraviolet region, other gases, such as nitrogen, are added to the 
main gas to act as wavelength shifters. Thin layers of fluorescent materials used 
for coating the inner walls of the gas container achieve the same effect. )
\newline Solid state detectors 
\subparte textbf{b) Compare and highlight the advantages and disadvantages if any in the use of the two types of detectors (8 marks)}

\end{subpartes}    
\end{partes}


\question
\citep{Question 4 20 marks}
Question 4 20 marks
\begin{partes}
\parte
\begin{subpartes}
\subparte 
\textbf{a) Explain what you understand by Geometric factor, and how it influences the efficiency of a detector (5 Marks)} 
\subparte \textbf{b) A detector with a rectangular window of width 'a' and breadth "b" is placed at a
distance 'd' from a point source S;

i. Divide the detector window into horizontal and vertical strips. Consider an element of area dxdy at a distance "y" from the horizontal. Let R be the distance from the centre
of the element of area to S. Draw the situation described and label all the important features in the sketch (5Marks)

ii. Write an expression and for the differential solid angle d2 (2 Marks)

iii. Using (i) and (ii) above, show that the solid angle subtended by the window at the
source is given by
(8 Marks)}        
\end{subpartes}    
\end{partes}


\question
\citep{Question 5 20 marks}
Question 5 20 marks
\begin{partes}
\parte
\begin{subpartes}
\subparte \textbf{
a) Explain what is meant by the "Dead time" of a detector (4 Marks)\
}
\newline The minimum amount of time that must separate 
two events in order that they be recorded as two separate pulses. The dead time or resolving time is the minimum time that can elapse after the 
arrival of two successive particles and still result in two separate pulses. Dead time, or resolving time, of a counting system is defined as the minimum 
time that can elapse between the arrival of two successive particles at the 
detector and the recording of two distinct pulses. 
\subparte b) State the main causes of detector dead time (6 Marks)
\newline  Because of the random nature of radioactive decay, there is always 
some probability that a true event will be lost because it occurs too quickly following a preceding event. 
\subparte c) With the help of neatly labelled diagrams, compare and contrast the Paralysable and Non- Paralysable models of detector "dead time" (4Marks)
\subparte d) While on a routine Radiation survey in the Mrima region in Kwale, your field Radiometer headphones record a rapidly increasing series of clicks which reach a
i.
crescendo as you approach a peculiar looking shrub. Suddenly the meter falls silent:
Explain what has happened (3 Marks).
ii. Explain carefully what action you should take if any (3 Marks).
\end{subpartes}    
\end{partes}





\end{questions}
%===========================================================
%          BIBLIOGRAFIA
%===========================================================
\begin{thebibliography}{99}
\thispagestyle{empty}%myheadings
%===========================================================
%==============Livro
\bibitem[Hazzan(1993)]{Hazzan(1993)}
[1] {Hazzan, Samuel}.
\emph{Combinatória e Probabilidade}.
{Volume  5}, {1;ed}, {São Paulo}, {Atual}, {1993}.

\bibitem[ENEM(2012)]{ENEM(2012)}
[2] \textbf{ENEM 2012}
(Exame Nacional do Ensino Médio).
\emph{INEP-Instituto Nacional de Estudos e Pesquisas Educacionais Anísio Teixeira}.{Ministério da Educação}. 
Disponível em:
\url{http://www.enem.inep.gov.br/}.
%===========================================================
\end{thebibliography}
\end{document}

