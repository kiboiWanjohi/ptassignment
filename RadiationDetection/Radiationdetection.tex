%description: Math 290 HW Template

%%%%% Beginning of preamble %%%%%

\documentclass[12pt]{article}  %What kind of document (article) and what size

%Packages to load which give you useful commands
\usepackage{graphicx}
\usepackage{amssymb, amsmath, amsthm}

%Sets the margins

\textwidth = 7 in
\textheight = 9.5 in
\oddsidemargin = -0.3 in
\evensidemargin = -0.3 in
\topmargin = -0.4 in
\headheight = 0.0 in
\headsep = 0.0 in
\parskip = 0.2in
\parindent = 0.0in

%defines a few theorem-type environments
\newtheorem{theorem}{Theorem}
\newtheorem{corollary}[theorem]{Corollary}
\newtheorem{definition}{Definition}

%%%%% End of preamble %%%%%

\begin{document}

%Identification, Change as necessary!

{\Large Jibri Kea} \hfill
{\large Math 206, Section 2,}  %Delete one
\hfill  \today

I worked with:


\begin{enumerate}
\item  \textbf{Problem 6: Number Systems} Perform D3F02 (hex) + C34D (hex). State your answer in hexadecimal and decimal.


D3F02h + C34Dh

\[ D		 3		 F		 0		 2 		h\]

\[1101 +	0011 +	1111 +	0000 +	0010	b\]
\[ 13	+	 3 	 +	 15   +	 0	 +	 2		d\]
\[13*16^{4} + 3*16^{3}  +  15*16^{2}  + 0*16^{1}  + 2*16^{0}\]
\[851,968 + 12,288 + 3,840 + 0 + 2d\]
\[868,098d\]


\[E024Fh = 918,095d\]

\item \textbf{a)
State and explain briefly the importance of two parameters necessary for the detection of
Radiation by a detector}
QUESTION 1 (30 MARKS)
a)
State and explain briefly the importance of two parameters necessary for the detection of
Radiation by a detector
[4 Marks]
[2 Marks]
b) Explain briefly what is meant by the "stopping power" of a material
c) Give a brief discussion of the main processes through which heavy charged particles interact
with matter
[6 Marks]
d) Using (b) and (c) above explain how you would select a material to give the best results for
detecting heavy charged particles
[4 Marks]
e) A particle of mass M and charge q-Z.e, moving with a velocity v interacts with a material
which contains N atoms per unit volume. If the density of the material is p gcm-³;
i.
Draw a sketch to represent the scenario described above, taking care to label all the
important features
[4Marks]
Derive an expression for the stopping power of the material
[10 Marks]
ii.
QUESTION 2 (20 MARKS)
a) Explain how Electromagnetic Radiation is produced and categorized
[4 Marks]
b) Give a brief summary of the main modes of interaction of Electromagnetic Radiation with
[6 marks]
matter
c) A y-ray photon of energy E-hv is incident on a Hyper pure Germanium detector material.
Explain how the photon losses energy and hence derive an expression to quantify this
[10 marks]
energy loss.
QUESTION 3 (20 MARKS)
a)
With the help of a neat diagram, describe and explain the main features of typical Voltage
Current characteristics of Gas filled detectors
[8 marks]
b) Using (a) above, compare and contrast the operation of a Geiger and proportional Counter
[6 marks]
A Radiologist and his Assistant are commissioning a new X-Ray machine to be used in
your Laboratory. They report to you that; after they switched on the machine, the Geiger in
the room stopped working. They switched off the machine, and after a few minutes the
counter started working. The machine was switched on and; the Geiger stopped yet again!
The Radiologist is convinced that it has something to do with the X-rays produced. On the
other hand the Assistant is convinced something is wrong with the detector. What do you
think? Do you agree with any or none of the two? If yes why? No, Why? As a Medical
[8 marks]
Physicist, how do you explain this?
Page 2 of 3
C
YO
AN
3 AN
D SI
ICA
ATI
AN
0 ma
Mark
tion

C
QUESTION 4 (20 MARKS)
a) State the factors which influence the efficiency of a typical detector
[5 Marks]
b) Using (a) above, find an expression for the relative efficiency of the detector above
[5 Marks]
c) A detector with a circular window of radius 'a' is placed at a distance 'd' from a point
source;
Give a definition for the solid angle d2 subtended at the source by the detector
[2 marks]
Using (i) above, find an expression for the solid angle subtended by the detector at the
[5 Marks]
source
Explain how the expression in (ii) above changes when the detector is very far from
the source i.e., for a
[3 Marks]
i.
ii.
iii.
H-H₂²
20²
QUESTION 5 (20 MARKS)
a) With the help of a diagram, give a definition of Resolution of a detector and distinguish
between "good" and "poor" resolution
[4Marks]
b) State under what conditions the detector response can be modelled by a Gaussian curve
A
of the form: G(H)= exp
A, Hoand H are related to the detector resolution
[6 Marks]
Assuming a linear relationship between H and N, find an expression for the limiting
Resolution of a detector
[4Marks]
17.
>
Page 3 of 3
and explain what and how the parameters
c)
d) Explain what the Fano factor is, and discuss the conditions which give rise to this factor
and hence explain how it modifies the expression for the limiting Resolution in (c) above
[6 Marks].


\item \textbf{Problem 8: Number Systems} 

TECHNICAL UNIVERSITY OF KENYA
FACULTY OF APPLIED SCIENCE AND TECHNOLOGY SCHOOL OF PHYSICAL SCIENCES AND TECHNOLOGY DEPARTMENT OF PHYSICS AND SPACE SCIENCE MASTER OF SCIENCE: MEDICAL PHYSICS
SEMESTER I EXAMINATIONS
2022 SERIES
SPPM 7113: RADIATION DETECTION AND MEASUREMENT TIME: 2 HOURS
INSTRUCTIONS TO CANDIDATES
Answer question ONE and any TWO questions.
Question one carries 30 marks, all other questions carry 20 marks each
Important information and formulae
V. A*B=B. V× A − A.V× B
7x (@A)=(7xA)-Ax(7)
Vx(AxB=B.V) A-(A. V) B+V. BA-(V.A)B
Generating function for Legendre polynomials
(1-2x h+h2) = P(x)ha
n=0
Q1.
a) State and explain briefly how Radiation is categorised. (4 Marks)
b) Explain briefly what needs to happen in a detector for detection of Radiation to occur (2 Marks).
Page 1 of 3



c) Nuclear medicine normally uses Electromagnetic Radiations. Give a brief discussion of the main processes through which Electromagnetic Radiation (EM) interacts with matter (6 Marks)
d) Using (b) and (c) above explain how you would select a material to give the best results for detecting EM radiation (4 Marks)
e) Gas filled detectors can be used to detect EM radiation. Sketch the Current Voltage characteristics of a gas filled detector and label all the important regions (5 Marks)
f) Most gas filled detectors have concentric cylindrical geometry, with an outer cylinder of radius "b" and an inner wire of radius "a". The outer cylinder is normally kept at ground potential.
i)
ii)
With the help of a neat diagram, derive an expression for the voltage V between the inner and outer electrode (6 Marks)
Using (i) above find an expression for the charge on the central electrode (2 Marks)
Q2. A Geiger counter with the dimensions given in (f) above, is used to monitor y radiation emitted by a specimen of sun dried Tilapia fish. An interaction event in the Geiger occurs at a position a+8 from the central electrode:
a) Describe what happens (2 marks)
b) Assuming that the Geiger can be modelled by a parallel plate capacitor, write an equation to represent the energy density in the detector and explain all the terms in your equation ( 4 marks)
c) Using the work energy theorem, calculate the change in energy density due to electrons at the site of the interaction event (6 Marks)
d) Calculate the change in energy density due to the positive ions (4 marks)
e) Using the results from (c) and (d) above, find an expression for the electric field inside the Geiger (4 Marks)
Q3.
Q4
a) With the help of a neat diagrams, write short notes on the following:
i.
Scintillation detectors (6 marks)
ii.
Solid state detectors (6 marks)
b) Compare and highlight the advantages and disadvantages if any in the use of the two types of detectors (8 marks)
a) Explain what you understand by Geometric factor, and how it influences the efficiency of a detector (5 Marks)
b) A detector with a rectangular window of width 'a' and breadth "b" is placed at a
distance 'd' from a point source S;
Page 2 of 3



i.
ii.
iii.
Divide the detector window into horizontal and vertical strips. Consider an element of area dxdy at a distance "y" from the horizontal. Let R be the distance from the centre
of the element of area to S. Draw the situation described and label all the important features in the sketch (5Marks)
Write an expression and for the differential solid angle d2 (2 Marks)
Using (i) and (ii) above, show that the solid angle subtended by the window at the
source is given by
(8 Marks)
a
1
1
1+
Q5.
a) Explain what is meant by the "Dead time" of a detector (4 Marks)\
b) State the main causes of detector dead time (6 Marks)
c) With the help of neatly labelled diagrams, compare and contrast the Paralysable and Non- Paralysable models of detector "dead time" (4Marks)
d) While on a routine Radiation survey in the Mrima region in Kwale, your field Radiometer headphones record a rapidly increasing series of clicks which reach a
i.
crescendo as you approach a peculiar looking shrub. Suddenly the meter falls silent:
Explain what has happened (3 Marks).
ii. Explain carefully what action you should take if any (3 Marks).


\[109 / 7 = 15 R4\]
\[15 / 7 = 2 R1\]

\[214_{7}\]
\[2*2^{2}*7^{1}+4*7^{0}\]
\[98+7+4 =109d=214_{7}\]



\item \textbf{Problem 19} 

Compton or recoil electrons, Beta particles and Pair Production electrons
QUESTION 1 (30 MARKS)
1.1.
Explain the following terms
(a)
(b)
(c)
Synchrotron radiation and Annihilation quanta
Remote afterloading technique and Neutron activation
[2 marks]
[2 marks]
[3 marks]
1.2. (a) Write a nuclear equation when cesium-137 decays into barium-137. [3 marks] (b) Calculate the decay energy Q- for the ẞ- decay of cesium-137 into barium-137. [5mks] (c) When Fluorodeoxyglucose (FDG) is labelled with radionuclide fluorine-18 it can be injected intravenously into a patient for use in positron emission tomography (PET) functional imaging. What are some of the uses of the FDG PET? [4 marks] Which physical quantities are conserved during a nuclear transformation? [5 marks] Briefly explain directly ionizing radiation and indirectly ionizing radiation. Give examples for each. [6 marks]
1.3.
1.4
2.1.
2.2.
QUESTION 2 (20 MARKS)
Write down the most important characteristics of radionuclides used in external beam radiotherapy.
[4 marks]
The figure below shows an energy level diagram for 228Ra (radium) decaying through a-decay into 222Ra (radon).
Relative mass-energy (MeV)
4.78
a decay (5.4%) Ex2
0.18 --
0
y decay Ey
Rn
Ra
24
a decay (94.6%)
Ex1
(a)
(b)
(c)
Explain the decay scheme shown above as it ends at the ground state. [5 marks] Calculate the decay energy (Q) for the ∞ decay of 228Ra (radium). [5 marks] Calculate the kinetic energy Ex1 of the x-particle.
[3 marks]
(d)
Explain why modern brachytherapy is now carried out with other radionuclides (e.g., iridium-192, cesium-137, iodine-125, etc.) instead of the radium-226 which was very popular in the past century?
[3 marks]
1 | Page



QUESTION 3 (20 MARKS)
3.1. Briefly describe the areas where lonizing radiation is used. [8 marks]
3.2
Electron capture and B' are competing processes. Write down conditions that are required for one process to proceed over the other process. [4 marks]
3.2. Discuss any two ways or mechanisms by which photons can interact with matter that play a very important role in therapeutic as well as in diagnostic medical physics. [8 marks]
QUESTION 4 (20 MARKS)
4.1. Explain the following terms.
4.2.
4.3.
4.4.
(a) the linear attenuation coefficient and Mass attenuation coefficient [2 marks] (b) First half-value layer HVL1, the second half-value layer HVL2 [2 marks]
Derive the functional relationship shown below between the activity of a radioactive substance and time. Explain the terms. [7 marks]
A =
Age-λt
Draw a typical curve for the radioactive substance as given by the expression in Question 4.2. Briefly explain how you will determine the 2. [6 marks]
The half-life of strontium-90, (39Sr), is 28.8 years. Calculate the (a) decay constant and (b) the initial activity of 1.00g of the material. [3 marks]
QUESTION 5 (20 MARKS)
5.1. Describe the term stopping power of a material. [2 marks]
5.2.
Explain the following quantities that are used for the purpose of quantifying radiation. (a) Activity A. and (c) Dose (H) [6 marks]
(b) Kerma (K),
5.3 Write down factors that are considered to choose a radiation beam and dose prescription in treatment of disease with radiation. [4 marks]
5.4 (a) The Bragg curve below shows a plot of the variation of ionization density as a function of distance travelled by an alpha (oc) particle in air. Explain that form of the curve shown below. [6 marks]
(b) What is the practical application of the enhanced ionization in the Bragg peak? [2 marks]
0.0
20
30
Distance travelled (cm)
2 | Page



QUESTION 3 (20 MARKS)
3.1 Electron capture and B are competing processes. Write down conditions that are required
3.2.
3.3.
for one process to proceed over the other process.
[4]
Discuss any two ways or mechanisms by which photons can interact with matter that play a very important role in therapeutic as well as in diagnostic medical physics.
[8]
An X-ray photon beam of photon energy 15 eV is allowed to fall on a metal surface. If the threshold frequency for the metal is 1.2 x 1015 Hz, then find the maximum kinetic energy gained by a photoelectron, the maximum speed of a photoelectron, and the stopping potential corresponding to the maximum kinetic energy.
QUESTION 4 (10 MARKS)
4.1. Explain the following terms.
(a) the linear attenuation coefficient and Mass attenuation coefficient
[8]
[2]
(b) First half-value layer HVL1, the second half-value layer HVL2 and tenth value layer [3]
4.2.
Derive the functional relationship shown below between the activity of a radioactive substance and time. Explain the terms.
[7]
A = Age-λt
4.3.
Draw a typical curve for the radioactive substance as given by the expression in Question 4.2. Briefly explain how you will determine the 2.
[6]
4.4.
If a 0.2-cm thickness of material transmits 25% of a monoenergetic beam of photons, calculate the half-value layer (HVL) of the beam for that material.
[2]
QUESTION 5 (20 MARKS)
5.1.
Describe the term stopping power of a material.
[2]
5.2.
Explain the following quantities that are used for the purpose of quantifying radiation. (b) Kerma (K), (a) Activity A,
and (c) Equivalent Dose (H)
[6]
5.3
Write down factors that are considered to choose a radiation beam and dose prescription in treatment of disease with radiation
[4]
5.4 Draw and explain the Bragg curve. What is the importance of the enhanced ionization in the Bragg peak?
[8]
2 | Page



QUESTION 1 (30 MARKS)
1.1.
Explain the following terms
(a)
(b)
Synchrotron radiation and Annihilation quanta
(c)
Compton or recoil electrons, Beta particles and Pair Production electrons
Remote afterloading technique and Neutron activation
[2 marks]
[2 marks]
1.2. (a) Write a nuclear equation when cesium-137 decays into barium-137. [3 marks]
1.3.
1.4
2.1.
2.2.
[3 marks]
(b) Calculate the decay energy Q- for the B- decay of cesium-137 into barium-137. [5mks] (c) When Fluorodeoxyglucose (FDG) is labelled with radionuclide fluorine-18 it can be injected intravenously into a patient for use in positron emission tomography (PET) functional imaging. What are some of the uses of the FDG PET? [4 marks]
Which physical quantities are conserved during a nuclear transformation? [5 marks] Briefly explain directly ionizing radiation and indirectly ionizing radiation. Give examples for each. [6 marks]
QUESTION 2 (20 MARKS)
Write down the most important characteristics of radionuclides used in external beam radiotherapy.
[4 marks]
88
The figure below shows an energy level diagram for 228Ra (radium) decaying through a-decay into 222Ra (radon).
Relative mass-energy (MeV)
4.78
a decay (5.4%) Ex2
0.18
y decay Ey 222 Rn
226
Ra
a decay (94.6%) Ex1
(a) Explain the decay scheme shown above as it ends at the ground state. [5 marks] (b) Calculate the decay energy (Q) for the ∞ decay of 229Ra (radium). [5 marks] Calculate the kinetic energy Ex of the o-particle. [3 marks]
200
(c)
(d) Explain why modern brachytherapy is now carried out with other radionuclides (e.g., iridium-192, cesium-137, iodine-125, etc.) instead of the radium-226 which was very popular in the past century?
[3 marks]
1 | Page
\[1,4,7,10,13,16,19\]
\[a_{n} = c+dn+fn^{2}\]
\[a_{1} = c+d+f=1\]
\[a_{2} = c+2d+4f=4\]
\[a_{3} = c+3d+9f=7\]
\[c=-2\]
\[d=3\]
\[f=0\]
\[a_{n}=-2+3n\]



\item \textbf{Problem 19} Find a closed form.
\[-1,1,3,5,7,9,11\]
For n is greater than/equal to 0.
\[GUESS a_{n}=2_{n-1}\] 
\[n=0: True -1=0-1=-1\]
Assume for some n >= 1. a(k) = 2(k) - 1
\[=4n-2-2n+3=2n+1=2(n+1)-1\]




\end{enumerate}

 \end{document}
